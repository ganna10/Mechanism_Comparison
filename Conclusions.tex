Tagged Ozone Production Potentials (TOPPs) were used to compare \ce{O_x} production during VOC degradation in reduced chemical mechanisms to the near-explicit MCM v3.2. 
\ce{O_x} production on the first day was generally comparable to that of the MCM v3.2.
Differences in \ce{O_x} production between reduced and near-explicit mechanisms were larger over multi-day scenarios.

Reduced mechanisms produce more \ce{O_x} on the first day from smaller sized degradation products than more detailed mechanisms as VOC are broken into smaller fragments quicker in reduced mechanisms.
The quicker break down of VOC limits the amount of \ce{O_x} produced on subsequent days.

In particular, the second day maximum in \ce{O_x} production during alkane degradation is lower in reduced mechanisms due to the quicker break down of the emitted VOC on the first day.
Thus, predicting \ce{O3} levels downwind of emissions would be underestimated using reduced mechanisms.
Minimising the speed of VOC break down in reduced mechanisms would increase \ce{O_x} production.  
Alkane representation is particularly important for emission control strategies targeting \ce{O3} production at the regional scale as alkanes can produce more \ce{O_x} than more-reactive VOC over multi-day time scales \citep{Butler:2011}.

The lumped intermediate mechanism, CRI v2, produces the most similar amounts of \ce{O_x} to the MCM v3.2 for each VOC with the largest differences obtained for aromatic VOC.
Improved treatment of unsaturated dicarbonyls produced during aromatic VOC degradation would lead to similar \ce{O_x} production to aromatic VOC in the MCM.

CBM-IV and CB05 (lumped structure mechanisms) produce similar amounts of \ce{O_x} to the MCM v3.2 during alkane degradation.
Aromatic and unsaturated VOC in CBM-IV and CB05 produce less \ce{O_x} than the MCM v3.2 due to either a more rapid break down of the VOC or inadequate representation of the emitted VOC.

The \ce{O_x} produced from VOC represented by lumped species in lumped molecule mechanisms (MOZART-4, RADM2, RACM, RACM2) varies the most from the MCM v3.2.
More lumped species representing VOC with similar functionality and OH-oxidation rate constants would lead to more representative \ce{O_x} production from individual VOC.
The increased speciation of aromatic VOC in RACM2 leads to more representative \ce{O_x} production from aromatic VOC than any other lumped molecule mechanism.

The tagging approach and TOPP calculation over multi-day scenarios is a useful tool for comparing the chemical degradation pathways in chemical mechanisms. 
This tagging approach could be used to investigate the effects on \ce{O_x} production by the treatment of degradation chemistry in mechanisms under using more realistic \ce{NO_x} conditions or a range of \ce{NO_x} conditions, simulating \ce{NO_x}-sensitive and VOC-sensitive conditions.
