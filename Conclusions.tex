Tagged Ozone Production Potentials (TOPPs) were used to compare \ce{O_x} production during VOC degradation in reduced chemical mechanisms to the near-explicit MCM v3.2. 
First day mixing ratios of \ce{O3} are similar to the MCM v3.2 for most mechanisms; the \ce{O3} mixing ratios in RACM were much lower than the MCM v3.2 due to a lack of \ce{O_x} production from the degradation of aromatic VOC.
Thus, RACM may not be the appropriate chemical mechanism when simulating atmospheric conditions having a large fraction of aromatic VOC.

The lumped intermediate mechanism, CRI v2, produces the most similar amounts of \ce{O_x} to the MCM v3.2 for each VOC.
The largest differences between \ce{O_x} production in CRI v2 and \mbox{MCM v3.2} were obtained for aromatic VOC, however overall these differences were much lower than any other reduced mechanism.
Thus, when developing chemical mechanisms the technique of using lumped intermediate species whose degradation are based upon more detailed mechanism should be considered.

Many VOC are broken down into smaller sized degradation products faster on the first day in reduced mechanisms than the MCM v3.2 leading to lower amounts of larger sized degradation products that can further degrade and produce \ce{O_x}.
Thus, many VOC in reduced mechanisms have a lower maximum of \ce{O_x} production and produce lower amounts of \ce{O_x} after the first day than the MCM v3.2.
This lower \ce{O_x} production from many VOC in reduced mechanisms leads to lower \ce{O3} mixing ratios compared to the MCM v3.2.

Alkanes produce maximum \ce{O3} on the second day of simulations and this maximum is lower in reduced mechanisms than the MCM v3.2 due to the faster break down of alkanes into smaller sized degradation products on the first day.
The lower maximum in \ce{O3} production during alkane degradation in reduced mechanisms would lead to an underestimation of the \ce{O3} levels downwind of VOC emissions, and an underestimation of the VOC contribution to tropospheric background \ce{O3} when using reduced mechanisms in regional or global modelling studies.

This study has determined the maximum potential of VOC represented in reduced mechanisms to produce \ce{O3}, this potential may not be reached as ambient \ce{NO_x} conditions may not induce \ce{NO_x}-VOC-sensitive chemistry.
Moreover, the maximum potential of the VOC to produce \ce{O3} may not be reached when using these reduced mechanisms in 3-D models due to the influence of additional processes, such as mixing and meteorology.
Future work shall examine the extent to which the maximum potential of VOC to produce \ce{O3} in reduced chemical mechanisms is reached using ambient \ce{NO_x} conditions and including processes found in 3-D models.
