Tagged Ozone Production Potentials (TOPPs) were used to compare NMVOC degradation in reduced chemical mechanisms to the near-explicit MCM v3.2. 
First day TOPP values were generally comparable to those of the MCM v3.2.
Differences in \ce{O_x} production between reduced and near-explicit mechanisms had a larger impact over multi-day scenarios.  
The largest discrepancies from the MCM v3.2 were obtained for aromatic VOCs and VOCs represented by lumped mechanism species.

Different NO emissions were required to obtain \ce{NO_x}-VOC-sensitive conditions in each mechanism.
NO emissions were determined using net radical to \ce{NO_x} yield thus different atmospheric regimes are represented in different mechanisms.
The impacts on \ce{O_x} production shall be investigated in future work.

Reduced mechanisms were unable to reach the \ce{O_x} levels of near-explicit mechanisms.
This is mainly due to increased rates of reactive carbon loss in less-explicit mechanisms.
For some VOCs, incorrect chemistry or inadequate representation in reduced mechanisms also led to reduced \ce{O_x} production.

The tagging approach and TOPP calculation over multi-day scenarios have proven to be a useful tool in comparing the chemical degradation pathways in chemical mechanisms. 
This approach could be used to further investigate how mechanisms treat degradation chemistry under more realistic \ce{NO_x} conditions or a range of \ce{NO_x} conditions, simulating different atmospheric regimes. 
