Tagged Ozone Production Potentials (TOPPs) were used to compare \ce{O_x} production during VOC degradation in reduced chemical mechanisms to the near-explicit MCM v3.2. 
First day mixing ratios of \ce{O3} are similar to the MCM v3.2 for most mechanisms, the \ce{O3} mixing ratios in RACM were much lower than the MCM v3.2 due to a lack of \ce{O_x} production from the degradation of aromatic VOC.

Unsaturated and aromatic VOC have the largest differences between mechanism on the first day of simulation.
Many unsaturated VOC produce similar amounts of \ce{O_x} between mechanisms, the \ce{O_x} production from unsaturated VOC that are not represented by explicit mechanisms species has the largest intra-mechanism differences.
The \ce{O_x} produced during the degradation of aromatic VOC shows the largest differences between the mechanisms, improved degradation chemistry and more mechanism species representing aromatic VOC would improve the representation of aromatic degradation.
Aromatic VOC are broken down quicker in reduced mechanism than in the MCM v3.2 thus producing less \ce{O_x} in reduced mechanisms.

Alkanes have largest differences in \ce{O_x} production on the second day of simulations.
In lumped molecule mechanisms (MOZART-4, RADM2, RACM, RACM2), alkanes produce more \ce{O_x} on the first day from smaller sized degradation products than more detailed mechanisms as VOC are broken into smaller fragments quicker in reduced mechanisms.
The quicker break down of VOC limits the amount of \ce{O_x} produced on subsequent days as there are less of the larger degradation products available for further degradation and hence more \ce{O_x} production.
Alkane representation is particularly important for emission control strategies targeting \ce{O3} production at the regional scale as alkanes can produce more \ce{O_x} than more-reactive VOC over multi-day time scales \citep{Butler:2011}.

The lumped intermediate mechanism, CRI v2, produces the most similar amounts of \ce{O_x} to the MCM v3.2 for each VOC with the largest differences obtained for aromatic VOC.
Improved treatment of unsaturated dicarbonyls produced during aromatic VOC degradation would lead to similar \ce{O_x} production to aromatic VOC in the MCM.

The lumped structure mechanisms (CBM-IV and CB05) produce similar amounts of \ce{O_x} to the MCM v3.2 on the first day during alkane degradation but also produce less \ce{O_x} on the second day than the MCM v3.2.
Aromatic and unsaturated VOC in CBM-IV and CB05 produce less \ce{O_x} than the MCM v3.2 due to either a more rapid break down of the VOC or inadequate representation of the emitted VOC.

The \ce{O_x} produced from VOC represented by lumped species in lumped molecule mechanisms (MOZART-4, RADM2, RACM, RACM2) varies the most from the MCM v3.2.
More lumped species representing VOC with similar functionality and OH-oxidation rate constants would lead to more representative \ce{O_x} production from individual VOC.

The tagging approach and TOPP calculation over multi-day scenarios is a useful tool for comparing the chemical degradation pathways in chemical mechanisms. 
This tagging approach could be used to investigate the effects on \ce{O_x} production by the treatment of degradation chemistry in mechanisms under using more realistic \ce{NO_x} conditions.
