Tagged Ozone Production Potentials (TOPPs) were used to compare \ce{O_x} production during VOC degradation in reduced chemical mechanisms to the near-explicit MCM v3.2. 
\ce{O_x} production on the first day was generally comparable to that of the MCM v3.2.
Differences in \ce{O_x} production between reduced and near-explicit mechanisms were larger over multi-day scenarios, with reduced mechanisms producing lower amounts of \ce{O_x}.

Reduced mechanisms produce more \ce{O_x} from smaller sized degradation products than more explicit mechanisms as VOC are broken into smaller fragments quicker in reduced mechanisms.
The quicker break down of VOC limits \ce{O_x} production during VOC degradation in reduced mechanisms limiting the amount of \ce{O_x} produced on subsequent days.
Minimising the rate of VOC break down in reduced mechanisms would increase the \ce{O_x} production in reduced mechanisms.

The lumped intermediate mechanism, CRI v2, produces the most similar amounts of \ce{O_x} to the MCM v3.2 with the largest differences obtained for aromatic VOC.
Improved treatment of unsaturated dicarbonyls produced during aromatic VOC degradation would lead similar \ce{O_x} production to aromatic VOC in the MCM.

The CBM-IV and CB05 (lumped structure mechanisms) produce the lowest amounts of \ce{O_x} for all VOC from the mechanisms compared in this study.
Low \ce{O_x} is produced due to low OH availability which result from increased \ce{HNO3} deposition rates thus inhibiting VOC oxidation rates.
The lumped structure approach assumes the \ce{O_x} production of VOC with the same functionality is a factor of the carbon number leading to a slower rate of increase in \ce{O_x} production with carbon number.
The increase in \ce{O_x} production with carbon number in more explicit mechanisms is a multiple of the carbon number thus larger amounts of \ce{O_x} are produced in more explicit mechanisms.

The \ce{O_x} produced from VOC represented by lumped species in lumped molecule mechanisms (MOZART-4, RADM2, RACM, RACM2) varies the most from the MCM v3.2.
Lumped species generalise the degradation of the VOC they represent which leads to non-representative \ce{O_x} production of the individual VOC.
More lumped species representing a limit range of VOC functionality would improve the \ce{O_x} production of the individual VOC.

The tagging approach and TOPP calculation over multi-day scenarios is a useful tool for comparing the chemical degradation pathways in chemical mechanisms. 
This tagging approach could be used to investigate the effects on \ce{O_x} production by the treatment of degradation chemistry in mechanisms under using more realistic \ce{NO_x} conditions or a range of \ce{NO_x} conditions, simulating different atmospheric regimes. 
