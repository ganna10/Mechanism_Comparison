Tagged Ozone Production Potentials (TOPPs) were used to compare the NMVOC degradation in reduced chemical mechanisms to the near-explicit MCM v3.2. 
The first day NMVOC TOPP values in all mechanisms are generally comparable to those of the MCM v3.2. 
TOPP value time series over multi-day scenarios have larger differences. 
Thus differences in \ce{O_x} production between reduced mechanisms and near-explicit mechanisms has a larger impact over multi-day scenarios. 

The largest discrepancies from the MCM v3.2 are the zero TOPP values of toluene and m-xylene in RACM. 
These are due to the net \ce{O_x} consumption from ozonolysis of aromatic-OH adduct species.

Attributing \ce{O_x} production budgets to the number of carbon atoms of each degradation species showed that reduced mechanisms break down NMVOCs faster than near-explicit mechanisms. 
This leads to reduced mechanisms being unable to reach the \ce{O_x} levels of near-explicit mechanisms.

Radical and PAN family production and consumption budgets were analysed in detail through tagging. 
Photolysis is the main radical souce in the MCM v3.2 whilst reduced mechanisms also produce radicals by thermal reactions. 
This leads to larger net radical production than in MCM v3.2.

Alkoxy radical decomposition is a PAN family production source in the MCM v3.2 that is substituted by thermal reactions or photolysis in reduced mechanisms. 
This leads to different PAN family chemistry in reduced mechanisms which impacts both radical and \ce{O_x} production.

The tagging approach and TOPP calculation over multi-day scenarios have proven to be a useful tool in comparing the chemical degradation pathways in chemical mechanisms. 
This approach could be used to further investigate how mechanisms treat degradation chemistry under more realistic \ce{NO_x} conditions or a range of \ce{NO_x} conditions, simulating different atmospheric regimes. 
