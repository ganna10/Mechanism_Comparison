%what is tropospheric O3 & its importance
Ground-level ozone (\ce{O3}) is both an air pollutant and a short-lived climate forcing pollutant (SLCP) that is detrimental to human health and crop growth \citep{AQEU:2013}. 
\ce{O3} is a secondary pollutant as it is not directly emitted but produced from the reactions of volatile organic compounds (VOCs) and nitrogen oxides (\ce{NO_x} = NO + \ce{NO2}) in the presence of sunlight \citep{Atkinson:2000}.

Emissions of \ce{O3} precursors have been steadily decreasing over Europe, however $98$\% of Europe's urban population are exposed to levels that exceed the World Health Organization (WHO) air quality guideline of $100$ $\mu$g/m$^3$ over an $8$-hour mean \citep{WHO:2006}. 
These exceedances are the result of local and regional emissions of \ce{O3} precursor gases, their intercontinental transport and the non-linear relationship of \ce{O3} concentrations on \ce{NO_x} and VOC levels \citep{AQEU:2013}.

Effective emission reduction strategies require accurate predictions of \ce{O3} concentrations using chemical transport models (CTMs). 
This requires adequate representation of gas-phase chemistry in the chemical mechanism used by the CTM. 
All mechanisms have some level of simplification due to the computational resourses needed for gas-phase chemistry calculations. 
These simplification levels arise from different modelling studies scopes -- global CTMs require more computationally efficient mechanisms than boxmodels. 
This study compares the impacts of different simplified mechanisms on \ce{O3} production levels focusing on the VOC degradation product chemistry representations.

%NOx-VOC-chemistry and use of Ox family
\ce{O3} is rapidily produced and destroyed in the null photochemical cycle \reactionref{r:NO_O3}--\reactionref{r:O2_O3P}. 
\begin{reactionlist}
    \reactionitem{\ce{NO + O3}}{\ce{NO2 + O2}}{new}{r:NO_O3}
    \reactionitem{\ce{NO2 + h$\nu$}}{\ce{NO + O(^3P)}}{new}{r:NO2_hv}
    \reactionitem{\ce{O2 + O(^3P) + M}}{\ce{O3 + M}}{new}{r:O2_O3P}
\end{reactionlist}
NO and \ce{NO2} reach a near-steady state via \reactionref{r:NO_O3} and \reactionref{r:NO2_hv} which is disturbed in two cases. 
Firstly, when \ce{O3} is removed via \reactionref{r:NO_O3} during night-time or near large NO sources and secondly, when \ce{O3} is produced through VOC--\ce{NO_x} chemistry \citep{Sillman:1999}.

The odd oxygen family, \ce{O_x}, is often used to remove the influence of rapid null cycles on \ce{O3} budgets \citep{Seinfeld:2006}. 
Here, \ce{O_x} is defined to include \ce{O3}, \ce{NO2}, \ce{O(^3P)}, \ce{O(^1D)} and other chemical species that are involved in fast photochemical cycles with \ce{O3} and \ce{NO2}, such as PAN (peroxy acetyl nitrate).

Emitted VOCs (RH) are typically oxidised in the troposphere by the hydoxyl radical (OH) forming peroxy radicals (\ce{RO2}) in the presence of \ce{O2} \reactionref{r:RH_OH}. 
Alkoxy radicals (RO) are produced from the reaction of \ce{RO2} with NO \reactionref{r:RO2_NO} and react quickly with \ce{O2} producing a carbonyl species (R$^{\prime}$CHO) and a hydroperoxy radical (\ce{HO2}), \reactionref{r:RO_O2}.
Both \ce{RO2} and \ce{HO2} radicals react with NO to produce \ce{NO2} (\reactionref{r:RO2_NO} and \reactionref{r:HO2_NO}).
This leads directly to \ce{O3} production via \reactionref{r:NO2_hv} and \reactionref{r:O2_O3P}. 
Thus the amount of \ce{O3} produced from the degradation of a VOC is directly related to the number of NO to \ce{NO2} conversions by \ce{RO2} and \ce{HO2} radicals \citep{Atkinson:2000}.
\begin{reactionlist}
    \reactionitem{\ce{RH + OH + O2}}{\ce{RO2 + H2O + O2}}{new}{r:RH_OH}
    \reactionitem{\ce{RO2 + NO}}{\ce{RO + NO2}}{new}{r:RO2_NO}
    \reactionitem{\ce{RO + O2}}{\ce{R$^{\prime}$CHO + HO2}}{new}{r:RO_O2}
    \reactionitem{\ce{HO2 + NO}}{\ce{OH + NO2}}{new}{r:HO2_NO}
\end{reactionlist}

% NOx and VOC sensitive regimes
This chemistry leads to \ce{O3} concentration being a non-linear function of \ce{NO_x} and VOCs concentrations. 
Three distinct atmospheric regimes with respect to \ce{O3} production can be defined \citep{Jenkin:2000}. 
In the \ce{NO_x}-sensitive regime, VOC concentrations are much higher than those of \ce{NO_x} and \ce{O3} production is dependant on \ce{NO_x} concentrations. 
On the other hand, when \ce{NO_x} concentrations are much higher than those of VOCs, the amount of \ce{O3} produced is determined by VOC concentrations. 
This is the VOC-sensitive regime. 
Finally, in the \ce{NO_x}-VOC-sensitive regime there is maximal \ce{O3} production which is controlled by both VOC and \ce{NO_x} concentrations.

These different atmospheric regimes give rise to distinct radical removal mechanisms \citep{Kleinman:1991}. 
In the \ce{NO_x}-sensitive regime, the radicals concentration is very high relative to \ce{NO_x} causing radical removal by either radical combination reactions \reactionref{r:RO2_HO2} or bimolecular destruction reactions \reactionref{r:HO2_OH} \citep{Kleinman:1994}.
\begin{reactionlist}
    \reactionitem{\ce{RO2 + HO2}}{\ce{ROOH + O2}}{new}{r:RO2_HO2}
    \reactionitem{\ce{HO2 + OH}}{\ce{H2O + O2}}{new}{r:HO2_OH}
\end{reactionlist}
Whereas in the VOC-sensitive regime, high \ce{NO_x} concentrations lead to radical removal by reaction with \ce{NO2}. 
These reactions lead to nitric acid \reactionref{r:NO2_OH} and PAN species \reactionref{r:RC(O)O2_NO2}.
\begin{reactionlist}
    \reactionitem{\ce{NO2 + OH}}{\ce{HNO3}}{new}{r:NO2_OH}
    \reactionitem{\ce{RC(O)O2 + NO2}}{\ce{RC(O)O2NO2}}{new}{r:RC(O)O2_NO2}
\end{reactionlist}
In the \ce{NO_x}-VOC-sensitive regime, radical and \ce{NO_x} amounts are comparable leading to no dominant radical removal mechanism. 

Accurately representing the chemistry of the separate regimes is one challenge when developing a chemical mechanism.
Some chemical mechanisms were developed to represent only polluted urban conditions leading to non-representative VOC-sensitive chemistry.

%OPP 
VOCs impact \ce{O3} production in different ways due to their diverse reaction rates and degradation pathways. 
Ozone Production Potentials (OPP) quantify the effect of individual VOCs on \ce{O3} production. 
OPPs are typically calculated through incremental reactivity (IR) studies using photochemical models. 
The concentration of a particular VOC is changed by a known increment and the resulting change in \ce{O3} production is compared to that of a standard VOC mixture. 

Examples of IR scales are the Maximal Incremental Reactivity (MIR) and Maximum Ozone Incremental Reactivity (MOIR) scales in \citet{Carter:1994}, as well as the Photochemical Ozone Creation Potential (POCP) scale of \citet{Derwent:1996} and \citet{Derwent:1998}. 
Different \ce{NO_x} conditions were used when calculating these IR scales thus calculating the OPPs of VOCs in different atmospheric regimes.

OPPs have been calculated using a tagging approach in \citet{Butler:2011}, resulting in the Tagged Ozone Production Potential (TOPP). 
TOPP values were calculated for the maximum potential \ce{O_x} produced hence in the \ce{NO_x}-VOC-sensitive regime.
Tagging a chemical mechanism involves labelling all VOC organic degradation products with the name of the emitted VOC.
Thus all degradation pathways of the emitted VOC are tracked, enabling \ce{O_x} production to be attributed to the parent VOC.
This study exploits the tagging approach to compare the VOC degradation chemistry of different chemical mechanisms. 

%chemical mechanisms are how they have to represent O3 chemistry
A near-explicit mechanism such as the Master Chemical Mechanism (MCM) \citep{Jenkin:2003, Saunders:2003, Bloss:2005} includes a large amount of mechanistic details ($\sim$ $12,000$ reactions). 
This makes the MCM ideal as a reference mechanism for comparing chemical mechanisms. 
The latest version, MCM v3.2, \citep{MCM_Site} is the reference mechanism in this study.

Further simplification can be achieved by using lumped structure or lumped molecule approachs \citep{Dodge:2000}. 
Lumped structure mechanisms speciate VOCs by the carbon bonds of the emitted VOC. 
Whilst in lumped molecule mechanisms, VOCs are represented by a specific organic or mechanism species representing a number of VOCs. 
Lumped molecule mechanisms use diverse approaches to create mechanism species representing emitted VOCs.  
Another simplification technique is lumping the intermediate species produced during VOC degradation rather than lumping the emitted VOCs.

%other chemical mechanism comparison studies
Chemical mechanism comparison studies consider modelled \ce{O3} concentration time series over varying VOC and \ce{NO_x} concentrations. 
Examples of such studies are \citet{Dunker:1984}, \citet{Kuhn:1998} and \citet{Emmerson:2009}.
A common outcome is that the largest discrepancies in \ce{O3} concentration times series arise when modelling urban rather than rural conditions.
These discrepancies are due to the treatment of radical production, organic nitrate and night-time chemistry.

IR scales have also been used to compare mechanisms.
In \citet{Derwent:2010}, the near-explicit mechanisms MCM v3.1 and SAPRC-07 were compared using first-day POCP values calculated under VOC-sensitive conditions. 
In general, the POCP values were comparable between the mechanisms. 

OPPs are a useful comparison tool as they relate \ce{O3} production to a single value. 
\citet{Butler:2011} compares first day TOPP values of a VOC mixture to the corresponding MIR, MOIR and POCP values.
TOPP values were most comparable to MOIR and POCP values, whilst MIR values for reactive VOC were overestimated.

In this study, TOPP values calculated for a number of VOCs using different mechanisms are compared to those calculated with the MCM v3.2. 
Processes influencing \ce{O_x} production in the mechanisms are compared by exploiting the tagging approach used to calculate the TOPP.
In particular, differences in the treatment of organic degradation products and subsequent chemistry is emphasized.

%paper outline
The compared chemical mechanisms and the comparison methodology are described in Section \ref{s:methodology}. 
Section \ref{s:overall_results} presents the comparison of the first-day TOPP values and their time series. 
Detailed analysis of \ce{O_x} production chemistry is correlated in Section \ref{s:detailed_results}. 
Finally conclusions and further work are outlined in \mbox{Section \ref{s:conclusions}}.  
