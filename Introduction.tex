%what is tropospheric O3 & its importance
Ground-level ozone (\ce{O3}) is both an air pollutant and a short-lived climate forcing pollutant (SLCP) that is detrimental to human health and crop growth \citep{AQEU:2014}. 
\ce{O3} is a secondary pollutant as it is not directly emitted but produced from the reactions of volatile organic compounds (VOC) and nitrogen oxides (\ce{NO_x} = NO + \ce{NO2}) in the presence of sunlight \citep{Atkinson:2000}.

Concentrations of background \ce{O3} have increased from preindustrial times due to increases in anthropogenic emissions of \ce{O3} precursors \citep{HTAP:2010}.
Despite decreases in emissions of \ce{O3} precursors over Europe, $98$\% of Europe's urban population are exposed to levels exceeding the World Health Organization (WHO) air quality guideline of \mbox{$100$ $\mu$g/m$^3$} over an \mbox{$8$-hour} mean \citep{WHO:2006}. 
These exceedances result from local and regional \ce{O3} precursor gas emissions, their intercontinental transport and the non-linear relationship of \ce{O3} concentrations on \ce{NO_x} and VOC levels \citep{AQEU:2014}.

Effective emission reduction strategies depend on accurate predictions of \ce{O3} concentrations using chemical transport models (CTMs). 
These predictions require adequate representation of gas-phase chemistry in the chemical mechanism used by the CTM. 
All mechanisms simplify gas-phase chemistry due to the computational resourses needed to solve the ordinary differential equations.
The modelling study scope determines the level of simplification --- global CTMs require more computationally efficient mechanisms than boxmodels. 
This study compares the impacts of different simplificaton levels on \ce{O3} production chemistry focusing on the representation of VOC degradation products.

%NOx-VOC-chemistry
The photochemical cycle (\reactionref{r:NO_O3}--\reactionref{r:O2_O3P}) rapdily produces and destroys \ce{O3}. 
\begin{reactionlist}
    \reactionitem{\ce{NO + O3}}{\ce{NO2 + O2}}{new}{r:NO_O3}
    \reactionitem{\ce{NO2 + h$\nu$}}{\ce{NO + O(^3P)}}{new}{r:NO2_hv}
    \reactionitem{\ce{O2 + O(^3P) + M}}{\ce{O3 + M}}{new}{r:O2_O3P}
\end{reactionlist}
NO and \ce{NO2} reach a near-steady state via \reactionref{r:NO_O3} and \reactionref{r:NO2_hv} which is disturbed in two cases. 
Firstly, via \ce{O3} removal (deposition, \reactionref{r:NO_O3} during night-time or near large NO sources) and secondly, when \ce{O3} is produced through VOC--\ce{NO_x} chemistry \citep{Sillman:1999}.

Emitted VOCs (RH) are oxidised in the troposphere by the hydoxyl radical (OH) forming peroxy radicals (\ce{RO2}) in the presence of \ce{O2} \reactionref{r:RH_OH}. 
In high-\ce{NO_x} conditions typical of urban environments, \ce{RO2} react with NO \reactionref{r:RO2_NO} to form alkoxy radicals (RO).
The RO react quickly with \ce{O2} \reactionref{r:RO_O2} producing a carbonyl species (R$^{\prime}$CHO) and a hydroperoxy radical (\ce{HO2}).
Both \ce{RO2} and \ce{HO2} react with NO to produce \ce{NO2} (\reactionref{r:RO2_NO} and \reactionref{r:HO2_NO}).
\ce{O3} is then produced via \reactionref{r:NO2_hv} and \reactionref{r:O2_O3P}. 
Thus the amount of \ce{O3} produced from VOC degradation is directly related to the number of NO to \ce{NO2} conversions by \ce{RO2} and \ce{HO2} radicals \citep{Atkinson:2000}.
\begin{reactionlist}
    \reactionitem{\ce{RH + OH + O2}}{\ce{RO2 + H2O + O2}}{new}{r:RH_OH}
    \reactionitem{\ce{RO2 + NO}}{\ce{RO + NO2}}{new}{r:RO2_NO}
    \reactionitem{\ce{RO + O2}}{\ce{R$^{\prime}$CHO + HO2}}{new}{r:RO_O2}
    \reactionitem{\ce{HO2 + NO}}{\ce{OH + NO2}}{new}{r:HO2_NO}
\end{reactionlist}

% NOx and VOC sensitive regimes
This chemistry leads to \ce{O3} concentration being a non-linear function of \ce{NO_x} and VOC concentrations.
Three atmospheric regimes with respect to \ce{O3} production can be defined \citep{Jenkin:2000}. 
In the \ce{NO_x}-sensitive regime, VOC concentrations are much higher than those of \ce{NO_x} and \ce{O3} production depends on \ce{NO_x} concentrations. 
On the other hand, when \ce{NO_x} concentrations are much higher than those of VOCs (VOC-sensitive regime), the VOC concentrations determine the amount of \ce{O3} produced.
Finally, the \ce{NO_x}-VOC-sensitive regime produces maximal \ce{O3} which is controlled by both VOC and \ce{NO_x} concentrations.

These atmospheric regimes lead to distinct mechanisms for radical removal \citep{Kleinman:1991}. 
In the \ce{NO_x}-sensitive regime, radical concentrations are high relative to \ce{NO_x} causing radical removal by radical combination reactions \reactionref{r:RO2_HO2} and bimolecular destruction reactions \reactionref{r:HO2_OH} \citep{Kleinman:1994}.
\begin{reactionlist}
    \reactionitem{\ce{RO2 + HO2}}{\ce{ROOH + O2}}{new}{r:RO2_HO2}
    \reactionitem{\ce{HO2 + OH}}{\ce{H2O + O2}}{new}{r:HO2_OH}
\end{reactionlist}
Whereas in the VOC-sensitive regime, radicals are removed by reacting with \ce{NO2}.
These reactions lead to nitric acid \reactionref{r:NO2_OH} and PAN species \reactionref{r:RC(O)O2_NO2}.
\begin{reactionlist}
    \reactionitem{\ce{NO2 + OH}}{\ce{HNO3}}{new}{r:NO2_OH}
    \reactionitem{\ce{RC(O)O2 + NO2}}{\ce{RC(O)O2NO2}}{new}{r:RC(O)O2_NO2}
\end{reactionlist}
There is no dominant radical removal mechanism in the \ce{NO_x}-VOC-sensitive regime as radical and \ce{NO_x} amounts are comparable.

%OPP 
Individual VOC impact \ce{O3} production differently due to their diverse reaction rates and degradation pathways. 
These impacts can be quantified using Ozone Production Potentials (OPP).
OPPs are calculated through incremental reactivity (IR) studies using photochemical models. 
In IR studies, VOC concentrations are changed by a known increment and the resulting change in \ce{O3} production is compared to that of a standard VOC mixture. 

Examples of IR scales are the Maximal Incremental Reactivity (MIR) and Maximum Ozone Incremental Reactivity (MOIR) scales in \citet{Carter:1994}, as well as the Photochemical Ozone Creation Potential (POCP) scale of \citet{Derwent:1996} and \citet{Derwent:1998}. 
These IR scales were determined using a separate \ce{NO_x} conditions thus calculating the OPPs in different atmospheric regimes.

OPPs have been calculated using a ``tagging'' approach in \citet{Butler:2011}, leading to the Tagged Ozone Production Potential (TOPP). 
The TOPP determines the maximum potential of a VOC to produce \ce{O3}, thus \ce{NO_x} conditions leading to \ce{NO_x}-VOC-sensitive chemistry were used.
The tagging approach involves labelling all organic degradtion products produced during VOC degradation with the name of the emitted VOC.
Tagging enables VOC source attribution of \ce{O3} production and a detailed insight into VOC degradation chemistry.
We exploit the tagging approach to compare the VOC degradation chemistry of different chemical mechanisms. 

%chemical mechanisms are how they have to represent O3 chemistry
A near-explicit mechanism, such as the Master Chemical Mechanism (MCM) \citep{Jenkin:2003, Saunders:2003, Bloss:2005}, includes a large amount of mechanistic details ($\sim$ $12,000$ reactions). 
This level of detail makes the MCM ideal as a reference for comparing chemical mechanisms.
The latest version, MCM v3.2 \citep{MCM_Site}, is the reference mechanism in this study.

Mechanisms are further simplified by using lumped structure or lumped molecule approachs \citep{Dodge:2000}. 
Lumped structure mechanisms speciate VOC by the carbon bonds of the emitted VOC.
Examples of the lumped structure approach are CBM-IV \citep{Gery:1989} and CB05 \citep{Yarwood:2005}.
Lumped structure mechanisms represent VOC by either an explicit or a lumped species that represents many VOC. 
Lumping emitted VOC into mechanism species may be achieved by similar functionality (MOZART-4 \citep{Emmons:2010}) or OH-reactivity (RADM2 \citep{Stockwell:1990}, RACM \citep{Stockwell:1997}, RACM2 \citep{Goliff:2013}).
A different lumped molecule approach is used in the CRI \citep{Jenkin:2008}, where the intermediate species produced during VOC degradation are lumped not the emitted VOC.

%other chemical mechanism comparison studies
Chemical mechanism comparison studies consider modelled time series of \ce{O3} concentrations over varying VOC and \ce{NO_x} concentrations. 
Examples of such studies are \citet{Dunker:1984}, \citet{Kuhn:1998} and \citet{Emmerson:2009}.
A common outcome is that the largest discrepancies in the time series of \ce{O3} concentrations arise when modelling urban rather than rural conditions.
These discrepancies have been attributed to the treatment of radical production, organic nitrate and night-time chemistry in mechanisms.

IR scales have also been used to compare mechanisms.
In \citet{Derwent:2010}, the near-explicit MCM v3.1 and SAPRC-07 mechanisms were compared using first-day POCP values calculated under VOC-sensitive conditions. 
The POCP values were comparable between the mechanisms. 

OPPs are a useful comparison tool as they relate \ce{O3} production to a single value. 
\citet{Butler:2011} compares first day TOPP values to the corresponding MIR, MOIR and POCP values.
TOPP values were most comparable to MOIR and POCP values.

In this study, TOPP values calculated for VOC using a number of mechanisms are compared to those calculated with the MCM v3.2. 
Differences in \ce{O3} production are explained by the treatment of VOC degradation.

%paper outline
The comparison methodology is described in \mbox{Section \ref{s:methodology}}, overall mechanism comparison results are detailed in \mbox{Section \ref{s:overall_results}} with detailed \ce{O3} production analysis in \mbox{Section \ref{s:detailed_results}} and finally, results are presented in \mbox{Section \ref{s:conclusions}}.
