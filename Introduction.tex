%what is tropospheric O3 & its importance
Ground-level ozone (\ce{O3}) is both an air pollutant and a short-lived climate forcing pollutant (SLCP) that is detrimental to human health and crop growth \citep{AQEU:2013}. 
\ce{O3} is a secondary pollutant as it is not directly emitted but produced from the reactions of volatile organic compounds (VOCs) and nitrogen oxides (\ce{NO_x} = NO + \ce{NO2}) in the presence of sunlight \citep{Atkinson:2000}.

Emissions of \ce{O3} precursors have been steadily decreasing over Europe however $98$\% of Europe's urban population are exposed to levels that exceed the World Health Organization (WHO) air quality guideline of $100$ $\mu$g/m$^3$ over an $8$-hour mean \citep{WHO:2006}. 
These exceedances are the result of intercontinental transport of emitted \ce{O3} precursor gases and the non-linear relationship of \ce{O3} concentrations on \ce{NO_x} and VOC levels \citep{AQEU:2013}.

Effective emission reduction strategies require accurate predictions of \ce{O3} concentrations using chemical transport models (CTMs). 
This requires adequate representation of gas-phase chemistry in the chemical mechanism used by the CTM. 
Many chemical mechanisms are available, these were developed using differing streamlining approaches to gas-phase chemistry resulting in varying levels of simplification. 
This study compares the impacts of these different simplified representations on \ce{O3} production levels with a focus on the treatment of VOC degradation product chemistry by the chemical mechanism.

There is also a need to improve the accuracy of modelling studies from a scientific perspective. 
\citet{Abbatt:2014} stresses the need to focus on the three fundamentals of atmospheric chemistry: observations, laboratory and modelling. 

%NOx-VOC-chemistry and use of Ox family
\ce{O3} is primarily produced and destroyed in the fast photochemical NO--\ce{NO2}--\ce{O3} cycle \reactionref{r:NO_O3}--\reactionref{r:O2_O3P}. 
\begin{reactionlist}
    \reactionitem{\ce{NO + O3}}{\ce{NO2 + O2}}{new}{r:NO_O3}
    \reactionitem{\ce{NO2 + h$\nu$}}{\ce{NO + O(^3P)}}{new}{r:NO2_hv}
    \reactionitem{\ce{O2 + O(^3P) + M}}{\ce{O3 + M}}{new}{r:O2_O3P}
\end{reactionlist}
NO and \ce{NO2} reach a near-steady state via \reactionref{r:NO_O3} and \reactionref{r:NO2_hv} which is disturbed in two cases. 
Firstly, when \ce{O3} is removed via \reactionref{r:NO_O3} which occurs during night-time or near large NO sources -- \ce{NO_x}-titration. 
Secondly, when \ce{O3} is produced through VOC--\ce{NO_x} chemistry \citep{Sillman:1999}.

The odd oxygen family, \ce{O_x}, is used to remove the influence of \ce{NO_x}-titration on \ce{O3} chemistry. 
In this study \ce{O_x} is defined to include \ce{O3}, \ce{NO2}, \ce{O(^3P)}, \ce{O(^1D)} and other chemical species that are involved in fast photochemical cycles with \ce{O3} and \ce{NO2}. 

Emitted VOCs (RH) are typically oxidised in the atmosphere by reacting with the hydoxyl radical (OH) forming peroxy radicals (\ce{RO2}) in the presence of \ce{O2} \reactionref{r:RH_OH}. 
Further reactions of VOC degradation products produces \ce{RO2} and hydroperoxy radicals (\ce{HO2}). 
Both \ce{RO2} and \ce{HO2} radicals react with NO to produce \ce{NO2} (\reactionref{r:RO2_NO}--\reactionref{r:HO2_NO}) leading directly to \ce{O3} production via \reactionref{r:NO2_hv} and \reactionref{r:O2_O3P}. 
Thus the amount of \ce{O3} produced from the degradation of a VOC is directly related to the number of NO to \ce{NO2} conversions by peroxy and hydroperoxy radicals \citep{Atkinson:2000}.
\begin{reactionlist}
    \reactionitem{\ce{RH + OH + O2}}{\ce{RO2 + H2O + O2}}{new}{r:RH_OH}
    \reactionitem{\ce{RO2 + NO}}{\ce{RO + NO2}}{new}{r:RO2_NO}
    \reactionitem{\ce{HO2 + NO}}{\ce{OH + NO2}}{new}{r:HO2_NO}
\end{reactionlist}

% NOx and VOC sensitive regimes
This chemistry leads to \ce{O3} concentration being a non-linear function of the \ce{NO_x} and VOCs concentrations. 
Three distinct atmospheric regimes with respect to \ce{O3} production can be defined \citep{Jenkin:2000}. 
In the \ce{NO_x}-sensitive regime, VOC concentrations are much higher than those of \ce{NO_x} and \ce{O3} production is dependant on \ce{NO_x} concentrations. 
On the other hand, when \ce{NO_x} concentration is much higher than that of VOCs, it is the VOC concentration that determines the amount of \ce{O3} produced. 
This is the VOC-sensitive regime. 
Finally, in the \ce{NO_x}-VOC-sensitive regime there is maximal \ce{O3} production which is controlled by both VOC and \ce{NO_x} concentrations.

These different atmospheric regimes give rise to distinct radical removal mechanisms \citep{Kleinman:1991}. 
In the \ce{NO_x}-sensitive regime, the concentration of radicals is very high relative to \ce{NO_x} causing radical removal by either radical combination reactions \reactionref{r:RO2_HO2} or bimolecular destruction reactions \reactionref{r:HO2_OH} \citep{Kleinman:1994}.
\begin{reactionlist}
    \reactionitem{\ce{RO2 + HO2}}{\ce{ROOH + O2}}{new}{r:RO2_HO2}
    \reactionitem{\ce{HO2 + OH}}{\ce{H2O + O2}}{new}{r:HO2_OH}
\end{reactionlist}
Whereas in the VOC-sensitive regime, high \ce{NO_x} concentrations lead to radical removal by reaction with \ce{NO2}. 
These reactions lead to nitric acid \reactionref{r:NO2_OH} and peroxy acetyl nitrate (PAN) species production \reactionref{r:RC(O)O2_NO2}.
\begin{reactionlist}
    \reactionitem{\ce{NO2 + OH}}{\ce{HNO3}}{new}{r:NO2_OH}
    \reactionitem{\ce{RC(O)O2 + NO2}}{\ce{RC(O)O2NO2}}{new}{r:RC(O)O2_NO2}
\end{reactionlist}
In the \ce{NO_x}-VOC-sensitive regime, the amount of radicals is comparable to the amount of \ce{NO_x} leading to no dominant radical removal mechanism. 
Hence, radical and PAN chemistry play an important role in \ce{O3} production as radical production helps fuel \ce{O3} production and PAN acts as a reservoir species for both radicals and \ce{NO2}. 

%OPP 
VOCs impact \ce{O3} production in different ways due to their diverse reaction rates and degradation pathways. 
Ozone Production Potentials (OPP) quantify the effect of individual VOCs on \ce{O3} production. 
OPPs are typically been calculated through incremental reactivity (IR) studies. 
This involves changing the concentration of a particular VOC by a known increment and calculating the resulting change in \ce{O3}. 

Examples of IR scales are the Maximal Incremental Reactivity (MIR) and Maximum Ozone Incremental Reactivity (MOIR) scales in \citet{Carter:1994}, as well as the Photochemical Ozone Creation Potential (POCP) scale of \citet{Derwent:1996} and \citet{Derwent:1998}. 
Different \ce{NO_x} conditions were used when calculating these IR scales thus calculating the OPPs of VOCs in different atmospheric regimes.

%The MIR scale was derived using \ce{NO_x} conditions that gave the highest incremental reactivity for the VOC mix and corresponds to the 
%VOC-sensitive regime. Whilst the MOIR scale was calculated with \ce{NO_x} conditions that gave peak \ce{O3} production for the VOC mix --
%the VOC-\ce{NO_x}-sensitive regime. 

%POCPs were calculated using \ce{NO_x} conditions representing polluted conditions in northwest Europe. The sensitivity of POCPs to \ce{NO_x} 
%was also investigated in \cite{Derwent:1998}. POCPs were determined using half and twice the base case \ce{NO_x} emissions, thus representing 
%\ce{NO_x}-sensitive and VOC-sensitive regimes respectively. This study is concerned with comparing maximal \ce{O3} production from and so the 
%\ce{NO_x}-conditions used induce the VOC-\ce{NO_x}-sensitive regime.

IR calculations lack detailed mechanistic information about the processes affecting \ce{O3} formation and combine both the direct and indirect effects of VOC degradation on \ce{O3} production. 
The direct effect is the impact of the increased VOC concentration on \ce{O3} production. 
This is related to the number of NO to \ce{NO2} conversions from \ce{RO2} produced during VOC degradation. 
Whereas the indirect effect is the influence of the increased radical species availability between the base VOC-mix and the increased VOC-mix on \ce{O3} production.

OPPs have been calculated using a tagging approach in \citet{Butler:2011}. 
All organic degradation products of a VOC are labelled with the name of the parent VOC enabling attribution of \ce{O_x} production back to the emitted VOC. 
The Tagged Ozone Production Potential (TOPP) of a VOC is obtained by normalising the total daily \ce{O_x} production by the total emissions of the VOC. 
The tagging approach follows each degradation pathway of each VOC making this approach ideal to compare the chemistry of different chemical mechanisms. 
Full details of the TOPP calculation is found in Section \ref{s:methodology}.

%chemical mechanisms are how they have to represent O3 chemistry
Chemical mechanisms represent atmospheric chemistry in chemical transport models. 
All mechanisms have some level of simplification due to the computational resourses needed for gas-phase chemistry calculations. 
These levels of simplification arise from the different scopes of modelling studies, for example a global CTM requires a more computational efficient mechanism than a boxmodel. 

A near-explicit mechanism such as the Master Chemical Mechanism (MCM) \citep{Jenkin:2003, Saunders:2003, Bloss:2005} includes a large amount of mechanistic details ($\sim$ $12,000$ reactions). 
This makes the MCM ideal as a reference mechanism when comparing chemical mechanisms. 
The latest version, MCM v3.2, \citep{MCM_Site} is the reference mechanism in this study.

Further simplification is achieved by using either a lumped structure or a lumped molecule approach \citep{Dodge:2000}. 
Lumped structure mechanisms speciate VOCs by the carbon bonds of the emitted VOC. 
Whilst in lumped molecule mechanisms, VOCs are represented by a specific organic or mechanism species representing a number of VOCs. 
Different lumped molecule mechanisms use diverse approaches to creating mechanism species representing emitted VOCs.  

%other chemical mechanism comparison studies
Other chemical mechanism comparison studies consider \ce{O3} concentration time series over varying VOC and \ce{NO_x} concentrations. 
Examples of such studies are \citet{Dunker:1984, Kuhn:1998, Gross:2003} and \citet{Emmerson:2009}. 
A common outcome is that the largest discrepancies in \ce{O3} concentration times series arise when modelling urban rather than rural conditions. 
Other atmospheric species, such as PAN or \ce{H2O2}, also show large concentration time series deviations between mechanisms.

IR scales have also been used as a tool for mechanism comparison. 
In \citet{Derwent:2010}, the near-explicit mechanisms MCM v3.1 and SAPRC-07 were compared using first-day POCP values calculated under VOC-sensitive conditions. 
In general, the POCP values were comparable between the mechanisms. 

OPPs are a useful mechanism comparison tool as they relate \ce{O3} production to a single number. 
In this study, TOPPs calculated for a number of VOCs using different mechanisms are compared to those calculated with the MCM v3.2. 
TOPPs are ideal for such a mechanism comparison study as they address multi-day scenarios and the tagging approach allows a detailed comparison of the VOC degradation chemistry between the mechanisms.

%paper outline
The chemical mechanisms compared and the methodology are described in Section \ref{s:methodology}. 
Section \ref{s:overall_results} presents the comparison of the first-day TOPPs and their time series. 
Detailed analysis of \ce{O_x} production chemistry as well as the treatment of radical and PAN chemistry is compared in Section \ref{s:detailed_results}. 
Finally the conclusions and further work are in outlined \mbox{Section \ref{s:conclusions}}.  
