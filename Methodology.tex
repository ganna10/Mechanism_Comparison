
\subsection{Chemical Mechanisms} \label{ss:mechanisms}

{
    \renewcommand{\arraystretch}{1.1}
    \begin{table}
        \centering
        \begin{tabular}{M{2.5cm}M{2.0cm}M{2.0cm}M{2.5cm}M{3.9cm}}
            \hline \hline
            \textbf{Chemical Mechanism} & \textbf{Number of Organic Species} & \textbf{Number of Organic Reactions} & \textbf{Type of Lumping} & \textbf{Reference} \\ \hline
            MCM v3.2 & \num[group-separator={,}]{5708} & \num[group-separator={,}]{16349} & No lumping & \citet{MCM_Site} \\ \hline
            MCM v3.1 & \num[group-separator={,}]{4351} & \num[group-separator={,}]{12691} & No lumping & \citet{Jenkin:1997, Saunders:2003, Jenkin:2003, Bloss:2005} \\ \hline
            CRI v2 & $411$ & \num[group-separator={,}]{1145} & Lumped intermediates & \citet{Jenkin:2008} \\ \hline
            MOZART-4 & $69$ & $145$ & Lumped molecule & \citet{Emmons:2010} \\ \hline
            RADM2 & $44$ & $103$ & Lumped molecule & \citet{Stockwell:1990} \\ \hline
            RACM & $58$ & $193$ & Lumped molecule & \citet{Stockwell:1997} \\ \hline
            RACM2 & $99$ & $315$ & Lumped molecule & \citet{Goliff:2013} \\ \hline
            CBM-IV & $20$ & $45$ & Lumped structure & \citet{Gery:1989} \\ \hline
            CB05 & $37$ & $99$ & Lumped structure & \citet{Yarwood:2005} \\ 
            \hline \hline
        \end{tabular}
        \vspace{0mm}
        \caption{The chemical mechanisms used in the study, MCM v3.2 is the reference mechanism.}
        \vspace{-4mm}
        \label{t:mechanisms}
    \end{table}
}

The chemical mechanisms compared in this study are outlined in Table \ref{t:mechanisms}; each mechanism is summarised below.

%MCM
The Master Chemical Mechanism (MCM) \citep{Jenkin:1997, Jenkin:2003, Saunders:2003, Bloss:2005, MCM_Site} is a near-explicit mechanism describing the degradation of $125$ primary VOC. 
The latest version (\mbox{MCM v3.2}) is the reference mechanism in this study.

%CRI
The Common Representative Intermediates (CRI v2) \citep{Jenkin:2008} is a reduced chemical mechanism describing the oxidation of the same primary VOC as the previous MCM version (MCM v3.1). 
VOC degradation is simplified by lumping the degradation products of many VOC into species whose \ce{O3} production reflects that of the \mbox{MCM v3.1}. 
We use the full version of the \mbox{CRI v2} (\url{http://mcm.leeds.ac.uk/CRI}).  
Differences in \ce{O3} production between the CRI v2 and MCM v3.2 may be due to changes in the MCM versions rather than the CRI reduction techniques, hence we include the MCM v3.1 in the study.

%MOZART
The Model for OZone and Related chemical Tracers version 4 (MOZART-4) represents global tropospheric and stratospheric chemistry \citep{Emmons:2010}. 
Explicit species exist for methane, ethane, propane, ethene, propene, isoprene and $\alpha$-pinene.
All other VOC are represented by lumped species which is determined by the functionality of the VOC.

%RADM2
The second generation Regional Acid Deposition Model (RADM2) describes regional scale atmospheric chemistry \citep{Stockwell:1990}. 
Explicit species represent methane, ethane, ethene and isoprene, while all other VOC are assigned to lumped species based on OH-reactivity and molecular weight.

%RACM
RADM2 was updated to the Regional Atmospheric Chemistry Mechanism (RACM) \citep{Stockwell:1997}. 
Updates include more explicit and lumped species representing VOC as well as revised chemistry.

%RACM2
RACM2 is the latest RACM version \citep{Goliff:2013} with substantial updates to the chemistry. 
More lumped and explicit species representing emitted VOC are also included.

%CBM4
The Carbon Bond Mechanism version four (CBM-IV) simulates polluted urban conditions \citep{Gery:1989}. 
CBM-IV represents ethene, formaldehyde and isoprene explicitly while all other emitted VOC are lumped by their carbon bond types. 

All primary VOC were assigned to lumped species in CBM-IV as described in \citet{Hogo:1989}. 
For example, pentane is represented as $5$ PAR as it has five single carbon bonds (PAR represents the C--C bond).
An initial mixing ratio of $1200$ pptv was assigned to \mbox{$6000$ ($= 1200 \times 5$) pptv} in CBM-IV.

%CB05
The fifth version of the Carbon Bond Mechanism (CB05) \citep{Yarwood:2005} includes explicit species representing methane, ethane and acetaldehyde. 
Other updates include revised allocation of primary VOC and updated reaction rate constants.

\subsection{Model Setup} \label{ss:model_setup}

\begin{sidewaystable}
    \centering
    \begin{tabular}{lllllllll}
        \hline \hline
        \multirow{2}{*}{\textbf{NMVOC}} & \textbf{Mixing} & \textbf{MCM v3.1, v3.2,} & \multirow{2}{*}{\textbf{MOZART-4}} & \multirow{2}{*}{\textbf{RADM2}} & \multirow{2}{*}{\textbf{RACM}} & \multirow{2}{*}{\textbf{RACM2}} & \multirow{2}{*}{\textbf{CBM-IV}} & \multirow{2}{*}{\textbf{CB05}}\\ & \textbf{Ratio (pptv)} & \textbf{CRI v2} & & & & & & \\ 
        \hline \hline \multicolumn{9}{c}{\textbf{Alkanes}}  \\ \hline
        Ethane & $6610$ & C2H6 & C2H6 & ETH & ETH & ETH & $0.4$ PAR & ETHA \\
        Propane  & $6050$ & C3H8 & C3H8 & HC3 & HC3 & HC3 & $1.5$ PAR & $1.5$ PAR \\
        Butane & $2340$ & NC4H10 & BIGALK & HC3 & HC3 & HC3 & $4$ PAR & $4$ PAR \\
        $2$-Methylpropane & $1240$ & IC4H10 & BIGALK & HC3 & HC3 & HC3 & $4$ PAR & $4$ PAR \\
        Pentane & $1200$ & NC5H12 & BIGALK & HC5 & HC5 & HC5 & $5$ PAR & $5$ PAR \\
        $2$-Methylbutane & $2790$ & IC5H12 & BIGALK & HC5 & HC5 & HC5 & $5$ PAR & $5$ PAR \\
        Hexane & $390$ & NC6H14 & BIGALK & HC5 & HC5 & HC5 & $6$ PAR & $6$ PAR \\
        Heptane & $160$ & NC7H16 &  BIGALK & HC5 & HC5 & HC5 & $7$ PAR & $7$ PAR \\
        Octane & $80$ & NC8H18 & BIGALK & HC8 & HC8 & HC8 & $8$ PAR & $8$ PAR \\ \hline 
        \multicolumn{9}{c}{\textbf{Alkenes}} \\ \hline
        Ethene & $2430$ & C2H4 & C2H4 & OL2 & ETE & ETE & ETH & ETH \\
        Propene & $490$ & C3H6 & C3H6 & OLT & OLT & OLT & OLE + PAR & OLE + PAR \\ 
        Butene & $65$ & BUT1ENE & BIGENE & OLT & OLT & OLT & OLE + $2$ PAR & OLE + $2$ PAR \\ 
        \multirow{2}{*}{$2$-Methylpropene} & \multirow{2}{*}{$130$} & \multirow{2}{*}{MEPROPENE} & \multirow{2}{*}{BIGENE} & \multirow{2}{*}{OLI} & \multirow{2}{*}{OLI} & \multirow{2}{*}{OLI} & PAR + FORM & FORM + \\ & & & & & & & \hspace{3mm}+ ALD2 & \hspace{3mm}$3$ PAR \\
        Isoprene & $270$ & C5H8 & ISOP & ISO & ISO & ISO & ISOP & ISOP \\ \hline
        \multicolumn{9}{c}{\textbf{Aromatics}} \\ \hline 
        Benzene & $480$ & BENZENE & TOLUENE & TOL & TOL & BEN & PAR & PAR \\
        Toluene & $1380$ & TOLUENE & TOLUENE & TOL & TOL & TOL & TOL & TOL \\
        m-Xylene & $410$ & MXYL & TOLUENE & XYL & XYL & XYM & XYL & XYL \\
        p-Xylene & $210$ & PXYL & TOLUENE & XYL & XYL & XYP & XYL & XYL \\
        o-Xylene & $200$ & OXYL & TOLUENE & XYL & XYL & XYO & XYL & XYL \\
        Ethylbenzene & $210$ & EBENZ & TOLUENE & TOL & TOL & TOL & TOL + PAR & TOL + PAR \\ \hline \hline
    \end{tabular}
    \vspace{1mm}
    \caption{NMVOC present in Los Angeles, mixing ratios taken from \citet{Baker:2008} and their representation in each chemical mechanism. The representation of the NMVOC in each mechanism is based upon the recommendations of the literature for each mechanism.}
    \vspace{-4mm}
    \label{t:initial_conditions}
\end{sidewaystable}

%general info
The modelling approach and set-up follows the original TOPP study in \citet{Butler:2011}.
A summary is provided here, further details are found in the supplement and \citet{Butler:2011}. 
Maximum \ce{O3} production is achieved by balancing the chemical source of radicals and \ce{NO_x} by emitting the appropriate amount of NO at each time step.
This balance was determined separately for each mechanism.

%initial conditions
The initial NMVOC are typical of Los Angeles and their initial mixing ratios are taken from \citet{Baker:2008}. 
Following \citet{Butler:2011}, the emissions required to keep the initial mixing ratios constant until noon of the first day of each VOC were calculated using the \mbox{MCM v3.2.}
These emissions are used for every mechanism, ensuring the amount of each VOC was the same in every model run.
Methane (\ce{CH4}) was fixed at \mbox{$1.8$ ppmv} whilst carbon monoxide (CO) and \ce{O3} were initialised at \mbox{$200$ ppbv} and \mbox{$40$ ppbv} and then allowed to evolve freely.

The NMVOC used in the study, their mixing ratios and representation in the each mechanism are outlined in Table \ref{t:initial_conditions}.
NMVOC are assigned to mechanism species following the recommendations from the literature of each mechanism.
Initial emissions are weighted by the carbon number of the mechanism species ensuring the amount of reactive carbon was constant.

\subsubsection{Mechanism Implementation} %implementation of mechanisms for the runs

The MECCA boxmodel \citep{Sander:2005} is based upon the Kinetic Pre-Processor (KPP) \citep{Damian:2002} and used as described in \citet{Butler:2011}. 
Hence, all chemical mechanisms were adapted into modularised KPP format.

The MCM v3.2 inorganic gas-phase chemistry was used in each run to remove any differences between treatment of inorganic chemistry in each mechanism.
Thus any differences between the \ce{O3} produced by the mechanisms are due to the treatment of organic degradation chemistry.
Inorganic gas-phase chemistry is relatively well-known and any differences arise from inconsistencies between IUPAC and JPL reaction rate constants \citep{Emmerson:2009}.

Some mechanisms include reactions which are only important in the stratosphere or free troposphere.
For example, PAN photolysis is only important in the free troposphere \citep{Harwood:2003} and has been removed from MOZART-4, RACM2 and CB05 for the purpose of the study. 
These reactions were removed as this study only considers processes occuring below the planetary boundary layer.

The MCM v3.2 approach to photolysis, dry deposition and \ce{RO2}--\ce{RO2} reactions was used for each mechanism, details are found in the supplement.

\subsection{Tagged Ozone Production Potential (TOPP)}
This section summarises the tagging approach described in \citet{Butler:2011} and applied to this study.

\subsubsection[Ox Family and Tagging Approach]{\ce{O_x} Family and Tagging Approach} \label{ss:tagging} %tagging of mechanisms

\ce{O3} production and loss budgets are dominated by rapid photochemical cycles such as \reactionref{r:NO_O3}--\reactionref{r:O2_O3P}.
The effect of rapid production and loss cycles are removed by using chemical families including the rapidly inter-converting species.
We define the \ce{O_x} family to include \ce{O3}, \ce{O(^3P)}, \ce{O(^1D)}, \ce{NO2} and other species that are involved in fast cycling with \ce{NO2}, such as \ce{HO2NO2} and PAN (peroxy acetyl nitrate) species.
The production of \ce{O_x} family is closely tied to that of \ce{O3} and is demonstrate in \citet{Butler:2011}.

Tagging follows the degradation of all emitted VOC through all possible pathways, re-writing every organic degradation product to include a label with the name of the emitted VOC.
Thus, each emitted VOC effectively has its own set of degradation reactions \citep{Butler:2011}.
This procedure is shown schematically for \ce{CH4} in Figure \ref{f:tagging_approach}.

\begin{figure}
    \centering
    \tikzstyle{int}=[minimum size=0.5em, line width=0.5mm]
    \begin{tikzpicture}[>=latex']
        \node [int] (a) [align=right]{\textbf{\large{\ce{CH4}}}};
        
        \node [int] (c) [right = 5.0mm of a, align=left]{\textbf{HCHO}};
        \node [int] (b) [above = 1cm of c.west, anchor=west, align=left]{\textbf{\ce{CH3O2}}};
        \node [int] (d) [below = 1cm of c.west, anchor=west, align=left]{\textbf{CO}};
        
        \draw [decoration={brace, amplitude=1.0em},decorate, ultra thick] (d.south -| d.west) -- (b.north -| b.west);
        \node [int] (f) [right = 1cm of c, align=left] {{\textbf{\small{HCHO\_CH4}}}};
        \node [int] (e) [above = 1cm of f.west, anchor=west, align=left] {{\textbf{\small{CH3O2\_CH4}}}};
        \node [int] (g) [below = 1cm of f.west, anchor=west, align=left] {{\textbf{\small{CO\_CH4}}}};
    \end{tikzpicture} 
    \vspace{1mm}
    \caption{The organic species, \ce{CH3O2}, HCHO and CO, are formed during \ce{CH4} degradation. These species are re-labelled with a CH4 tag, enabling attribution of \ce{O3} production to \ce{CH4}.}
    \vspace{-4mm}
    \label{f:tagging_approach}
\end{figure} 

The tags of an emitted VOC allocates the impact of the VOC degradation on \ce{O_x} production, this is shown in Figure \ref{f:Ox_budget} (a).
Without tagging, \ce{O_x} production can be attributed to the individual reactions (Figure \ref{f:Ox_budget} (b)) but this lacks the information about the VOC source of the organic reactants.

\begin{figure}
    \centering
    \includegraphics[width=0.8\textwidth]{img/MCMv3_2_tagged_non_tagged_Ox_budget}
    \vspace{1mm}
    \caption{Day-time \ce{O_x} production budgets using the MCM v3.2, \mbox{(a) with} tagging, where \ce{O_x} production is attributed to the emitted VOC and (b) without tagging.}
    \vspace{-4mm}
    \label{f:Ox_budget}
\end{figure} 

\subsubsection{Tagged Ozone Production Potential (TOPP) Definition} %final definition of TOPP

Allocating \ce{O_x} production to emitted VOC using the tagging approach is the basis for calculating the TOPP of an emitted VOC.
The TOPP for a given VOC is the number of \ce{O_x} molecules produced per emitted VOC molecule.  
After determing the total daily contribution to \ce{O_x} production from an emitted VOC, this is normalised by the total emissions of the VOC on the first day of the model run.
