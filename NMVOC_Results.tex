
\subsection[Carbon Number of Ox Producing Degradation Products]{Carbon Number of \ce{O_x} Producing Degradation Products} \label{ss:c_number} %comparison by carbon number breakdown

\begin{figure}
    \centering
    \includegraphics[width=\textwidth]{img/carbon_percent_total_Ox_production}
    \vspace{0mm}
    \caption{Pentane and toluene day-time \ce{O_x} production percentage contributions by degradation product carbon number.}
    \vspace{-4mm}
    \label{f:percent_carbon}
\end{figure}

\begin{figure}
    \centering
    \includegraphics[width=\textwidth]{img/TOL_HO2x_intermediates}
    \vspace{0mm}
    \caption{Day-time \ce{HO2_x} production budgets from toluene degradation.}
    \vspace{-4mm}
    \label{f:toluene_HO2x}
\end{figure} 

The different NMVOC \ce{O_x} production time profiles in Figure \ref{f:TOPP_dailies} result from the varying rates NMVOCs break up into smaller fragments \citep{Butler:2011}.
Day-time \ce{O_x} production percentage contributions are allocated to degradation product carbon numbers during pentane and toluene degradation in Figure \ref{f:percent_carbon}.
\ce{HO2} and operator species, such as XO2, contributions are allocated to their organic reactant source carbon number.
Mechanism species with fractional carbon numbers are included as the nearest integral carbon number. 

Figure \ref{f:percent_carbon} indicates that degradation products having the same carbon number as the emitted NMVOC have more influence on \ce{O_x} production in near-explicit than less-explicit mechanisms.
MOZART-4 toluene degradation is an exception.

Rapid reactive carbon loss during toluene degradation is responsible for the sharp decrease in \ce{O_x} production over the MOZART-4 model run.
This is shown in \mbox{Figure \ref{f:net_carbon_loss}} and further explained in Section \ref{ss:carbon_loss}.
Despite C7 degradation products having larger relative contribution to \ce{O_x} production in MOZART-4 than \mbox{MCM v3.2}, the absolute \ce{O_x} production (Figure \ref{f:TOPP_dailies}) is lower in MOZART-4.

C1 and C2 degradation products are responsible for \ce{O_x} production from alkane degradation in CBM-IV and CB05.
Alkane degradation is depicted by PAR, whose degradation consists of either C1 or C2 mechanism species.
PAR represents singly bonded carbons and alkane emissions are scaled by carbon number to account for increased \ce{O_x} production with alkane carbon number.

CRI v2 and RACM2 toluene degradation differs from other mechanisms as maximum \ce{O_x} production is reached on the second day. 
Higher second day \ce{HO2} production in CRI v2 and RACM2 is responsible. 
Figure \ref{f:toluene_HO2x} illustrates MCM v3.2, CRI v2 and RACM2 day-time \ce{HO2_x} (= \ce{HO2 + HO2NO2}) production budgets allocated to the responsible reactions. 
\ce{HO2_x} production influences \ce{O_x} production through \ce{HO2} converting NO to \ce{NO2} via \reactionref{r:HO2_NO}.

\mbox{Figure \ref{f:toluene_HO2x}} shows more \ce{HO2_x} production from the CARB3 and OH reaction in \mbox{CRI v2} than its corresponding MCM v3.2 reaction (\mbox{GLYOX + OH}).  
Glyoxal is represented as CARB3 in CRI v2 and GLYOX in MCM v3.2, but there are differences in the chemistry of these species.
In CRI v2, CARB3 is only produced from aromatic degradation whilst GLYOX is also a non-aromatic VOC degradation product in MCM v3.2. 

CRI v2 glyoxal degradation is through OH-oxidation and photolysis, extra degradation options are available in MCM v3.2. 
Moreover, the OH-oxidation rate constant in CRI v2 is $\sim$ $15$\% faster than in MCM v3.2. 

Glyoxal has three photolysis pathways in MCM v3.2 and only one in \mbox{CRI v2}, outlined in Table \ref{t:glyoxal}. 
The additional photolysis MCM v3.2 pathways are non-\ce{HO2_x} producing leading to lower \ce{HO2_x} production.  
The combination of the higher rate constant for the glyoxal--OH reaction and additional \ce{HO2_x} production during photolysis are responsible for the higher \ce{HO2_x} production in CRI v2. 
{
    \renewcommand{\arraystretch}{1.3}
    \begin{table}
        \centering
        \small
        \begin{tabular}{lP{6.8cm}P{3.0cm}}
            \hline \hline
            \textbf{Mechanism} & \textbf{Photolysis Pathway} & \textbf{Rate Parameter} \\ \hline \hline
            \multirow{3}{*}{MCM v3.2} & GLYOX + hv = CO + CO + H2 & J$_{31}$ \\
            & GLYOX + hv = HCHO + CO & J$_{32}$ \\
            & GLYOX + hv = CO + CO + HO2 + HO2 & J$_{33}$ \\ \hline
            CRI v2 & CARB3 + hv = CO + CO + HO2 + HO2 & J$_{33}$ \\ \hline \hline
        \end{tabular}
        \vspace{1mm}
        \caption{Glyoxal photolysis in MCM v3.2 and CRI v2 with specified rate parameters.}
        \vspace{-4mm}
        \label{t:glyoxal}
    \end{table}
}

RACM2 \ce{HO2_x} production during toluene degradation includes pathways unique to RACM2.
Initial OH-oxidation produces the mechanism species TR2 that degrades immediately producing the epoxy species EPX and \ce{HO2}.
Epoxy production in \mbox{MCM v3.2} is a minor toluene oxidation branch.
The EPX ozonolysis rate constant is $100$ times faster than the analogous \mbox{MCM v3.2} species (TLEPOXMUC).
Furthermore, EPX ozonolysis produces $1.5$ \ce{HO2}, thus the reacted \ce{O3} is regenerated with extra \ce{HO2} causing the RACM2 toluene degradation second day \ce{O_x} production maximum.

\subsection[Reactive Carbon Loss during Ox Production]{Reactive Carbon Loss during \ce{O_x} Production} \label{ss:carbon_loss}

\begin{figure}
    \centering
    \includegraphics[width=\textwidth]{img/net_reactive_carbon_loss}
    \vspace{0mm}
    \caption{Daily net carbon loss rate during \ce{O_x} production from pentane and toluene degradation.}
    \vspace{-4mm}
    \label{f:net_carbon_loss}
\end{figure}

\begin{figure}
    \centering
    \includegraphics[width=\textwidth]{img/CH3CO3_budget_comparison}
    \vspace{0mm}
    \caption{Day-time \ce{CH3CO3} production and loss budgets.}
    \vspace{-4mm}
    \label{f:CH3CO3_budget}
\end{figure}

Day-time net carbon loss rates during \ce{O_x} production from pentane and toluene degradation are plotted in Figure \ref{f:net_carbon_loss}.
The net carbon yield of each \ce{O_x} producing reaction was calculated and multiplied by the reaction rate.
The supplement contains plots showing the reactions responsible for carbon loss in pentane and toluene degradation.

RACM2 pentane degradation has the largest carbon loss rate and a different time series to previous versions (RADM2 and RACM).  
The \mbox{NO + HC5P} (pentyl peroxy radical) reaction is largely responsible for carbon loss.
The reaction products and their yields were updated from RACM to RACM2 due to increased oxygenated product speciation in RACM2.
For example, ALD describes acetaldehyde and higher aldehyde chemistry in RACM whilst in RACM2 acetaldehyde is represented explicitly (ACD) and ALD represents C3 and higher aldehydes \citep{Goliff:2013}.
These RACM2 updates result in more carbon loss through \mbox{HC5P + NO} reaction.

Another contributing factor to increased carbon loss in RACM2 is the updated \mbox{NO + HC3P} (propyl peroxy radical) reaction.
In RACM, this reaction gave net carbon gain whereas it contributes to net carbon loss in RACM2.

MOZART-4 pentane degradation has high carbon loss via the \ce{CH3CO3 + NO} reaction.
This reaction leads to carbon loss in every mechanism however it has the most influence in MOZART-4. 
Figure \ref{f:CH3CO3_budget} shows extra \ce{CH3CO3} production through the \mbox{MEKO2 + NO} reaction.
This reaction contributes twice to \ce{CH3CO3} production -- as a primary product and OH-oxidation of \ce{CH3CHO}.

CBM-IV and CB05 show little carbon loss throughout pentane degradation.
Since pentane degradation is described by C2 and C1 products there is little carbon to be lost.

MOZART-4 and RADM2 have the highest loss carbon rates during toluene degradation.
The high net carbon loss from BIGALD (unsaturated dicarbonyls) photolysis coupled with its large yields during MOZART-4 toluene degradation reactions leads to rapid carbon loss.
BIGALD is produced from other reactions that result in carbon loss: \mbox{NO + TOLO2} and \mbox{NO2 + XOH}, where TOLO2 is the peroxy radical from \mbox{toluene + OH} reaction and XOH represents \ce{C7H10O6} \citep{Emmons:2010}.

RADM2 carbon loss during toluene degradation is mainly from the \mbox{NO + TOLP} reaction, TOLP is the peroxy radical formed from toluene OH-oxidation.
RACM and RACM2 aromatic degradation has been heavily updated from RADM2 and this reaction removed altogether \citep{Stockwell:1997, Goliff:2013}.

CRI v2 has faster carbon loss from toluene degradation than the MCM v3.2.
Both branches of the \mbox{NO + RA16O2} reaction, RA16O2 is the peroxy radical formed from toluene OH-oxidation, are responsible for the carbon loss in CRI v2.

\subsection[Degradation Fragment Ox Production Efficiency]{Degradation Fragment \ce{O_x} Production Efficiency} \label{ss:OxPE}

\begin{figure}
    \centering
    \includegraphics[width=\textwidth]{img/OxPEs_by_C_number}
    \vspace{0mm}
    \caption{Degradation product \ce{O_x} production efficiency during pentane and toluene degradation.}
    \vspace{-4mm}
    \label{f:OxPE}
\end{figure}

Reduced mechanisms have similar \ce{O_x} production to more-explicit mechanisms despite quicker reactive carbon loss.
Moreover, C1 and C2 degradation products have higher influence on \ce{O_x} production in less-explicit mechanisms.
\ce{O_x} production efficiencies (OxPE) for the degradation fragment sizes were calculated by dividing total \ce{O_x} production from the degradation fragment size by total \ce{O_x} consumption.
This is illustrated in Figure \ref{f:OxPE} for pentane and toluene degradation.

All less-explicit mechanisms have higher C1 and C2 OxPEs than the MCMs during pentane and toluene degradation.
CBM-IV and CB05 have low OxPEs during toluene degradation mirroring the low \ce{O_x} production noted in Section \ref{sss:aromatic}.
Reduced mechanisms simulate similar \ce{O_x} production to more-explicit mechanisms by being more efficient at producing \ce{O_x} particularly from smaller sized fragments.

The \ce{CH3O2 + NO} reaction is the main C1 \ce{O_x} production pathway in reduced mechanisms with a larger contribution to the MCMs.
Reduced mechanisms have higher aldehyde production starting a degradation chain resulting in higher \ce{CH3O2} amounts.
For example, acetaldehyde (\ce{CH3CHO}) OH-oxidation produces \ce{CH3CO3} whose reaction with NO produces \ce{O_x} as well as \ce{CH3O2}.
The supplement shows total aldehyde production budgets from pentane and toluene degradation in each mechanism.  

Pentane degradation has larger total aldehyde production in all reduced mechanisms.
Whilst during toluene degradation, total aldehyde production is more variable between reduced mechanisms.
The overall total aldehyde production budgets mirrors the OxPEs of Figure \ref{f:OxPE}.

Increased aldehyde amounts in reduced mechanisms result in increased OxPEs but also increased reactive carbon loss.
Section \ref{ss:carbon_loss} noted that the \ce{CH3CO3 + NO} reaction is a major source of reactive carbon loss in all mechanisms. 
Thus increased \ce{CH3CO3} levels produced from higher aldehyde production in less-explicit mechanisms leads to higher reactive carbon loss rates.
