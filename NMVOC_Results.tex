\todo{merge this whole section as subsection of Results section}
This section details \ce{O_x} production in the chemical mechanisms of this study by comparing the treatment of the degradation products.
\ce{O_x} production during pentane and toluene degradation is compared between mechanisms by analysing the distribution of \ce{O_x} production with the carbon number of the degradation products and the loss rates of reactive carbon.

Some mechanism species in RADM2, RACM and RACM2 have fractional carbon numbers \citep{Stockwell:1990, Stockwell:1997, Goliff:2013}.
\ce{O_x} production such mechanism species is assigned as \ce{O_x} production of the nearest integral carbon number.

Many reduced mechanisms use operator species as a proxy for peroxy radicals during VOC degradation enabling more \ce{O_x} production without including additional species in the mechanism.
\ce{O_x} production from operator species was assigned as \ce{O_x} production from the organic degradation product producing the operator species.
This approach was also used to assign the \ce{O_x} production via \reactionref{r:HO2_NO} from \ce{HO2}.

\subsection[Carbon Number of Ox Producing Degradation Products]{Carbon Number of \ce{O_x} Producing Degradation Products} \label{ss:c_number} %comparison by carbon number breakdown

\begin{figure}
    \centering
    \includegraphics[width=1.10\textwidth]{img/Ox_production_by_C_number}
    \vspace{0mm}
    \caption{Day-time \ce{O_x} production during pentane and toluene degradation is attributed to the number of carbons of the degradation products for each mechanism.}
    \vspace{-4mm}
    \label{f:carbon}
\end{figure}

The time dependent \ce{O_x} production of different VOC in Figure \ref{f:TOPP_dailies} results from the varying rates at which VOC break up into smaller fragments \citep{Butler:2011}.
The day-time \ce{O_x} production during pentane and toluene is distributed by the carbon numbers of the \ce{O_x} producing degradation products in Figure \ref{f:carbon} to compare the break down between mechanisms.

During pentane degradation, more \ce{O_x} is produced from degradation products having the same carbon number as pentane throughout the model run in explicit mechanisms.
Reduced mechanisms produce similar amounts of \ce{O_x} on the first day by increasing the \ce{O_x} production from smaller degradation products, indicating that pentane is broken into smaller fragments quicker in reduced mechanisms.  

The first day break down of pentane affects the \ce{O_x} production of subsequent days by limiting the availability of higher degradation products to produce further \ce{O_x}.
Thus \ce{O_x} production on subsequent days cannot reach the same levels as in explicit mechanisms, this is analysed for all alkanes in Section \ref{sss:alkanes}.

The \ce{O_x} produced during toluene degradation has a high spread between all mechanisms (Figure \ref{f:TOPP_dailies}).
Distributing this \ce{O_x} production to the carbon numbers of toluene degradation products in Figure \ref{f:carbon} shows a variability in \ce{O_x} production from the different sizes of degradation products between mechanisms.

All reduced mechanisms omit \ce{O_x} production from at least one degradation fragment size that produces \ce{O_x} in the MCM v3.2 indicating that reactive carbon during toluene degradation is lost quicker than more explicit mechanisms.
For example, toluene degradation does not produce \ce{O_x} from degradation products with six carbons in RACM2 unlike the MCM v3.2.

%The \ce{O_x} production during pentane and toluene degradation distributed by the sizes of the degradation products indicates that VOC are broken down quicker in reduced mechanisms.
%
%If reduced mechanisms lose reactive carbon at a quicker rate than the MCM v3.2 this would mean that the reduced mechanisms could not produce the same amount of \ce{O_x} as the MCM as the degradation chains that produce \ce{O_x} are terminated quicker.
%The net loss of reactive carbon during pentane and toluene degradation is analysed in more detail in Section \ref{ss:carbon_loss}.

\subsubsection[Ox Production during Alkane Degradation]{\ce{O_x} Production during Alkane Degradation} \label{sss:alkanes}

\begin{figure}
    \centering
    \includegraphics[width=\textwidth]{img/Alkanes_vs_C}
    \vspace{0mm}
    \caption{First day TOPP value of each alkane versus its carbon number in each mechanism.}
    \vspace{-4mm}
    \label{f:alkanes_C}
\end{figure}

Pentane is represented in CBM-IV and CB05 (lumped structure mechanisms) as $5$ PAR -- the paraffin (C--C) bond -- whose degradation is described using mechanism species having either one or two carbons \citep{Gery:1989, Yarwood:2005}. 
\ce{O_x} production is sustained throughout PAR degradation by recycling of \ce{O_x} producing mechanism species.
An example is the oxy organic radical ROR that is a primary degradation product of PAR and further degradation of ROR also produces ROR with a product yield of $0.02$ \citep{Gery:1989}.
This approach of using representing alkanes as a multiple of the base C--C bonds limits the amount of \ce{O_x} that can be produced during pentane degradation as this reduces the number of available pathways for \ce{O_x} production.

\subsection[Reactive Carbon Loss during Ox Production]{Reactive Carbon Loss during \ce{O_x} Production} \label{ss:carbon_loss}

\begin{figure}
    \centering
    \includegraphics[width=0.8\textwidth]{img/net_reactive_carbon_loss_pentane_toluene}
    \vspace{0mm}
    \caption{Daily net carbon loss rate during \ce{O_x} production from pentane and toluene degradation.}
    \vspace{-4mm}
    \label{f:net_carbon_loss}
\end{figure}

The analysis of the \ce{O_x} production allocated to the different sizes of the pentane and degradation products in Section \ref{ss:c_number} indicates that reduced mechanisms may lose reactive carbon quicker than more explicit mechanisms.
This section compares the reactive carbon loss rates during pentane and toluene degradation in Figure \ref{f:net_carbon_loss} by calculating the net carbon yield from each \ce{O_x} producing reaction and multiplying this yield by the reaction rate.

Pentane degradation shows comparable loss rates of reactive carbon between the MCM v3.2 and the other mechanisms, besides CBM-IV and CB05.
The loss rates of reative carbon during pentane degradation in CBM-IV and CB05 is represented by C1 and C2 species this leads to little carbon loss during pentane degradation in CBM-IV and CB05, hence the lumped species reduced mechanisms have different loss of carbon profiles to other types of mechanisms.

RADM2, RACM and RACM2 have the highest loss rates of reactive carbon on the first day (Figure \ref{f:net_carbon_loss}), this leads to these mechanisms being unable to reach the same \ce{O_x} production as other mechanisms on subsequent days (Figure \ref{f:carbon}).
CRI v2 and MOZART-4 have slower loss rates of reactive carbon than the MCM v3.2 and this results in higher \ce{O_x} production than RADM2, RACM and RACM2 (Figure \ref{f:carbon}).

All reduced mechanisms have quicker loss rate of reactive carbon during toluene degradation (Figure \ref{f:net_carbon_loss}).
In particular, MOZART-4 loses reactive carbon at least twice as fast as other mechanisms and this explains why \ce{O_x} production during toluene degradation in MOZART-4 sharply decreases after the first day (Figure \ref{f:carbon}).
This same trend is also seen in the \ce{O_x} production during toluene degradation in CBM-IV and CB05 (Figure \ref{f:carbon}).

CRI v2 and RACM2 both have a higher loss rate of reactive carbon during toluene degradation on the second day compared to the MCM v3.2 but the both have in increase in \ce{O_x} production on the second day (Figure \ref{f:carbon}) and this is due\ldots

The \ce{CH3CO3 + NO} reaction is the main source of reactive carbon loss during pentane degradation in all mechanisms.
Reactive carbon loss during toluene degradation also has significant loss of carbon from the \ce{CH3CO3 + NO} reaction but the degradation of the toluene peroxy radical formed after initial OH-oxidation also leads to significant reactive carbon loss.
The supplement to this paper includes analysis of which reactions are the main sources of reactive carbon loss during pentane and toluene degradation.

Mechanisms which have high reactive carbon loss rates during the degradation of a VOC limit the amount of \ce{O_x} that can be produced.
Loss of reactive carbon is the constraining factor during toluene degradation in all reduced mechanisms and is a limiting factor during pentane degradation in RADM2, RACM and RACM2.
