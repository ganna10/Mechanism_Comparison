
\subsection[Carbon Number of Ox Producing Degradation Products]{Carbon Number of \ce{O_x} Producing Degradation Products} \label{ss:c_number} %comparison by carbon number breakdown

\begin{figure}
    \centering
    \includegraphics[width=\textwidth]{img/carbon_percent_total_Ox_production}
    \vspace{0mm}
    \caption{Pentane and toluene daily \ce{O_x} production percentage contributions by carbon number of degradation products.}
    \vspace{-4mm}
    \label{f:percent_carbon}
\end{figure}

\begin{figure}
    \centering
    \includegraphics[width=\textwidth]{img/TOL_HO2x_intermediates}
    \vspace{0mm}
    \caption{Day-time \ce{HO2_x} production and consumption budgets from toluene degradation in (a) MCM v3.2, (b) CRI v2 and (c) RACM2.}
    \vspace{-4mm}
    \label{f:toluene_HO2x}
\end{figure} 

The different NMVOC \ce{O_x} production time profiles in Figure \ref{f:TOPP_dailies} result from the different rates at which NMVOCs break up into smaller fragments \citep{Butler:2011}.
The daily percent contributions from the carbon numbers of the \ce{O_x} producing degradation products during pentane and toluene degradation are illustrated in Figure \ref{f:percent_carbon}.
The contribution of \ce{HO2} and operator species, such as XO2 in CBM-IV and CB05, are allocated to the carbon number of its source. 

Figure \ref{f:percent_carbon} indicates that degradation products having the same carbon number as the emitted NMVOC have more influence on \ce{O_x} production throughout the model run in near-explicit than less-explicit mechanisms.
Toluene degradation in MOZART-4 is an exception.

The rapid loss of toluene's reactive carbon shown in Figure \ref{f:net_carbon_loss} is responsible for the sharp decrease in \ce{O_x} production over the MOZART-4 model run.
Even though C7 degradation products have a larger relative contribution to \ce{O_x} production in MOZART-4 than in \mbox{MCM v3.2}, the absolute \ce{O_x} production is lower in MOZART-4.
The loss of toluene's reactive carbon determines the \ce{O_x} production from toluene degradation in MOZART-4.

C1 and C2 degradation products are the sole influence on \ce{O_x} production from alkane degradation in CBM-IV and CB05.
Alkane degradation is depicted using PAR, whose degradation consists of mechanism species having either one or two carbon atoms.
PAR represents singly bonded carbons and alkane emissions are scaled by the carbon number to account for increased \ce{O_x} production with alkane carbon number.
\citet{Hogo:1989} and \citet{Yarwood:2005} contain details for the CBM-IV and CB05, whilst Section \ref{ss:mechanisms} has a brief summary.

CRI v2 and RACM2 toluene degradation differs from that in other mechanisms as maximum \ce{O_x} production is reached on the second day. 
This is due to higher second day \ce{HO2} production in CRI v2 and RACM2 which is not the case for MCM v3.2.  
Figure \ref{f:toluene_HO2x} illustrates the day-time \ce{HO2_x} (= \ce{HO2 + HO2NO2}) production and consumption budgets allocated to the responsible reactions in MCM v3.2, CRI v2 and RACM2. 
\ce{HO2_x} production influences \ce{O_x} production through \ce{HO2} converting NO to \ce{NO2} via \reactionref{r:HO2_NO}.

\mbox{Figure \ref{f:toluene_HO2x}} shows more \ce{HO2_x} production from the reaction of CARB3 and OH in \mbox{CRI v2} than its corresponding MCM v3.2 reaction (\mbox{GLYOX + OH}).  
Glyoxal is represented as CARB3 in CRI v2 and GLYOX in MCM v3.2, but there are differences in the chemistry of these species.
In CRI v2, CARB3 is only produced from aromatic degradation whilst GLYOX is also a degradation product of other non-aromatic NMVOCs in MCM v3.2. 

CRI v2 glyoxal degradation is through reaction with OH and photolysis whilst extra degradation options are available in MCM v3.2. 
Moreover, the rate constant for the reaction with OH radical in CRI v2 is $\sim$ $15$\% faster than in MCM v3.2. 

Glyoxal has three available photolysis pathways in MCM v3.2 and only one in \mbox{CRI v2}, these are outlined in Table \ref{t:glyoxal}. 
The additional photolysis pathways in MCM v3.2 are non-\ce{HO2_x} producing pathways leading to less \ce{HO2_x} production.  
The combination of the higher rate constant for the glyoxal--OH reaction and additional \ce{HO2_x} production during photolysis are responsible for the higher \ce{HO2_x} production in CRI v2. 
{
    \renewcommand{\arraystretch}{1.3}
    \begin{table}
        \centering
        \small
        \begin{tabular}{lP{6.8cm}P{3.0cm}}
            \hline \hline
            \textbf{Mechanism} & \textbf{Photolysis Pathway} & \textbf{Rate Parameter} \\ \hline \hline
            \multirow{3}{*}{MCM v3.2} & GLYOX + hv = CO + CO + H2 & J$_{31}$ \\
            & GLYOX + hv = HCHO + CO & J$_{32}$ \\
            & GLYOX + hv = CO + CO + HO2 + HO2 & J$_{33}$ \\ \hline
            CRI v2 & CARB3 + hv = CO + CO + HO2 + HO2 & J$_{33}$ \\ \hline \hline
        \end{tabular}
        \vspace{1mm}
        \caption{Glyoxal photolysis in MCM v3.2 and CRI v2 with specified rate parameters.}
        \vspace{-4mm}
        \label{t:glyoxal}
    \end{table}
}

RACM2 \ce{HO2_x} production during toluene degradation includes reactions unique to RACM2.
Initial OH reaction produces the lumped mechanism species TR2 which degrades immediately producing the epoxy species EPX along with \ce{HO2}.
Epoxy production in \mbox{MCM v3.2} is a minor pathway of initial toluene oxidation.
The EPX ozonolysis rate constant is $100$ times faster than the analogous \mbox{MCM v3.2} species (TLEPOXMUC).
EPX ozonolysis yields $1.5$ \ce{HO2}, thus the reacted \ce{O3} is regenerated with extra \ce{HO2}.
This causes the RACM2 toluene degradation second day \ce{O_x} production maximum.

\subsection[Reactive Carbon Loss during Ox Production]{Reactive Carbon Loss during \ce{O_x} Production} \label{ss:carbon_loss}

\begin{figure}
    \centering
    \includegraphics[width=\textwidth]{img/net_reactive_carbon_loss}
    \vspace{0mm}
    \caption{Daily net rate of carbon loss during \ce{O_x} production resulting from pentane and toluene degradation.}
    \vspace{-4mm}
    \label{f:net_carbon_loss}
\end{figure}

\begin{figure}
    \centering
    \includegraphics[width=\textwidth]{img/CH3CO3_budget_comparison}
    \vspace{0mm}
    \caption{Day-time \ce{CH3CO3} production and loss budgets in (a) MCM v3.2 and \mbox{(b) MOZART-4}.}
    \vspace{-4mm}
    \label{f:CH3CO3_budget}
\end{figure}

Day-time net carbon loss during \ce{O_x} production from pentane and toluene degradation is plotted in Figure \ref{f:net_carbon_loss}.
%The net carbon yield of each \ce{O_x} producing reaction was calculated, multiplied by the reaction rate and then integrated over each day-time period.
This analysis complements that of Section \ref{ss:c_number} by comparing the VOC break down rate into smaller fragments.
The supplement contains additional plots showing the reactions responsible for carbon loss in pentane and toluene degradation.

RACM2 pentane degradation has the largest carbon loss rate and a different time series to RADM2 and RACM, its previous versions.  
The reaction of HC5P (pentyl peroxy radical) with NO is largely responsible for carbon loss.
The reaction products and their yields were updated from RACM to RACM2 due to the increased speciation of the oxygenated products.
For example, ALD describes acetaldehyde and higher aldehyde chemistry in RACM whilst in RACM2 acetaldehyde is represented explicitly (ACD) and ALD represents C3 and higher aldehydes \citep{Goliff:2013}.
These updates result in more carbon loss in RACM2 through \mbox{HC5P + NO} reaction.

Another contributing factor to increased carbon loss in RACM2 is the updated reaction of HC3P (propyl peroxy radical) with NO.
In RACM, this reaction gave net carbon gain whereas in RACM2 it contributes to net carbon loss.

MOZART-4 pentane degradation has high carbon loss via the \ce{CH3CO3 + NO} reaction.
This reaction leads to carbon loss in every mechanism however it has the most influence in MOZART-4. 
Figure \ref{f:CH3CO3_budget} shows extra \ce{CH3CO3} production through the \mbox{MEKO2 + NO} reaction.
This reaction contributes twice to \ce{CH3CO3} production -- as a primary product and reaction of OH with another product, \ce{CH3CHO}.

CBM-IV and CB05 show little carbon loss throughout pentane degradation.
Since pentane degradation is described by C2 and C1 products there is little carbon be lost.

MOZART-4 and RADM2 have the highest loss carbon rates during toluene degradation.
The high net carbon loss from BIGALD (unsaturated dicarbonyls) photolysis coupled with large BIGALD yields during MOZART-4 toluene degradation reactions leads to rapid carbon loss.
BIGALD is produced from other reactions that result in carbon loss: \mbox{NO + TOLO2} and \mbox{NO2 + XOH}, where TOLO2 is the peroxy radical from \mbox{toluene + OH} reaction and XOH represents \ce{C7H10O6} \citep{Emmons:2010}.

RADM2 carbon loss during toluene degradation is mainly from the \mbox{NO + TOLP} reaction, TOLP is the peroxy radical formed from toluene OH oxidation.
RACM and RACM2 aromatic degradation has been heavily updated from RADM2 and this reaction has been removed altogether \citep{Stockwell:1997, Goliff:2013}.

CRI v2 has faster carbon loss than the MCM v3.2.
Both branches of the \mbox{NO + RA16O2} reaction, RA16O2 is the peroxy radical formed from toluene OH oxidation, are responsible for the carbon loss in CRI v2.

\subsection[Ox Production Efficiency of Carbon Number Sizes]{\ce{O_x} Production Efficiency of Carbon Number Sizes} \label{ss:OPE}

\begin{figure}
    \centering
    \includegraphics[width=\textwidth]{img/C_number_distribution_OPE}
    \vspace{0mm}
    \caption{Pentane and toluene day-time \ce{O_x} production efficiency normalised by VOC emissions per carbon number size of \ce{O_x} producing degradation products.}
    \vspace{-4mm}
    \label{f:OPE}
\end{figure}

The reactive carbon loss analysis of Section \ref{ss:carbon_loss} shows that reduced mechanisms tend to loss carbon quicker than in more explicit mechanisms.
Despite this, reduced mechanisms still produce significant \ce{O_x} amounts during VOC degradation as shown in Figure \ref{f:TOPP_dailies}.
\ce{O_x} production efficiencies were calculated for pentane and toluene degradation products carbon numbers to investigate how reduced mechanisms manage to produce \ce{O_x} despite large carbon loss.

The \ce{O_x} production efficiency (OPE) calculation followed the definition in \citet{Kleinman:2002}.
Instantaneous \ce{O_x} production, using the definition of the \ce{O_x} family used in this study, was calculated and distributed to the different carbon numbers of pentane and toluene degradation products.
This was then normalised by the emissions of the respective VOCs and then divided by the \ce{NO_z} production, which following \citet{Kleinman:2002} was defined as \ce{HNO3} production.

\subsection{Radical Sinks and Sources} \label{ss:radicals}

%\begin{figure}
    %\centering
    %\includegraphics[width=\textwidth]{img/radicals_budget_assigned}
    %\vspace{0mm}
    %\caption{Radical family production and loss budgets in (a) MCM v3.2, \mbox{(b) MCM v3.1}, \mbox{(c) CRI v2}, (d) RADM2, (e) RACM, (f) RACM2, (g) MOZART-4, (h) CBM-IV and \mbox{(i) CB05}.}
    %\vspace{-4mm}
    %\label{f:radical_budgets} 
%\end{figure} 

%\begin{figure}
    %\centering
    %\includegraphics[width=\textwidth]{img/HCHO_budget_comparison}
    %\vspace{0mm}
    %\caption{Formaldehyde production budgets in (a) MCM v3.2, (b) MCM v3.1, \mbox{(c) CRI v2}, (d) RADM2, (e) RACM, (f) RACM2, (g) MOZART-4, (h) CBM-IV and \mbox{(i) CB05}.}
    %\vspace{-4mm}
    %\label{f:HCHO_budgets} 
%\end{figure} 

\ce{O_x} production is directly related to the conversion of NO to \ce{NO2} by peroxy radicals. 
Moreover, in this study maximum \ce{O3} production was achieved by emitting the amount of NO required to balance the radical source at each time step. 
Differences in radical chemistry will not only impact \ce{O_x} production but also NO emissions.
%relate to atmospheric regimes??

The sources and sinks of the radical family, consisting of all radicals, is depicted for each mechanism in Figure \ref{f:radical_budgets}.
The radical production from \ce{O(^1D)}, HONO, \ce{HO2NO2}, PAN and \ce{CH3O2NO2} decomposition are all affected by the NO source of each mechanism.
The same is true for the radical loss through \ce{HNO3}, HONO, \ce{HO2NO2}, PAN and \ce{CH3O2NO2} formation.
However, radical production from formaldehyde (HCHO) photolysis impacts each mechanism differently.
VOC degradation is the sole formaldehyde source in this study and any differences in its treatment will impact on the radical budget.

HCHO production and consumption budgets for each mechanism are illustrated in Figure \ref{f:HCHO_budgets}.

In general, photolysis is the major radical production process and the reactions of radicals with other species, such as NO and \ce{HO2}, is the main radical sink. 
Reduced mechanisms include other processes to maintain radical production.

Figure shows that initial VOC oxidation contributes to radical production in RADM2. 
This is represented by
\begin{reactionlist}
    \reactionitem{HC5 + OH}{HC5P + 0.25 XO2 + H2O}{new}{r:HC5_OH}
\end{reactionlist}
where HC5P is the pentyl peroxy radical and XO2 is an operator species which accounts for extra NO to \ce{NO2} conversions. 
XO2 is included in the radical family as it functions as a peroxy radical. 
This additional radical production pathway is responsible for the net first day radical production in RADM2.

The reactions of radicals lead to large net radical production in RACM after the first day. 
The main source being the reaction of HC5P with NO, 
\begin{center}
\refstepcounter{reaction}\label{r:HC5P_NO}
    \begin{tabular}{l@{\hskip 0.3cm}c@{\hskip 0.3cm}l@{\hskip 0.2cm}r}
        HC5P + NO & \reaction & 0.021 HCHO + 0.211 ALD + 0.722 KET & \reactionref{r:HC5P_NO} \\
        & & \hspace{2mm} + 0.599 HO2 + 0.031 MO2 + 0.245 ETHP & \\
        & & \hspace{2mm} + 0.334 XO2 + 0.124 ONIT + 0.876 NO2 &
    \end{tabular}
\end{center}
where HO2, MO2, ETHP and XO2 are the produced radical species. 
The larger net RADM2 and RACM radical production in Figure is traced back to the extra radical production from reactions \reactionref{r:HC5_OH} and \reactionref{r:HC5P_NO} respectively.

Radical production in CBM-IV and CB05 is largely impacted by NMVOC initial oxidation and production from other radical reactions. 
Pentane degradation in these mechanisms is very similar and the CB05 reactions are considered here. 

Initial pentane degradation is via the PAR reaction with OH,
\begin{center}
\refstepcounter{reaction}\label{r:PAR_OH}
    \begin{tabular}{l@{\hskip 0.3cm}c@{\hskip 0.3cm}l@{\hskip 0.2cm}r}
        PAR + OH & \reaction & 0.87 XO2 + 0.13 XO2N + 0.11 HO2 + 0.06 ALD2 & \reactionref{r:PAR_OH} \\
        & & \hspace{2mm}  + 0.76 ROR + 0.05 ALDX - 0.11 PAR &
    \end{tabular}
\end{center}
where XO2, HO2 and ROR are the produced radical species. 
The mechanism species ROR decomposes immediately to either HO2 or \mbox{0.96 XO2 + 0.04 XO2N + 0.94 HO2 + 0.6 ALD2 + 0.02 ROR + 0.5 ALDX - 2.1 PAR}, once again XO2, HO2 and ROR are the produced radical species. 
The very fast decomposition of ROR and the use of XO2 to rapidly convert NO to \ce{NO2} leads to a rapid loss of radicals. 
The large production and consumption processes are balanced so that the net radical production correlates with that of the MCM v3.2.

The first day net radical production due to aromatic degradation are overestimated using RADM2 and RACM2 chemistry and underestimated in RACM. 
The RACM underestimation results from the degradation chemistry discussed in Section \ref{sss:aromatic}.

Figure indicates that the RACM2 overestimation is due to radical production from reactions that do not involve radicals. 
The ozonolysis of the unsaturated dicarbonyl mechanism species DCB1 and DCB2 as well as the epoxy mechanism species EPX are additional reactions used to produce radicals.

The MCM v3.2 species TLEPOXMUC is analogous to EPX in RACM2. 
The ozonolysis rate constants are \mbox{$5 \times 10^{-18}$} and \mbox{$1 \times 10^{-16}$ cm$^3$ s$^{-1}$} respectively. 
The RACM2 rate constant is 20 times larger than that of the MCM v3.2 leading to excess radical production in RACM.

The ozonolysis of dicarbonyl species are included in RACM2 based on the recommendations of \citet{Bierbach:1994}. 
Dicarbonyl ozonolysis was not included in the MCM due to the uncertainties of these reactions \citep{Bloss:2005}.  
This additional radical source leads to larger net radical production.

