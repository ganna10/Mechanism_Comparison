
This section looks into more detail of how \ce{O_x} is produced in the chemical mechanisms of this study by comparing the treatment of the degradation products.
\ce{O_x} production during pentane and toluene degradation is compared between the mechanisms by analysing the distribution of the carbon number of the degradation products, the loss rates of reactive carbon and the \ce{O_x} production efficiences.

Some mechanism species in RADM2, RACM and RACM2 have fractional carbon numbers \citep{Stockwell:1990, Stockwell:1997, Goliff:2013}.
\ce{O_x} production from these mechanism species was assigned as \ce{O_x} production of the nearest integral carbon number.

Many reduced mechanisms use operator species as a surrogate for peroxy radicals during VOC degradation, this enables these mechanisms to produce more \ce{O_x} without including many additional species in the mechanism.
\ce{O_x} production from operator species was assigned as \ce{O_x} production from the organic degradation species that produced the operator.
This allocation was also used to assign \ce{O_x} production from \ce{HO2} via \reactionref{r:HO2_NO}.

\subsection[Carbon Number of Ox Producing Degradation Products]{Carbon Number of \ce{O_x} Producing Degradation Products} \label{ss:c_number} %comparison by carbon number breakdown

\begin{figure}
    \centering
    \includegraphics[width=1.10\textwidth]{img/Ox_production_by_C_number}
    \vspace{0mm}
    \caption{\ce{O_x} production during pentane and toluene degradation is attributed to the number of carbons of the degradation products for each mechanism.}
    \vspace{-4mm}
    \label{f:carbon}
\end{figure}

The time dependent \ce{O_x} production of the different NMVOC in Figure \ref{f:TOPP_dailies} results from the varying rates at which the NMVOC break up into smaller fragments \citep{Butler:2011}.
To investigate whether the time dependent \ce{O_x} production from a VOC differs between mechanisms are due to different rates of break down the day-time \ce{O_x} production budgets are allocated to the carbon numbers of the degradation products during pentane and toluene degradation in Figure \ref{f:carbon}.  

The \ce{O_x} production during pentane degradation of the same number of carbons as the emitted VOC retains more influence on \ce{O_x} production in the MCMs (Figure \ref{f:carbon}).
As the influence of the larger sized degradation fragments reduces quicker in reduced mechanisms this indicates that VOC degradation proceeds quicker in reduced mechanisms which results in a lower total \ce{O_x} production.
A quicker VOC break down would also imply that the reduced mechanisms are quicker at removing reactive carbon from the system, and this is analysed in Section \ref{ss:carbon_loss}.

Pentane is represented as $5$ PAR in CBM-IV and CB05, where PAR represents the paraffin (C--C) bond \citep{Gery:1989, Yarwood:2005}, and PAR degradation is described using mechanism species having either one or two carbons.
\ce{O_x} production is sustained throughout PAR degradation by recycling of \ce{O_x} producing mechanism species.
An example of this product recycling is the oxy organic radical ROR is a primary degradation product of PAR and further degradation of ROR also produces ROR with a product yield of $0.02$ \citep{Gery:1989}.
This approach of using representing alkanes as a multiple of the base C--C bonds limits the amount of \ce{O_x} that can be produced during pentane degradation as this reduces the number of available pathways for \ce{O_x} production.

\ce{O_x} in reduced mechanisms during toluene degradation is not produced from all of the same degradation sizes as the MCM v3.2 (Figure \ref{f:carbon}).
That some sizes of degradation products are missing during reduced mechanism chemistry also indicates that reduced mechanisms are losing reactive carbon quicker then in more explicit mechanisms.
If reduced mechanisms lose reactive carbon at a quicker rate than the MCM v3.2 this would mean that the reduced mechanisms could not produce the same amount of \ce{O_x} as the MCM as the degradation chains that produce \ce{O_x} are terminated quicker.
The net loss of reactive carbon during pentane and toluene degradation is analysed in more detail in Section \ref{ss:carbon_loss}.

\subsection[Reactive Carbon Loss during Ox Production]{Reactive Carbon Loss during \ce{O_x} Production} \label{ss:carbon_loss}

\begin{figure}
    \centering
    \includegraphics[width=\textwidth]{img/net_reactive_carbon_loss_pentane_toluene}
    \vspace{0mm}
    \caption{Daily net carbon loss rate during \ce{O_x} production from pentane and toluene degradation.}
    \vspace{-4mm}
    \label{f:net_carbon_loss}
\end{figure}

The analysis of the \ce{O_x} production allocated to the different sizes of the pentane and degradation products in Section \ref{ss:c_number} indicates that reduced mechanisms may lose reactive carbon quicker than more explicit mechanisms.
This section compares the reactive carbon loss rates during pentane and toluene degradation in Figure \ref{f:net_carbon_loss} by calculating the net carbon yield from each \ce{O_x} producing reaction and multiplying this yield by the reaction rate.

Pentane degradation shows comparable loss rates of reactive carbon between the MCM v3.2 and the other mechanisms, besides CBM-IV and CB05.
The loss rates of reative carbon during pentane degradation in CBM-IV and CB05 is represented by C1 and C2 species this leads to little carbon loss during pentane degradation in CBM-IV and CB05, hence the lumped species reduced mechanisms have different loss of carbon profiles to other types of mechanisms.

RADM2, RACM and RACM2 have the highest loss rates of reactive carbon on the first day (Figure \ref{f:net_carbon_loss}), this leads to these mechanisms being unable to reach the same \ce{O_x} production as other mechanisms on subsequent days (Figure \ref{f:carbon}).
CRI v2 and MOZART-4 have slower loss rates of reactive carbon than the MCM v3.2 and this results in higher \ce{O_x} production than RADM2, RACM and RACM2 (Figure \ref{f:carbon}).

Although MOZART-4, CBM-IV and CB05 lose reactive carbon at a slower rate than the MCM v3.2, \ce{O_x} production from these mechanisms is lower than the MCM v3.2.
This lower \ce{O_x} production results from a lower \ce{O_x} production efficiency of these mechanisms and is detailed in Section \ref{ss:OxPE}.

All reduced mechanisms have quicker loss rate of reactive carbon during toluene degradation (Figure \ref{f:net_carbon_loss}).
In particular, MOZART-4 loses reactive carbon at least twice as fast as other mechanisms and this explains why \ce{O_x} production during toluene degradation in MOZART-4 sharply decreases after the first day (Figure \ref{f:carbon}).
This same trend is also seen in the \ce{O_x} production during toluene degradation in CBM-IV and CB05 (Figure \ref{f:carbon}).

CRI v2 and RACM2 both have a higher loss rate of reactive carbon during toluene degradation on the second day compared to the MCM v3.2 but the both have in increase in \ce{O_x} production on the second day (Figure \ref{f:carbon}) and this is due to a higher Ox production efficiency in CRI v2 and RACM2 compared to the MCM v3.2 (Section \ref{ss:OxPE}).

The \ce{CH3CO3 + NO} reaction is the main source of reactive carbon loss during pentane degradation in all mechanisms.
Reactive carbon loss during toluene degradation also has significant loss of carbon from the \ce{CH3CO3 + NO} reaction but the degradation of the toluene peroxy radical formed after initial OH-oxidation also leads to significant reactive carbon loss.
The supplement to this paper includes analysis of which reactions are the main sources of reactive carbon loss during pentane and toluene degradation.

Mechanisms which have high reactive carbon loss rates during the degradation of a VOC limit the amount of \ce{O_x} that can be produced.
Loss of reactive carbon is the constraining factor during toluene degradation in all reduced mechanisms and is a limiting factor during pentane degradation in RADM2, RACM and RACM2.
The fact that reduced mechanisms are still able to produce similar amounts of \ce{O3} (Figure \ref{f:time_series}) shows that loss of reactive carbon is only one factor, the \ce{O_x} production efficiency (ratio of \ce{O_x} production per loss of \ce{O_x}) is another important factor when considering \ce{O_x} production and is discuss in Section \ref{ss:OxPE}.

\subsection[Ox Production Efficiency of Degradation Fragments]{\ce{O_x} Production Efficiency of Degradation Fragments} \label{ss:OxPE}

\begin{figure}
    \centering
    \includegraphics[width=1.1\textwidth]{img/OxPEs_by_C_number}
    \vspace{0mm}
    \caption{\ce{O_x} production efficiency of the degradation fragments during pentane and toluene degradation.}
    \vspace{-4mm}
    \label{f:OxPE}
\end{figure}

Reduced mechanisms can produce similar amounts of \ce{O_x} during NMVOC degradation to more-explicit mechanisms as shown in Figure \ref{f:carbon} but the \ce{O_x} is produced via the degradation fragments with a smaller number of carbons than in explicit mechanisms (Section \ref{ss:c_number}).
Also, in Section \ref{ss:carbon_loss}, the loss rates of reactive carbon during VOC degradation plays a factor in determining how much \ce{O_x} can be produced throughout the degradation of the VOC.
To determine how effective reduced mechanisms are at producing \ce{O_x} the \ce{O_x} production efficiency (OxPE) was calculated for the different sizes of the degradation products.
The OxPE of each degradation fragment size was determined by dividing the total \ce{O_x} production from each degradation fragment size (this was shown in Figure \ref{f:carbon}) by the total \ce{O_x} consumption, Figure \ref{f:OxPE} shows this for pentane and toluene degradation.

Reactions during VOC degradation that lead to increased OxPE are also those that result in loss of reactive carbon (Section \ref{ss:carbon_loss}).
For example, the \ce{CH3O2 + NO} reaction is the main C1 \ce{O_x} production pathway during pentane degradation in reduced mechanisms with a larger contribution to the MCMs.
Reduced mechanisms have higher acetaldehyde (\ce{CH3CHO}) production which results in a degradation chain leading to higher \ce{CH3O2} amounts.
The OH-oxidation of \ce{CH3CHO} produces \ce{CH3CO3} whose reaction with NO produces \ce{O_x} as well as \ce{CH3O2}.

Increased acetaldehyde amounts in reduced mechanisms result in increased OxPEs but also increased reactive carbon loss as noted in Section \ref{ss:carbon_loss}, the \ce{CH3CO3 + NO} reaction is also a major source of reactive carbon loss in all mechanisms. 
Thus increased \ce{CH3CO3} levels produced from acetaldehyde production in less-explicit mechanisms leads to higher reactive carbon loss rates.

Acetaldehyde is used as a surrogate for higher aldehydes in many reduced mechanisms, such as CBM-IV, CB05, RADM2 and RACM, this leads to a quicker break down of the parent VOC in the case of pentane as pentane has 5 carbons and acetaldehyde as 2.
This quicker breakdown results in faster production of \ce{O_x} but that cannot be sustained over the model run and is evident in the similar \ce{O_x} production on the first day but not on the second (Figure \ref{f:carbon}).
A better representation of aldehydes in reduced mechanisms would improve the time-dependent \ce{O_x} production.

OxPE during toluene degradation is mainly attributed to the reactions of the toluene peroxy radical formed during the initial OH-oxidation of toluene.
The reactions of the toluene peroxy radical are also the main source of reactive carbon loss during toluene degradation (Section \ref{ss:carbon_loss}).
