
This section looks into more detail of how \ce{O_x} is produced in the chemical mechanisms of this study by comparing the treatment of the degradation products.
\ce{O_x} production during pentane and toluene degradation is compared between the mechanisms by analysing the distribution of the carbon number of the degradation products, the loss rates of reactive carbon and the \ce{O_x} production efficiences.

Some mechanism species in RADM2, RACM and RACM2 have fractional carbon numbers \citep{Stockwell:1990, Stockwell:1997, Goliff:2013}.
\ce{O_x} production from these mechanism species was assigned as \ce{O_x} production of the nearest integral carbon number.

Many reduced mechanisms use operator species as a surrogate for peroxy radicals during VOC degradation, this enables these mechanisms to produce more \ce{O_x} without including many additional species in the mechanism.
\ce{O_x} production from operator species was assigned as \ce{O_x} production from the organic degradation species that produced the operator.
This allocation was also used to assign \ce{O_x} production from \ce{HO2} via \reactionref{r:HO2_NO}.

\subsection[Carbon Number of Ox Producing Degradation Products]{Carbon Number of \ce{O_x} Producing Degradation Products} \label{ss:c_number} %comparison by carbon number breakdown

\begin{figure}
    \centering
    \includegraphics[width=1.10\textwidth]{img/Ox_production_by_C_number}
    \vspace{0mm}
    \caption{\ce{O_x} production during pentane and toluene degradation is attributed to the number of carbons of the degradation products for each mechanism.}
    \vspace{-4mm}
    \label{f:carbon}
\end{figure}

The time dependent \ce{O_x} production of the different NMVOC in Figure \ref{f:TOPP_dailies} results from the varying rates at which the NMVOC break up into smaller fragments \citep{Butler:2011}.
To investigate whether the time dependent \ce{O_x} production from a VOC differs between mechanisms are due to different rates of break down the day-time \ce{O_x} production budgets are allocated to the carbon numbers of the degradation products during pentane and toluene degradation in Figure \ref{f:carbon}.  

The \ce{O_x} production during pentane degradation of the same number of carbons as the emitted VOC retains more influence on \ce{O_x} production in the MCMs (Figure \ref{f:carbon}).
As the influence of the larger sized degradation fragments reduces quicker in reduced mechanisms this indicates that VOC degradation proceeds quicker in reduced mechanisms which results in a lower total \ce{O_x} production.
A quicker VOC break down would also imply that the reduced mechanisms are quicker at removing reactive carbon from the system, and this is analysed in Section \ref{ss:carbon_loss}.

Pentane is represented as $5$ PAR in CBM-IV and CB05, where PAR represents the paraffin (C--C) bond \citep{Gery:1989, Yarwood:2005}, and PAR degradation is described using mechanism species having either one or two carbons.
\ce{O_x} production is sustained throughout PAR degradation by recycling of \ce{O_x} producing mechanism species.
An example of this product recycling is the oxy organic radical ROR is a primary degradation product of PAR and further degradation of ROR also produces ROR with a product yield of $0.02$ \citep{Gery:1989}.
This approach of using representing alkanes as a multiple of the base C--C bonds limits the amount of \ce{O_x} that can be produced during pentane degradation as this reduces the number of available pathways for \ce{O_x} production.

\ce{O_x} in reduced mechanisms during toluene degradation is not produced from all of the same degradation sizes as the MCM v3.2 (Figure \ref{f:carbon}).
That some sizes of degradation products are missing during reduced mechanism chemistry also indicates that reduced mechanisms are losing reactive carbon quicker then in more explicit mechanisms.
If reduced mechanisms lose reactive carbon at a quicker rate than the MCM v3.2 this would mean that the reduced mechanisms could not produce the same amount of \ce{O_x} as the MCM as the degradation chains that produce \ce{O_x} are terminated quicker.
The net loss of reactive carbon during pentane and toluene degradation is analysed in more detail in Section \ref{ss:carbon_loss}.

\subsection[Reactive Carbon Loss during Ox Production]{Reactive Carbon Loss during \ce{O_x} Production} \label{ss:carbon_loss}

\begin{figure}
    \centering
    \includegraphics[width=\textwidth]{img/net_reactive_carbon_loss_pentane_toluene}
    \vspace{0mm}
    \caption{Daily net carbon loss rate during \ce{O_x} production from pentane and toluene degradation.}
    \vspace{-4mm}
    \label{f:net_carbon_loss}
\end{figure}

Day-time net carbon loss rates during \ce{O_x} production from pentane and toluene degradation are plotted in Figure \ref{f:net_carbon_loss}.
The net carbon yield of each \ce{O_x} producing reaction was calculated and multiplied by the reaction rate.
The supplement contains plots showing the reactions responsible for carbon loss in pentane and toluene degradation.

RACM2 pentane degradation has the largest carbon loss rate and a different time series to previous versions (RADM2 and RACM).  
The \mbox{NO + HC5P} (pentyl peroxy radical) reaction is largely responsible for carbon loss.
The reaction products and their yields were updated from RACM to RACM2 due to increased oxygenated product speciation in RACM2.
For example, ALD describes acetaldehyde and higher aldehyde chemistry in RACM whilst in RACM2 acetaldehyde is represented explicitly (ACD) and ALD represents C3 and higher aldehydes \citep{Goliff:2013}.
These RACM2 updates result in more carbon loss through \mbox{HC5P + NO} reaction.

Another contributing factor to increased carbon loss in RACM2 is the updated \mbox{NO + HC3P} (propyl peroxy radical) reaction.
In RACM, this reaction gave net carbon gain whereas it contributes to net carbon loss in RACM2.

MOZART-4 pentane degradation has high carbon loss via the \ce{CH3CO3 + NO} reaction.
This reaction leads to carbon loss in every mechanism however it has the most influence in MOZART-4. 
This reaction contributes twice to \ce{CH3CO3} production -- as a primary product and OH-oxidation of \ce{CH3CHO}.

CBM-IV and CB05 show little carbon loss throughout pentane degradation.
Since pentane degradation is described by C2 and C1 products there is little carbon to be lost.

MOZART-4 and RADM2 have the highest loss carbon rates during toluene degradation.
The high net carbon loss from BIGALD (unsaturated dicarbonyls) photolysis coupled with its large yields during MOZART-4 toluene degradation reactions leads to rapid carbon loss.
BIGALD is produced from other reactions that result in carbon loss: \mbox{NO + TOLO2} and \mbox{NO2 + XOH}, where TOLO2 is the peroxy radical from \mbox{toluene + OH} reaction and XOH represents \ce{C7H10O6} \citep{Emmons:2010}.

RADM2 carbon loss during toluene degradation is mainly from the \mbox{NO + TOLP} reaction, TOLP is the peroxy radical formed from toluene OH-oxidation.
RACM and RACM2 aromatic degradation has been heavily updated from RADM2 and this reaction removed altogether \citep{Stockwell:1997, Goliff:2013}.

CRI v2 has faster carbon loss from toluene degradation than the MCM v3.2.
Both branches of the \mbox{NO + RA16O2} reaction, RA16O2 is the peroxy radical formed from toluene OH-oxidation, are responsible for the carbon loss in CRI v2.

\subsection[Degradation Fragment Ox Production Efficiency]{Degradation Fragment \ce{O_x} Production Efficiency} \label{ss:OxPE}

\begin{figure}
    \centering
    \includegraphics[width=1.1\textwidth]{img/OxPEs_by_C_number}
    \vspace{0mm}
    \caption{Degradation product \ce{O_x} production efficiency during pentane and toluene degradation.}
    \vspace{-4mm}
    \label{f:OxPE}
\end{figure}

Reduced mechanisms have similar \ce{O_x} production to more-explicit mechanisms despite quicker reactive carbon loss.
Moreover, C1 and C2 degradation products have higher influence on \ce{O_x} production in less-explicit mechanisms.
\ce{O_x} production efficiencies (OxPE) for the degradation fragment sizes were calculated by dividing total \ce{O_x} production from the degradation fragment size by total \ce{O_x} consumption.
This is illustrated in Figure \ref{f:OxPE} for pentane and toluene degradation.

All less-explicit mechanisms have higher C1 and C2 OxPEs than the MCMs during pentane and toluene degradation.
CBM-IV and CB05 have low OxPEs during toluene degradation mirroring the low \ce{O_x} production noted in Section \ref{sss:aromatic}.
Reduced mechanisms simulate similar \ce{O_x} production to more-explicit mechanisms by being more efficient at producing \ce{O_x} particularly from smaller sized fragments.

The \ce{CH3O2 + NO} reaction is the main C1 \ce{O_x} production pathway in reduced mechanisms with a larger contribution to the MCMs.
Reduced mechanisms have higher aldehyde production starting a degradation chain resulting in higher \ce{CH3O2} amounts.
For example, acetaldehyde (\ce{CH3CHO}) OH-oxidation produces \ce{CH3CO3} whose reaction with NO produces \ce{O_x} as well as \ce{CH3O2}.
The supplement shows total aldehyde production budgets from pentane and toluene degradation in each mechanism.  

Pentane degradation has larger total aldehyde production in all reduced mechanisms.
Whilst during toluene degradation, total aldehyde production is more variable between reduced mechanisms.
The overall total aldehyde production budgets mirrors the OxPEs of Figure \ref{f:OxPE}.

Increased aldehyde amounts in reduced mechanisms result in increased OxPEs but also increased reactive carbon loss.
Section \ref{ss:carbon_loss} noted that the \ce{CH3CO3 + NO} reaction is a major source of reactive carbon loss in all mechanisms. 
Thus increased \ce{CH3CO3} levels produced from higher aldehyde production in less-explicit mechanisms leads to higher reactive carbon loss rates.
