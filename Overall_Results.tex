
\subsection[Ozone Time Series and Ox Production Budgets]{Ozone Time Series and \ce{O_x} Production Budgets}

\begin{figure}
    \centering
    \includegraphics[width=0.6\textwidth]{img/O3_mixing_ratios}
    \vspace{1mm}
    \caption{Time series of \ce{O3} mixing ratios.}
    \vspace{-4mm}
    \label{f:time_series}
\end{figure}

\begin{figure}
    \centering
    \includegraphics[width=\textwidth]{img/Ox_production_budgets_by_VOC_de-allocated}
    \vspace{1mm}
    \caption{The day-time \ce{O_x} production budgets allocated to individual VOC in each mechanism.}
    \vspace{-4mm}
    \label{f:Ox_tagged_budgets}
\end{figure}

Figure \ref{f:time_series} shows the time series of \ce{O3} mixing ratios obtained using each mechanism.
The day-time production budgets of the \ce{O_x} family for each mechanism are displayed in Figure \ref{f:Ox_tagged_budgets}, the \ce{O_x} production budgets are allocated to each VOC.
The contributions from the individual VOC in Table \ref{t:initial_conditions} are re-assigned from lumped mechanism emitted species using the inverse calculation when assigning the initial conditions to the respective mechanism species.

The mechanisms including the highest amount of chemical details (MCM v3.2, MCM v3.1 and CRI v2) produce the largest amount of \ce{O3} in Figure \ref{f:time_series}.
Whilst both the CBM-IV and CB05, that include the least amount of chemical detail, produce the lowest amount of \ce{O3}.
This trend is also evident in Figure \ref{f:Ox_tagged_budgets} where the more-detailed mechanisms have largest day-time \ce{O_x} production and the less-detailed mechanisms have the lowest day-time \ce{O_x} production.

The \ce{O3} mixing ratio levels in RACM are much lower on the first day compared to mechanisms having a similar amount of chemical detail.
These low \ce{O3} amounts are due to the lack of \ce{O_x} production from aromatic VOC in RACM, which differs from \ce{O_x} production on the first day in all other mechanisms.
This lack of \ce{O_x} production from aromatic species arises from aromatic degradation chemistry which results in net \ce{O_x} consumption and is further detailed in \mbox{Section \ref{sss:aromatic}}.

Low \ce{O_x} production in CBM-IV and CB05 is related to the lower OH levels produced during the model runs using these mechanisms compared to the MCM v3.2. 
The lower OH levels arise from the high NO emissions required for maximal \ce{O_x} production conditions in CBM-IV and CB05 --- high NO levels suppresses the OH amounts through \ce{HNO3} formation \reactionref{r:NO2_OH} in which both \ce{NO2} and OH are removed from the system.
Details are found in Section \ref{ss:radicals} which compares the NO emissions calculated for maximal \ce{O_x} production in each mechanism.
Lower \ce{O3} levels using CBM-IV and CB05 compared to other mechanisms have also been noted in previous modelling studies such as \citet{Luecken:2008, Emmerson:2009} and \citet{Saylor:2012}.

\subsection{First Day Ozone Production} \label{ss:day1} %first day comparison

\begin{figure}
    \centering
    \includegraphics[width=\textwidth]{img/first_day_values}
    \vspace{1mm}
    \caption{MCM v3.2 first day TOPP values and corresponding values in each mechanism. The root mean square error (RMSE) of each set of TOPP values is also displayed.}
    \vspace{-4mm}
    \label{f:first_day}
\end{figure}

The first day TOPP values calculated from each mechanism are compared to those obtained with the MCM v3.{2} in Figure \ref{f:first_day}. 
The root mean square error (RMSE) are also illustrated in \mbox{Figure \ref{f:first_day}}.
TOPP values in MCM v3.1 and CRI v2 match those of the MCM v3.2 as indicated by their low RMSE values.
The more reduced mechanisms have a much higher spread in their first day TOPP values for each VOC and this is also confirmed by the large RMSE values.

The TOPP values for aromatic VOC tend to be lower than in MCM v3.2 in many mechanisms, this is particularly the case in RACM and  MOZART-4.
As these TOPP values for aromatic VOC vary the most from MCM v3.2, they are the root of the high RMSE values in RACM and  MOZART-4.
The lower TOPPs for aromatic VOC show that aromatic degradation chemistry is difficult to represent in chemical mechanisms as many products, their yields and reactions are not known or subject to uncertainties \citep{Vereecken:2012}.

The first day TOPP values of $2$-methylpropene in RACM, RACM2, MOZART-4 and CB05 indicate that its degradation is treated differently to the MCM v3.2. 
The variation between RACM, RACM2 and MCM v3.2 arises from differences in $2$-methylpropene ozonolysis rate constants.
In RACM and RACM2, this rate constant is calculated as a weighted mean of all VOC ozonolysis rate constants represented as OLI \citep{Stockwell:1997, Goliff:2013} and is an order of magnitude faster than MCM v3.2 despite a common source (IUPAC).
The faster rate constant results in increased radical production which leads to more \ce{O_x} production than in the MCM v3.2.

In MOZART-4, higher amounts of \ce{CH3CO3}, which also leads to higher amounts of \ce{CH3O2}, during $2$-methylpropene degradation are produced compared to the MCM v3.2.
These higher \ce{CH3CO3} amounts lead to more \ce{O_x} production in MOZART-4 and hence a higher TOPP value compared to the MCM v3.2.

$2$-methylpropene is represented in CB05 as \mbox{FORM + $3$ PAR}, where FORM represents formaldehyde and PAR the paraffin \ce{C-C} bond \citep{Yarwood:2005}. 
The initial formaldehyde oxidation immediately produces radicals which leads to \ce{O_x} production.
On the other hand PAR is a slow reacting species which slows down the \ce{O_x} production resulting in lower \ce{O_x} production than in the MCM v3.2.

\subsection{Ozone Production on Subsequent Days} \label{ss:profiles} %TOPP time series of all species

\begin{figure}
    \centering
    \includegraphics[width=\textwidth]{img/TOPP_daily_values_all_species}
    \vspace{0mm}
    \caption{TOPP value time series for all NMVOCs obtained with each mechanism.}
    \vspace{-4mm}
    \label{f:TOPP_dailies}
\end{figure}

NMVOC daily TOPP value time series are presented in \mbox{Figure \ref{f:TOPP_dailies}}. 
NMVOCs, such as ethene, whose degradation is described using dedicated mechanism species show similar time-dependent \ce{O_x} production.
Higher variability emerges between the time series of those NMVOCs represented by lumped mechanism species, such as pentane.

\ce{O_x} production from alkane degradation has a second day maximum increasing with carbon number in CRI v2 and both MCM mechanisms.
Octane degradation in RADM2, RACM and RACM2 is the exception as it is broken down into smaller sized fragments quicker than the MCM v3.2.
This analysis is outlined in Section \ref{s:detailed_results} and octane results are found in the supplement.

Aromatic VOC \ce{O_x} production has the most variability between mechanisms. 
In particular, zero \ce{O_x} production for toluene and m-xylene when using RACM.

\ce{O_x} production from toluene degradation in CBM-IV and CB05 is much lower than in the MCM v3.2.
This impacts ethylbenzene \ce{O_x} production as it is represented as \mbox{TOL + PAR}. 
Toluene degradation in RACM, CBM-IV and CB05 is summarised in Section \ref{sss:aromatic}.  

\subsubsection{Toluene Degradation in RACM, CBM-IV and CB05} \label{sss:aromatic}

\begin{figure}
    \centering
    \includegraphics[width=\textwidth]{img/TOL_Ox_intermediates}
    \vspace{0mm}
    \caption{Day-time \ce{O_x} production and consumption budgets from toluene degradation in \mbox{(a) MCM v3.2}, (b) RACM, (c) CBM-IV and (d) CB05.}
    \vspace{-4mm}
    \label{f:toluene_Ox}
\end{figure}

\ce{O_x} production and consumption day-time budgets from toluene degradation in \mbox{MCM v3.2}, RACM, CBM-IV and CB05 are depicted in Figure \ref{f:toluene_Ox}. 
The tagging approach allows attribution to the contributing organic reactions.

RACM chemistry results in net \ce{O_x} loss on the first two days in contrast to net \ce{O_x} production using the MCM v3.2.
This is due to several \ce{O_x}-consuming reactions in RACM not present in the MCM.
Cresol OH-adduct mechanism species ADDC ozonolysis contributes the most to \ce{O_x} loss.
A fast rate constant \mbox{($5 \times 10^{-11}$ cm$^3$ s$^{-1}$)} was assigned making it the main ADDC reaction pathway. 
This reaction was included due to improved cresol product yields when comparing RACM predictions with experimental data \citep{Stockwell:1997}.

Other mechanisms including cresol OH-adduct species do not include ozonolysis.
Including aromatic OH-adduct species ozonolysis in RACM results in non-representative \ce{O_x} production. 
This was updated in RACM2 and aromatic OH-adduct species ozonolysis are no longer included.

Cresol OH-oxidation is the highest contributor to both \ce{O_x} production and reactive carbon loss during CBM-IV and CB05 toluene degradation.
Reactive carbon loss leads to overall lower ability to produce \ce{O_x} during toluene degradation in CBM-IV and CB05.
Section \ref{s:detailed_results} details reactive carbon loss analysis and the supplement shows the reactions responsible for reactive carbon loss from toluene degradation.

\subsection{Radical Sources} \label{ss:radicals}

\begin{figure}
    \centering
    \includegraphics[width=\textwidth]{img/radical_production_analysis}
    \vspace{0mm}
    \caption{Net radical production day-time budgets for each mechanism. Net radical production calculated as the difference between radical and \ce{NO_x} yields. O1D represents the reaction of \ce{O(^1D)} with water vapour.}
    \vspace{-4mm}
    \label{f:radical_production} 
\end{figure} 

Maximum \ce{O_x} production was achieved by emitting the NO amount required to balance the radical source at each time step. 
Mechanism differences in radical production lead to different NO emissions.

Figure \ref{f:radical_production} depicts the day-time net radical production budgets in each mechanism allocated to the contributing reactions.
Reactions having net radical to \ce{NO_x} yield were determined for each mechanism and the rate was multiplied by this yield.

CBM-IV and CB05 have much higher net radical production on the first two days than any other mechanism.
Thus larger NO emissions are required for maximal \ce{O_x} production indicating that CBM-IV and CB05 represent \ce{NO_x}-sensitive chemistry.

CBM-IV was developed for the high-\ce{NO_x} conditions of urban and polluted regions \citep{Gery:1989}.
Hence, CBM-IV lacks low-\ce{NO_x} chemistry such as organic peroxide formation.
CB05 includes more oxygenated species so that organic peroxide chemistry is represented \citep{Yarwood:2005}.
However, CB05 net radical production is still significantly large.

The \ce{O(^1D)} and water vapour reaction is the main radical source in each mechanism.
Varying \ce{O3} levels in each mechanism leads to different \ce{O(^1D)} amounts as its main source is \ce{O3} photolysis.
Carbonyl, mainly HCHO, photolysis is the main radical source from organic chemistry.
Moreover, each mechanism has a similar contribution from HCHO photolysis.

Methyl glyoxal photolysis has a varying impact on net radical production in each mechanism.
\mbox{Section \ref{sss:mglyox}} describes methyl glyoxal production in all mechanisms and explains these differences.

Most reduced mechanisms lack sufficient carbonyl species to sustain radical production solely through carbonyl photolysis.
Initial VOC oxidation, NO--\ce{RO2} reactions and ozonolysis are used to further sustain radical production in less-explicit mechanisms.
Examples of non-photolysis radical sources are given in Table \ref{t:thermal_radicals}.

Mechanisms require different NO emissions to simulate \ce{NO_x}-VOC-sensitive conditions.
Thus, a constant NO source may lead to different atmospheric regimes being simulated depending on the mechanism.
This shall be investigated in future work.
{%
    \renewcommand{\arraystretch}{1.3}
    \begin{table}
        \centering
        \small
        \begin{tabular}{lP{3.0cm}P{2.5cm}P{2.5cm}}
            \hline \hline
            \textbf{Mechanism} & \textbf{VOC Oxidation} & \textbf{\ce{RO2 + NO}} & \textbf{Ozonolysis} \\ \hline \hline
            RADM2 & HC5 + OH & NO + TCO3 & \\ \hline
            \multirow{2}{*}{RACM2} & & & DCB + O3 \\
            & & & EPX + O3 \\ \hline
            \multirow{3}{*}{CBM-IV} & C2H4 + OH & C2O3 + NO & \\
            & CH4 + OH & & \\
            & OH + PAR & & \\ \hline
            \multirow{3}{*}{CB05} & C2H6 + OH & CXO3 + NO & \\
            & C2H4 + OH & & \\
            & OH + PAR & & \\ \hline \hline
        \end{tabular}
        \vspace{1mm}
        \caption{Non-photolysis radical producing reactions.}
        \vspace{-4mm}
        \label{t:thermal_radicals}
    \end{table}
}%

\subsubsection{Methyl Glyoxal Production} \label{sss:mglyox}

\begin{figure}
    \centering
    \includegraphics[width=\textwidth]{img/MGLY_VOC_allocated_production_rates}
    \vspace{0mm}
    \caption{Methyl glyoxal production budgets allocated to parent VOCs in each mechanism.}
    \vspace{-4mm}
    \label{f:mglyox_budgets} 
\end{figure} 

Methyl glyoxal is represented in each mechanism -- MGLYOX in both MCM versions, CARB6 in CRI v2, CH3COCHO in MOZART-4 and MGLY in RADM2, RACM, RACM2, CBM-IV and CB05.
NMVOC degradation is the only source and Figure \ref{f:mglyox_budgets} shows the individual sources in each mechanism.

Isoprene and aromatic VOC degradation are the main methyl glyoxal sources in all mechanisms except RADM2, RACM and RACM2.
In these mechanisms, pentane and propane degradation produce the most methyl glyoxal.
MOZART-4 pentane degradation also produces methyl glyoxal.
The supplement shows the pentane degradation reactions influencing the methyl glyoxal production budget.

Acetone degradation in low-\ce{NO_x} conditions produces methyl glyoxal through the acetone peroxy radical reaction with other \ce{RO2} \citep{Fu:2008}.
This is represented in both MCM versions and is a minor source as \ce{NO_x}-VOC-sensitive conditions are used.  
MOZART-4 also includes this chemistry but the rate constant is four times faster than the MCM v3.2, leading to higher MOZART-4 production.

The NO + KETP reaction, KETP is the ketone peroxy radical, produces additional methyl glyoxal under high-\ce{NO_x} conditions in RADM2, RACM and RACM2.
This additional pathway results in higher methyl glyoxal production from alkane degradation in these mechanisms.

Methyl glyoxal production from acetone degradation is not included in CBM-IV and CB05 as high-\ce{NO_x} conditions are assumed.
In CRI v2, the acetone peroxy radical (RN8O2) skips methyl glyoxal production altogether and immediately produces \ce{CH3CO3}.
