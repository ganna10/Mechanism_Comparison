
\subsection[Ozone Time Series and Ox Production Budgets]{Ozone Time Series and \ce{O_x} Production Budgets} \label{ss:O3_time_series}

\begin{figure}
    \centering
    \includegraphics[width=0.8\textwidth]{img/O3_mixing_ratios}
    \vspace{1mm}
    \caption{Time series of \ce{O3} and OH mixing ratios obtained using each mechanism.}
    \vspace{-4mm}
    \label{f:time_series}
\end{figure}

\begin{figure}
    \centering
    \includegraphics[width=\textwidth]{img/Ox_production_budgets_by_VOC_de-allocated}
    \vspace{1mm}
    \caption{Day-time \ce{O_x} production budgets in each mechanism allocated to individual VOC.}
    \vspace{-4mm}
    \label{f:Ox_tagged_budgets}
\end{figure}

\todo{Ox instead of O3?}
Figure \ref{f:time_series} shows the time series of \ce{O3} and OH mixing ratios obtained with each mechanism.
The day-time \ce{O_x} production budgets allocated to VOC for each mechanism are displayed in \mbox{Figure \ref{f:Ox_tagged_budgets}}.
The \ce{O_x} production from lumped mechanism species is re-assigned to the VOC of \mbox{Table \ref{t:initial_conditions}} by applying a contributing factor to the \ce{O_x} production of the lumped species.
For example, TOL in RACM2 represents toluene and ethylbenzene with contributing factors of $0.87$ and $0.13$.
Multiplying these factors to the \ce{O_x} production from TOL gives the \ce{O_x} production from toluene and ethylbenzene in RACM2.

The mechanisms with the most detailed chemistry (MCM v3.2, MCM v3.1 and CRI v2) produce the highest \ce{O3} mixing ratios (Figure \ref{f:time_series}).
Whilst CBM-IV and CB05, with the least chemical detail, produce the lowest \ce{O3} mixing ratios.
The trends in \ce{O3} mixing ratios are mirrored in Figure \ref{f:Ox_tagged_budgets} with mechanisms producing high \ce{O_x} levels having high \ce{O3} mixing ratios.

The \ce{O3} mixing ratios on the first day in RACM are lower than mechanisms having similarly detailed chemistry (such as MOZART-4 or RADM2).
The lower \ce{O3} mixing ratios are due to a lack of \ce{O_x} production from aromatic VOC on the first day in RACM (Figure \ref{f:Ox_tagged_budgets}).
Aromatic degradation chemistry in RACM results in net \ce{O_x} consumption on the first day and is detailed in \mbox{Section \ref{sss:aromatic}}.

The low OH mixing ratios obtained with CBM-IV and CB05 (Figure \ref{f:time_series}) inhibit VOC oxidation leading to lower \ce{O3} mixing ratios in CBM-IV and CB05 compared to the MCM v3.2. 
CBM-IV and CB05 represent the high-\ce{NO_x} chemistry of polluted urban areas leading to larger NO emissions for conditions of maximal \ce{O3} production than any other mechanism, detailed in \mbox{Section \ref{ss:radicals}}.
These large NO emissions induce higher \ce{HNO3} production and deposition leading to greater loss of OH and \ce{NO_x} in CBM-IV and CB05.
Lower \ce{O3} levels using CBM-IV and CB05 compared to other mechanisms have also been noted in previous modelling studies such as \citet{Emmerson:2009} and \citet{Saylor:2012}.

\subsection{First Day Ozone Production} \label{ss:day1} %first day comparison

\begin{figure}
    \centering
    \includegraphics[width=\textwidth]{img/first_day_values}
    \vspace{1mm}
    \caption{The first day TOPP values for each VOC calculated using MCM v3.2 and the corresponding values in each mechanism. The root mean square error (RMSE) of each set of TOPP values is also displayed.}
    \vspace{-4mm}
    \label{f:first_day}
\end{figure}

The first day TOPP values of each VOC calculated using each mechanism are compared to those obtained with the MCM v3.{2} in Figure \ref{f:first_day}.
The root mean square error (RMSE) of all first day TOPP values in each mechanism from those in the MCM v3.2 are also included.
The RMSE values show that mechanisms using lumped structure and lumped molecule techniques have a higher spread in their first day \ce{O_x} production than the CRI v2, which lumps intermediate species.

CBM-IV and CB05 produce low amounts of \ce{O_x} (Figure \ref{f:Ox_tagged_budgets}) resulting in low TOPP values for most VOC.
Low OH levels lead to lower \ce{O_x} production in the CBM-IV and CB05 (Section \ref{ss:O3_time_series}).

The \ce{O_x} production from VOC represented as lumped mechanism species differs the most from that in the MCM v3.2.
For example, all aromatic VOC are represented as toluene in MOZART-4 and the \ce{O_x} production for each VOC is obtained by scaling the toluene \ce{O_x} production as described in Section \ref{ss:O3_time_series}.
Scaling the \ce{O_x} production of lumped species does not capture the \ce{O_x} production from the VOC as its characteristics can be misrepresented.
Representing less reactive aromatic VOC, such as benzene, as toluene leads to higher \ce{O_x} production from benzene whilst representing more reactive aromatic VOC, such as the xylenes, as toluene leads to lower \ce{O_x} production from the xylenes in MOZART-4 than MCM v3.2.

The first day TOPP values of $2$-methylpropene in RACM, RACM2, MOZART-4 and CB05 signifies differences in its degradation to the MCM v3.2. 
The variation between RACM, RACM2 and MCM v3.2 arises from differences in the ozonolysis rate constant of $2$-methylpropene.
This rate constant is an order of magnitude faster in RACM and RACM2 than in MCM v3.2 as the RACM, RACM2 rate constant is a weighted mean of the ozonolysis rate constants of all VOC represented as OLI \citep{Stockwell:1997, Goliff:2013}.
The faster rate constant produces more radicals leading to more \ce{O_x} in RACM and RACM2 than the MCM v3.2.

The degradation of $2$-methylpropene in MOZART-4 produces acetaldehyde (\ce{CH3CHO}) through the reaction of NO with the $2$-methylpropene peroxy radical, whereas no \ce{CH3CHO} is produced during $2$-methylpropene degradation in MCM v3.2.
\ce{CH3CHO} initiates an \ce{O_x}-producing degradation chain involving \ce{CH3CO3} and \ce{CH3O2} producing more \ce{O_x} in MOZART-4 than \mbox{MCM v3.2}.

The $2$-methylpropene representation in CB05 was updated to \mbox{HCHO + $3$ PAR} from \mbox{\ce{CH3CHO} + HCHO + PAR} in CBM-IV, where PAR is the paraffin \ce{C-C} bond \citep{Gery:1989, Yarwood:2005}.
The representation in CB05 includes more of the slower reacting PAR at the expense of the more reactive \ce{CH3CHO} producing less \ce{O_x} than CBM-IV.

\subsection{Ozone Production on Subsequent Days} \label{ss:profiles} %TOPP time series of all species

\begin{figure}
    \centering
    \includegraphics[width=\textwidth]{img/TOPP_daily_time_series_all_VOC}
    \vspace{0mm}
    \caption{TOPP value time series using each mechanism for each VOC.}
    \vspace{-4mm}
    \label{f:TOPP_dailies}
\end{figure}

Time series of daily TOPP values for each VOC are presented in \mbox{Figure \ref{f:TOPP_dailies}}. 
Higher variability in the time dependent \ce{O_x} production is evident for all VOC represented by lumped mechanism species.
The lumped structure approach of CBM-IV and CB05 produces the lowest \ce{O_x} throughout the model run, whilst the lumped intermediate mechanism (CRI v2) leads to similar \ce{O_x} production to the MCM v3.2.

Alkane degradation in CRI v2 and both MCM mechanisms produces a second day maximum in \ce{O_x} that increases with carbon number (Figure \ref{f:TOPP_dailies}).
The increase in \ce{O_x} production on the second day is reproduced by the reduced mechanisms but not the magnitude, especially alkanes represented as lumped species.
This is detailed in Section \ref{sss:alkanes}.
\todo{summarise Ox production from pentane section and forward reference}

The lumped structure approach in CBM-IV and CB05 also does not capture the second day increase in \ce{O_x} production with alkane carbon number.
This approach scales the \ce{O_x} production of PAR (\ce{C-C}) by the carbon number of the alkane leading to a slower increase in \ce{O_x} production with increasing carbon number than the MCM v3.2.
The supplement shows the rate of increase of \ce{O_x} production with alkane carbon number.

The time-dependent \ce{O_x} production during degradation of aromatic VOC indicate different treatments between the mechanisms.
In particular, no \ce{O_x} is produced from aromatic VOC in RACM as aromatic degradation chemistry leads to net \ce{O_x} loss, \ce{O_x} production and loss during toluene degradation is detailed in Section \ref{sss:aromatic}.
Lower \ce{O_x} is produced during degradation of aromatic VOC in all reduced mechanisms due to a quicker loss rate of reactive carbon than the MCM v3.2, presented in Section \ref{ss:carbon_loss}.

\subsubsection[Ox Production during Pentane Degradation]{\ce{O_x} Production during Pentane Degradation} \label{sss:alkanes}

\subsubsection[Ox Production during Toluene Degradation]{\ce{O_x} Production during Toluene Degradation} \label{sss:aromatic}

\begin{figure}
    \centering
    \includegraphics[height=0.98\textheight]{img/TOL_Ox_intermediates}
    \vspace{0mm}
    \caption{Day-time \ce{O_x} production and consumption budgets allocated to the responsible reactions from toluene degradation in all mechanisms.}
    \vspace{-4mm}
    \label{f:toluene_Ox}
\end{figure}

\ce{O_x} production during toluene degradation in Figure \ref{f:TOPP_dailies} has the largest spread between all mechanisms.
In order to explain the reasons for the large spread in \ce{O_x} production the day-time \ce{O_x} production and consumption budgets due to toluene degradation are shown in Figure \ref{f:toluene_Ox}.
The \ce{O_x} production and consumption during toluene degradation are allocated to the responsible reactions by following the tags in each mechanism.

All reduced mechanisms except CRI v2, are unable to produce similar amounts of \ce{O_x} as the MCMv3.2.
The lower amounts of \ce{O_x} produced in reduced mechanisms is constrained by their rapid loss of reactive carbon during degradation of toluene, this analysis is described in Section \ref{ss:carbon_loss}.
Reduced mechanisms have a similar or lower \ce{O_x} production efficiency to the MCM v3.2 as highlighted in Section \ref{ss:OxPE}, in the case of toluene degradation this lower OxPE does not compensate enough for the loss of reactive carbon leading to lower \ce{O_x} production.

The \ce{O_x} production in CRI v2 and RACM2 reaches its maximum on the second day which differs from the MCM v3.2 which produces peak \ce{O_x} on the first day.
The second day maximum of \ce{O_x} production in CRI v2 and RACM2 results from additional \ce{C2H5O2} (C2H5O2 in CRI v2 and ETHP in RACM2) production that is not found in MCM v3.2.
The extra \ce{C2H5O2} production results from differing treatments to the MCM v3.2 in the degradation of unsaturated dicarbonyls producing during toluene degradation.

In CRI v2, unsaturated dicarbonyls from toluene degradation produce propionaldehyde (\ce{C2H5CHO}) leading to \ce{C2H5O2}.
Another degradation pathway for unsaturated dicarbonyls in CRI v2 produces the HOCH2CH2O2 peroxy radical which is also an additional \ce{O_x} source unique to CRI v2 in Figure \ref{f:toluene_Ox}.

Unsaturated dicarbonyls in RACM2 also lead to production of \ce{C2H5CHO} but this is reached via different pathways to the CRI v2.
\ce{C2H5CHO} is produced through the degradation of higher peroxides (OP2) and propyl peroxy radical (HC3P) which are degradation products of unsaturated dicarbonyl, in addition to \ce{C2H5CHO} production directly from unsaturated dicarbonyl degradation.

RACM chemistry results in net \ce{O_x} loss on the first two days in contrast to net \ce{O_x} production in the MCM v3.2 due to several \ce{O_x}-consuming reactions in RACM that are not present in the MCM.
Ozonolysis of the cresol OH-adduct mechanism species ADDC contributes the most to \ce{O_x} loss in RACM.
This reaction was included in RACM due to improved cresol product yields when comparing RACM predictions with experimental data \citep{Stockwell:1997}, other mechanisms that include cresol OH-adduct species do not include ozonolysis.
Including ozonolysis of aromatic OH-adduct species in RACM results in non-representative \ce{O_x} production, these ozonolysis reactions are not included in the updated RACM2.

\subsection{Production of Radicals} \label{ss:radicals}

\begin{figure}
    \centering
    \includegraphics[height=0.98\textheight]{img/radical_NOx_production_budgets}
    \vspace{0mm}
    \caption{Net radical production day-time budgets for each mechanism. Net radical production calculated as the difference between radical and \ce{NO_x} yields. O1D represents the reaction of \ce{O(^1D)} with water vapour.}
    \vspace{-4mm}
    \label{f:radical_production} 
\end{figure} 

Production of radicals impacts \ce{O_x} production by the conversion of NO to \ce{NO2} by peroxy radicals produced from VOC degradation that is typically initiated through OH reaction.
In this study, the amount of NO emissions are also affected by radical production as described in Section \ref{ss:model_setup} conditions of maximal \ce{O_x} production are achieved for each mechanism by emitting the amount of NO required to balance the source of radicals at each time step. 

In order to investigate the differences in NO emissions required by each mechanism to achieve maximal \ce{O_x} production, the net radical to \ce{NO_x} production budgets allocated to the contributing reactions are presented in Figure \ref{f:radical_production}.
The net radical to \ce{NO_x} production budgets were determined using the same calculation that was used to calculate the amount of NO emitted at each time step and hence the net radical production budgets in Figure \ref{f:radical_production} directly correlate to the amount of NO emitted during each mechanism's model run.

CBM-IV and CB05 were developed for the high-\ce{NO_x} conditions of urban and polluted regions \citep{Gery:1989, Yarwood:2005} and both mechanisms represent VOC-sensitive chemistry.
The representation of VOC-sensitive chemistry in CBM-IV and CB05 leads to much higher net radical production on the first two days corresponding to larger NO emissions than any other mechanism.
The higher NO emissions in CBM-IV and CB05 limit the amount of OH available for initiating VOC degradation as larger amount of \ce{HNO3} are produced through \reactionref{r:NO2_OH}.
\ce{HNO3} production also removes \ce{NO_x} from the system which constrains the amount of NO available for reaction with \ce{RO2} that would produce \ce{O_x}.
High NO emissions leads to high \ce{NO2} and increased \ce{HNO3} production via \reactionref{r:NO2_OH} in CBM-IV and CB05.
Increased \ce{HNO3} levels result in an increased sink for OH and \ce{NO_x} in CBM-IV and CB05 due to a higher \ce{HNO3} deposition rate.

Photolysis of carbonyl species, mainly HCHO, is the main source of radicals from organic chemistry.
Reduced mechanisms do not include as many carbonyl species as more explicit mechanisms --- CBM-IV with 2 and RACM2 with 17 carbonyl species represent the lower and upper limits the number of carbonyl species in reduced mechanisms (besides the CRI v2).
The lack of carbonyl species in non-CRI v2 reduced mechanisms means that these reduced mechanisms produce radicals through other reactions such as initial VOC oxidation, NO--\ce{RO2} reactions and ozonolysis.
Examples of radical sources through non-photolysis pathways in reduced mechanisms are given in Table \ref{t:thermal_radicals}.

The mechanisms compared here require different NO emissions to simulate \ce{NO_x}-VOC-sensitive conditions, where the source of radicals is balanced by the source of \ce{NO_x}.
Thus, a constant NO source may lead to different atmospheric regimes (\ce{NO_x}- or VOC-sensitive) being simulated depending on the mechanism, this shall be investigated in future work.
{%
    \renewcommand{\arraystretch}{1.1}
    \begin{table}
        \centering
        \small
        \begin{tabular}{lP{3.0cm}P{2.5cm}P{2.5cm}}
            \hline \hline
            \textbf{Mechanism} & \textbf{VOC Oxidation} & \textbf{\ce{RO2 + NO}} & \textbf{Ozonolysis} \\ \hline \hline
            RADM2 & HC5 + OH & NO + TCO3 & \\ \hline
            \multirow{2}{*}{RACM2} & & & DCB + O3 \\
            & & & EPX + O3 \\ \hline
            \multirow{3}{*}{CBM-IV} & C2H4 + OH & C2O3 + NO & \\
            & CH4 + OH & & \\
            & OH + PAR & & \\ \hline
            \multirow{3}{*}{CB05} & C2H6 + OH & CXO3 + NO & \\
            & C2H4 + OH & & \\
            & OH + PAR & & \\ \hline \hline
        \end{tabular}
        \vspace{1mm}
        \caption{Non-photolysis radical producing reactions.}
        \vspace{-4mm}
        \label{t:thermal_radicals}
    \end{table}
}%
