
\subsection[Ozone Time Series and Ox Production Budgets]{Ozone Time Series and \ce{O_x} Production Budgets}

\begin{figure}
    \centering
    \includegraphics[width=0.6\textwidth]{img/O3_mixing_ratios}
    \vspace{1mm}
    \caption{Time series of \ce{O3} mixing ratios.}
    \vspace{-4mm}
    \label{f:time_series}
\end{figure}

\begin{figure}
    \centering
    \includegraphics[width=\textwidth]{img/Ox_production_budgets_by_VOC_de-allocated}
    \vspace{1mm}
    \caption{The day-time \ce{O_x} production budgets allocated to individual VOC in each mechanism.}
    \vspace{-4mm}
    \label{f:Ox_tagged_budgets}
\end{figure}

Figure \ref{f:time_series} shows the time series of \ce{O3} mixing ratios obtained using each mechanism.
The day-time production budgets of the \ce{O_x} family for each mechanism are displayed in Figure \ref{f:Ox_tagged_budgets}, the \ce{O_x} production budgets are allocated to each VOC.
The contributions from the individual VOC in Table \ref{t:initial_conditions} are re-assigned from lumped mechanism emitted species using the inverse calculation when assigning the initial conditions to the respective mechanism species.

The mechanisms including the highest amount of chemical details (MCM v3.2, MCM v3.1 and CRI v2) produce the largest amount of \ce{O3} in Figure \ref{f:time_series}.
Whilst both the CBM-IV and CB05, that include the least amount of chemical detail, produce the lowest amount of \ce{O3}.
This trend is also evident in Figure \ref{f:Ox_tagged_budgets} where the more-detailed mechanisms have largest day-time \ce{O_x} production and the less-detailed mechanisms have the lowest day-time \ce{O_x} production.

The \ce{O3} mixing ratio levels in RACM are much lower on the first day compared to mechanisms having a similar amount of chemical detail.
These low \ce{O3} amounts are due to the lack of \ce{O_x} production from aromatic VOC in RACM, which differs from \ce{O_x} production on the first day in all other mechanisms.
This lack of \ce{O_x} production from aromatic species arises from aromatic degradation chemistry which results in net \ce{O_x} consumption and is further detailed in \mbox{Section \ref{sss:aromatic}}.

Low \ce{O_x} production in CBM-IV and CB05 is related to the lower OH levels produced during the model runs using these mechanisms compared to the MCM v3.2. 
The lower OH levels arise from the high NO emissions required for maximal \ce{O_x} production conditions in CBM-IV and CB05 --- high NO levels suppresses the OH amounts through \ce{HNO3} formation \reactionref{r:NO2_OH} in which both \ce{NO2} and OH are removed from the system.
Details are found in Section \ref{ss:radicals} which compares the NO emissions calculated for maximal \ce{O_x} production in each mechanism.
Lower \ce{O3} levels using CBM-IV and CB05 compared to other mechanisms have also been noted in previous modelling studies such as \citet{Luecken:2008, Emmerson:2009} and \citet{Saylor:2012}.

\subsection{First Day Ozone Production} \label{ss:day1} %first day comparison

\begin{figure}
    \centering
    \includegraphics[width=\textwidth]{img/first_day_values}
    \vspace{1mm}
    \caption{MCM v3.2 first day TOPP values and corresponding values in each mechanism. The root mean square error (RMSE) of each set of TOPP values is also displayed.}
    \vspace{-4mm}
    \label{f:first_day}
\end{figure}

The first day TOPP values calculated from each mechanism are compared to those obtained with the MCM v3.{2} in Figure \ref{f:first_day}. 
The root mean square error (RMSE) are also illustrated in \mbox{Figure \ref{f:first_day}}.
TOPP values in MCM v3.1 and CRI v2 match those of the MCM v3.2 as indicated by their low RMSE values.
The more reduced mechanisms have a much higher spread in their first day TOPP values for each VOC and this is also confirmed by the large RMSE values.

The TOPP values for aromatic VOC tend to be lower than in MCM v3.2 in many mechanisms, this is particularly the case in RACM and  MOZART-4.
As these TOPP values for aromatic VOC vary the most from MCM v3.2, they are the root of the high RMSE values in RACM and  MOZART-4.
The lower TOPPs for aromatic VOC show that aromatic degradation chemistry is difficult to represent in chemical mechanisms as many products, their yields and reactions are not known or subject to uncertainties \citep{Vereecken:2012}.

The first day TOPP values of $2$-methylpropene in RACM, RACM2, MOZART-4 and CB05 indicate that its degradation is treated differently to the MCM v3.2. 
The variation between RACM, RACM2 and MCM v3.2 arises from differences in $2$-methylpropene ozonolysis rate constants.
In RACM and RACM2, this rate constant is calculated as a weighted mean of all VOC ozonolysis rate constants represented as OLI \citep{Stockwell:1997, Goliff:2013} and is an order of magnitude faster than MCM v3.2 despite a common source (IUPAC).
The faster rate constant results in increased radical production which leads to more \ce{O_x} production than in the MCM v3.2.

In MOZART-4, acetaldehyde is producted through the $2$-methylpropene peroxy radical reaction with NO.
Whereas there is no acetaldehyde production during $2$-methylpropene degradation in MCM v3.2.
The acetaldehyde production initiates a degradation chain involving \ce{CH3CO3} and \ce{CH3O2} that produces \ce{O_x}.

$2$-methylpropene is represented in CBM-IV and CB05 as a mixture of aldehydes (acetaldehyde and formaldehyde in CBM-IV and formaldehyde in CB05) and PAR the paraffin \ce{C-C} bond \citep{Gery:1989, Yarwood:2005}. 
The initial aldehyde oxidation immediately produces radicals which leads to \ce{O_x} production.
On the other hand PAR is a slow reacting species which slows down the \ce{O_x} production resulting in lower \ce{O_x} production than in the MCM v3.2.

\subsection{Ozone Production on Subsequent Days} \label{ss:profiles} %TOPP time series of all species

\begin{figure}
    \centering
    \includegraphics[width=\textwidth]{img/TOPP_daily_time_series_all_VOC}
    \vspace{0mm}
    \caption{TOPP value time series for all NMVOCs obtained with each mechanism.}
    \vspace{-4mm}
    \label{f:TOPP_dailies}
\end{figure}

Time series of daily TOPP values for each NMVOC are presented in \mbox{Figure \ref{f:TOPP_dailies}}. 
NMVOC such as ethene, whose degradation is described using dedicated mechanism species show similar time-dependent \ce{O_x} production in all mechanisms.
Higher variability in the time dependent \ce{O_x} production is evident for all aromatic VOC and those NMVOC represented by lumped mechanism species, such as pentane.

\ce{O_x} production from alkane degradation has a second day maximum that increases with carbon number in CRI v2 and both MCM mechanisms (Figure \ref{f:TOPP_dailies}).
This second day increase in \ce{O_x} production is reproduced by the reduced mechanisms but the magnitude of this increase is not -- this is most pronounced for hexane, heptane and octane.
Hexane, heptane and octane are represented by lumped species in all reduced mechanisms, except CRI v2 and the \ce{O_x} production during the degradation of these lumped species has a significantly lower total \ce{O_x} yield than the explicit species representing these alkanes.
The total \ce{O_x} yield of hexane, heptane and octane in each mechanism is found in the supplement to the paper.

All aromatic VOC are represented as toluene in MOZART-4, this leads to overestimating the \ce{O_x} production from less reactive VOC (benzene) and also underestimating the \ce{O_x} production from more reactive aromatic VOC (xylenes).
Aromatic VOC in RADM2 and RACM are represent as either toluene or xylene depending on the \textit{k}$_{\text{OH}}$ of the VOC.
This approach also leads to an over-estimation of the \ce{O_x} production from benzene and ethylbenzene but fares better for the \ce{O_x} production from the xylenes.
There is no \ce{O_x} production from aromatic VOC in RACM due to a change in degradation chemistry that leads to \ce{O_x} consumption, details are found in \mbox{Section \ref{sss:aromatic}}.
\ce{O_x} production from all aromatic VOC in CBM-IV and CB05 is much lower than in MCM v3.2 due to a combination of rapid loss of reactive carbon and a low \ce{O_x} Production Efficiency (OxPE, the ratio of \ce{O_x} production to \ce{O_x} loss).
Details of the reactive carbon loss and OxPE for toluene are found in Sections \ref{ss:carbon_loss} and \ref{ss:OxPE}.

\subsubsection[Ox Production during Toluene Degradation]{\ce{O_x} Production during Toluene Degradation} \label{sss:aromatic}

\begin{figure}
    \centering
    \includegraphics[width=\textwidth]{img/TOL_Ox_intermediates}
    \vspace{0mm}
    \caption{Day-time \ce{O_x} production and consumption budgets allocated to the responsible reactions from toluene degradation in all mechanisms.}
    \vspace{-4mm}
    \label{f:toluene_Ox}
\end{figure}

\ce{O_x} production during toluene degradation in Figure \ref{f:TOPP_dailies} has the largest spread between all mechanisms.
In order to explain the reasons for the large spread in \ce{O_x} production the day-time \ce{O_x} production and consumption budgets due to toluene degradation are shown in Figure \ref{f:toluene_Ox}.
The \ce{O_x} production and consumption during toluene degradation are allocated to the responsible reactions by following the tags in each mechanism.

All reduced mechanisms except CRI v2, are unable to produce similar amounts of \ce{O_x} as the MCMv3.2.
The lower amounts of \ce{O_x} produced in reduced mechanisms is constrained by their rapid loss of reactive carbon during degradation of toluene, this analysis is described in Section \ref{ss:carbon_loss}.
Reduced mechanisms have a similar or lower \ce{O_x} production efficiency to the MCM v3.2 as highlighted in Section \ref{ss:OxPE}, in the case of toluene degradation this lower OxPE does not compensate enough for the loss of reactive carbon leading to lower \ce{O_x} production.

The \ce{O_x} production in CRI v2 and RACM2 reaches its maximum on the second day which differs from the MCM v3.2 which produces peak \ce{O_x} on the first day.
The second day maximum of \ce{O_x} production in CRI v2 and RACM2 results from additional \ce{C2H5O2} (C2H5O2 in CRI v2 and ETHP in RACM2) production that is not found in MCM v3.2.
The extra \ce{C2H5O2} production results from differing treatments to the MCM v3.2 in the degradation of unsaturated dicarbonyls producing during toluene degradation.

In CRI v2, unsaturated dicarbonyls from toluene degradation produce propionaldehyde (\ce{C2H5CHO}) leading to \ce{C2H5O2}.
Another degradation pathway for unsaturated dicarbonyls in CRI v2 produces the HOCH2CH2O2 peroxy radical which is also an additional \ce{O_x} source unique to CRI v2 in Figure \ref{f:toluene_Ox}.

Unsaturated dicarbonyls in RACM2 also lead to production of \ce{C2H5CHO} but this is reached via different pathways to the CRI v2.
\ce{C2H5CHO} is produced through the degradation of higher peroxides (OP2) and propyl peroxy radical (HC3P) which are degradation products of unsaturated dicarbonyl, in addition to \ce{C2H5CHO} production directly from unsaturated dicarbonyl degradation.

RACM chemistry results in net \ce{O_x} loss on the first two days in contrast to net \ce{O_x} production in the MCM v3.2 due to several \ce{O_x}-consuming reactions in RACM that are not present in the MCM.
Ozonolysis of the cresol OH-adduct mechanism species ADDC contributes the most to \ce{O_x} loss in RACM.
This reaction was included in RACM due to improved cresol product yields when comparing RACM predictions with experimental data \citep{Stockwell:1997}, other mechanisms that include cresol OH-adduct species do not include ozonolysis.
Including ozonolysis of aromatic OH-adduct species in RACM results in non-representative \ce{O_x} production, these ozonolysis reactions are not included in the updated RACM2.

\subsection{Production of Radicals} \label{ss:radicals}

\begin{figure}
    \centering
    \includegraphics[width=\textwidth]{img/radical_NOx_production_budgets}
    \vspace{0mm}
    \caption{Net radical production day-time budgets for each mechanism. Net radical production calculated as the difference between radical and \ce{NO_x} yields. O1D represents the reaction of \ce{O(^1D)} with water vapour.}
    \vspace{-4mm}
    \label{f:radical_production} 
\end{figure} 

Production of radicals impacts \ce{O_x} production by the conversion of NO to \ce{NO2} by peroxy radicals produced from VOC degradation that is typically initiated through OH reaction.
In this study, the amount of NO emissions are also affected by radical production as described in Section \ref{ss:model_setup} conditions of maximal \ce{O_x} production are achieved for each mechanism by emitting the amount of NO required to balance the source of radicals at each time step. 

In order to investigate the differences in NO emissions required by each mechanism to achieve maximal \ce{O_x} production, the net radical to \ce{NO_x} production budgets allocated to the contributing reactions are presented in Figure \ref{f:radical_production}.
The net radical to \ce{NO_x} production budgets were determined using the same calculation that was used to calculate the amount of NO emitted at each time step and hence the net radical production budgets in Figure \ref{f:radical_production} directly correlate to the amount of NO emitted during each mechanism's model run.

CBM-IV and CB05 were developed for the high-\ce{NO_x} conditions of urban and polluted regions \citep{Gery:1989, Yarwood:2005} and both mechanisms represent VOC-sensitive chemistry.
The representation of VOC-sensitive chemistry in CBM-IV and CB05 leads to much higher net radical production on the first two days corresponding to larger NO emissions than any other mechanism.
The higher NO emissions in CBM-IV and CB05 limit the amount of OH available for initiating VOC degradation as larger amount of \ce{HNO3} are produced through \reactionref{r:NO2_OH}.
\ce{HNO3} production also removes \ce{NO_x} from the system which constrains the amount of NO available for reaction with \ce{RO2} that would produce \ce{O_x}.

Photolysis of carbonyl species, mainly HCHO, is the main source of radicals from organic chemistry.
Reduced mechanisms do not include as many carbonyl species as more explicit mechanisms --- CBM-IV with 2 and RACM2 with 17 carbonyl species represent the lower and upper limits the number of carbonyl species in reduced mechanisms (besides the CRI v2).
The lack of carbonyl species in non-CRI v2 reduced mechanisms means that these reduced mechanisms produce radicals through other reactions such as initial VOC oxidation, NO--\ce{RO2} reactions and ozonolysis.
Examples of radical sources through non-photolysis pathways in reduced mechanisms are given in Table \ref{t:thermal_radicals}.

The mechanisms compared here require different NO emissions to simulate \ce{NO_x}-VOC-sensitive conditions, where the source of radicals is balanced by the source of \ce{NO_x}.
Thus, a constant NO source may lead to different atmospheric regimes (\ce{NO_x}- or VOC-sensitive) being simulated depending on the mechanism, this shall be investigated in future work.
{%
    \renewcommand{\arraystretch}{1.1}
    \begin{table}
        \centering
        \small
        \begin{tabular}{lP{3.0cm}P{2.5cm}P{2.5cm}}
            \hline \hline
            \textbf{Mechanism} & \textbf{VOC Oxidation} & \textbf{\ce{RO2 + NO}} & \textbf{Ozonolysis} \\ \hline \hline
            RADM2 & HC5 + OH & NO + TCO3 & \\ \hline
            \multirow{2}{*}{RACM2} & & & DCB + O3 \\
            & & & EPX + O3 \\ \hline
            \multirow{3}{*}{CBM-IV} & C2H4 + OH & C2O3 + NO & \\
            & CH4 + OH & & \\
            & OH + PAR & & \\ \hline
            \multirow{3}{*}{CB05} & C2H6 + OH & CXO3 + NO & \\
            & C2H4 + OH & & \\
            & OH + PAR & & \\ \hline \hline
        \end{tabular}
        \vspace{1mm}
        \caption{Non-photolysis radical producing reactions.}
        \vspace{-4mm}
        \label{t:thermal_radicals}
    \end{table}
}%
