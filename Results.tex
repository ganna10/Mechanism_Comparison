%
\subsection[O3 Time Series and Ox Production Budgets]{Ozone Time Series and \ce{O_x} Production Budgets} \label{ss:O3_time_series}

\begin{figure}
    \centering
    \includegraphics[width=0.8\textwidth]{img/O3_mixing_ratios}
    \vspace{1mm}
    \caption{Time series of \ce{O3} and OH mixing ratios obtained using each mechanism.}
    \vspace{-4mm}
    \label{f:time_series}
\end{figure}

\begin{figure}
    \centering
    \includegraphics[width=\textwidth]{img/Ox_production_budgets_by_VOC_de-allocated}
    \vspace{1mm}
    \caption{Day-time \ce{O_x} production budgets in each mechanism allocated to individual VOC.}
    \vspace{-4mm}
    \label{f:Ox_tagged_budgets}
\end{figure}

Figure \ref{f:time_series} shows the time series of \ce{O3} and OH mixing ratios obtained with each mechanism.
The mechanisms with the most detailed chemistry (MCM v3.2, MCM v3.1 and CRI v2) produce the highest \ce{O3} mixing ratios (Figure \ref{f:time_series}).
Whilst CBM-IV and CB05, with the least chemical detail, produce the lowest \ce{O3} mixing ratios.

The day-time \ce{O_x} production budgets allocated to VOC for each mechanism are displayed in \mbox{Figure \ref{f:Ox_tagged_budgets}}.
The \ce{O_x} production from lumped mechanism species are re-assigned to the VOC of \mbox{Table \ref{t:initial_conditions}} by scaling the \ce{O_x} production of the lumped species by the fractional contribution of each represented VOC.
For example, TOL in RACM2 represents toluene and ethylbenzene with fractional contributions of $0.85$ and $0.15$.
Scaling the \ce{O_x} production from TOL by these factors gives the \ce{O_x} production from toluene and ethylbenzene in RACM2.
The trends in \ce{O3} mixing ratios are mirrored in Figure \ref{f:Ox_tagged_budgets} where mechanisms producing high amounts of \ce{O_x} have high \ce{O3} mixing ratios.
The first day \ce{O3} mixing ratios in RADM2 are slightly less than in MCM v3.2 despite larger \ce{O_x} production due to higher \ce{NO2} mixing ratios in RADM2 caused by larger NO emissions in RADM2 than MCM v3.2.

The first day mixing ratios of \ce{O3} in RACM are lower than mechanisms having similarly detailed chemistry (such as MOZART-4 or RADM2).
The lower \ce{O3} mixing ratios are due to a lack of \ce{O_x} production from aromatic VOC on the first day in RACM (Figure \ref{f:Ox_tagged_budgets}).
Aromatic degradation chemistry in RACM results in net loss of \ce{O_x} on the first day (\mbox{Section \ref{ss:profiles}}).

The low OH mixing ratios obtained with CBM-IV and CB05 (Figure \ref{f:time_series}) inhibit VOC oxidation leading to low \ce{O3} mixing ratios.
CBM-IV and CB05 represent the high-\ce{NO_x} chemistry of polluted urban areas leading to much larger NO emissions for conditions of maximal \ce{O3} production than any other mechanism.
These large NO emissions lead to higher \ce{HNO3} production through \reactionref{r:NO2_OH} and an increased sink for OH and \ce{NO_x} due to higher \ce{HNO3} deposition rates in CBM-IV and CB05.
Thus less OH is available for VOC oxidation leading to low \ce{O3} mixing ratios.
Lower \ce{O3} levels using CBM-IV and CB05 compared to other mechanisms have also been noted in previous
mechanism comparison studies such as \citet{Emmerson:2009} and \citet{Saylor:2012}.

\subsection{First Day Ozone Production} \label{ss:day1} %first day comparison

\begin{figure}
    \centering
    \includegraphics[width=\textwidth]{img/first_day_values}
    \vspace{1mm}
    \caption{The first day TOPP values for each VOC calculated using MCM v3.2 and the corresponding values in each mechanism. The root mean square error (RMSE) of each set of TOPP values is also displayed.}
    \vspace{-4mm}
    \label{f:first_day}
\end{figure}

The first day TOPP values of each VOC, representing ozone production due to freshly emitted VOC near their source
region, from each mechanism are compared to those obtained with the MCM v3.{2} in Figure \ref{f:first_day}.
The root mean square error (RMSE) of all first day TOPP values in each mechanism relative to
those in the MCM v3.2 are also included.
The RMSE values show that lumped structure and lumped molecule mechanisms have a higher spread in their first day \ce{O_x} production than the CRI v2, which lumps intermediate species.
CBM-IV and CB05 produce low amounts of \ce{O_x} (Figure \ref{f:Ox_tagged_budgets}) resulting in low TOPP values for most VOC
(Figure \ref{f:first_day}).
Low OH levels lead to lower \ce{O_x} production in the CBM-IV and CB05 (Section \ref{ss:O3_time_series}).

Largest differences in TOPP values between the reduced mechanisms and the \mbox{MCM v3.2} are observed for aromatic species.
For example, all aromatic VOC in MOZART-4 are represented as toluene, thus less reactive aromatic VOC, such as benzene, produce higher \ce{O_x} whilst more reactive aromatic VOC, such as the xylenes, produce less \ce{O_x} in MOZART-4 than in \mbox{MCM v3.2}.
RACM2 includes lumped species for benzene, toluene and each xylene resulting in \ce{O_x} production from aromatic VOC that is the most similar to the MCM v3.2 than other lumped molecule mechanisms.

The first day TOPP values of $2$-methylpropene in RACM, RACM2, MOZART-4 and CB05 signify differences in its degradation to the MCM v3.2.
The variation between RACM, RACM2 and MCM v3.2 arises from differences in the ozonolysis rate constant of $2$-methylpropene.
This rate constant is an order of magnitude faster in RACM and RACM2 than in MCM v3.2 as the RACM, RACM2 rate constant is a weighted mean of the ozonolysis rate constants of each VOC represented as OLI \citep{Stockwell:1997, Goliff:2013}.
The faster rate constant promotes increased radical production leading to more \ce{O_x} in RACM and RACM2 than the MCM v3.2.

The degradation of $2$-methylpropene in MOZART-4 produces acetaldehyde (\ce{CH3CHO}) through the reaction between NO and the $2$-methylpropene peroxy radical, whereas no \ce{CH3CHO} is produced during $2$-methylpropene degradation in the MCM v3.2.
\ce{CH3CHO} \mbox{initiates a} degradation chain producing \ce{O_x} involving \ce{CH3CO3} and \ce{CH3O2} leading to more \ce{O_x} in MOZART-4 than \mbox{MCM v3.2}.

The $2$-methylpropene representation in CB05 was updated to \mbox{HCHO + $3$ PAR} from \mbox{\ce{CH3CHO} + HCHO + PAR} in CBM-IV, where PAR is the \ce{C-C} bond \citep{Gery:1989, Yarwood:2005}.
The representation in CB05 includes more of the slower reacting PAR at the expense of the more reactive \ce{CH3CHO} producing less \ce{O_x} than CBM-IV.

\subsection{Ozone Production on Subsequent Days} \label{ss:profiles} %TOPP time series of all species

\begin{figure}
    \centering
    \includegraphics[width=\textwidth]{img/TOPP_daily_time_series_all_VOC}
    \vspace{0mm}
    \caption{TOPP value time series using each mechanism for each VOC.}
    \vspace{-4mm}
    \label{f:TOPP_dailies}
\end{figure}

Time series of daily TOPP values for each VOC are presented in \mbox{Figure \ref{f:TOPP_dailies}}. 
Higher variability in the time dependent \ce{O_x} production is evident for VOC represented by lumped mechanism species.
The lowest \ce{O_x} is produced from lumped structure mechanisms (CBM-IV and CB05) whilst the lumped intermediate mechanism (CRI v2) produces the most similar \ce{O_x} to the \mbox{MCM v3.2}.
In general, aromatic and unsaturated species have their largest differences between mechamisms on the first
day of the simulations, while alkanes show their largest inter-mechanism differences in \ce{O_x} production
on the second and third days of the simulations.

Alkane degradation in CRI v2 and both MCM mechanisms produces a second day maximum in \ce{O_x} that increases with alkane carbon number (Figure \ref{f:TOPP_dailies}).
The increase in \ce{O_x} production on the second day is reproduced by the reduced mechanisms but not the magnitude.
This difference is most pronounced for the larger alkanes in this study.

Many alkanes are not explicitly represented in reduced mechanisms, so the \ce{O_x} production of the individual alkanes is determined by scaling the \ce{O_x} production of the mechanism species by the alkane carbon number (Section \ref{sss:TOPP}).
This approach leads to a lower increase of \ce{O_x} with alkane carbon number in lumped molecule and lumped structure mechanisms as the \ce{O_x} production per carbon number is lower than the explicit representations in MCM v3.2, MCMv3.1 and CRI v2 (Figure \ref{f:alkane_C}).

Propane is represented as HC3 in RADM2, RACM and RACM2.
This treatment of propane degradation leads to higher \ce{O_x} production than the MCM v3.2
as HC3 has a higher glyoxal yield compared to alkane degradation in MCM v3.2 since HC3 also represents alcohols, esters and alkynes \citep{Stockwell:1990, Stockwell:1997, Goliff:2013}.
Higher glyoxal levels produce more radicals and hence \ce{O_x} leading to dissimilar \ce{O_x} production for propane between MCM v3.2 and RACM.

\ce{O_x} production from branched alkanes is larger in reduced mechanisms than the \mbox{MCM v3.2} as the degradation of branched and straight-chain alkanes is not differentiated in reduced mechanisms.
This approach does not account for the lower reaction rate constant of branched alkanes with OH.

\begin{figure}
    \centering
    \includegraphics[width=\textwidth]{img/Alkanes_vs_C_sums}
    \vspace{0mm}
    \caption{Cumulative TOPP values per carbon number of each alkane in the mechanisms.}
    \vspace{-4mm}
    \label{f:alkane_C}
\end{figure}

\ce{O_x} production during toluene degradation has a high spread in \mbox{Figure \ref{f:TOPP_dailies}} indicating that aromatic degradation is treated differently between mechanisms.
The reactions contributing to \ce{O_x} production and loss during toluene degradation are determined by following the ``toluene'' tags in each mechanism and compared in Figure \ref{f:toluene_Ox}.

\begin{figure}
    \centering
    \includegraphics[height=0.95\textheight]{img/TOL_Ox_intermediates}
    \vspace{0mm}
    \caption{Day-time \ce{O_x} production and loss budgets allocated to the responsible reactions during toluene degradation in all mechanisms.}
    \vspace{-4mm}
    \label{f:toluene_Ox}
\end{figure}

Toluene degradation in RACM includes several reactions consuming \ce{O_x} that are not present in the MCM resulting in net \ce{O_x} loss on the first two days.
Ozonolysis of the cresol OH-adduct mechanism species ADDC contributes significantly to \ce{O_x} loss in RACM.
This reaction was included in RACM due to improved cresol product yields when comparing RACM predictions with experimental data \citep{Stockwell:1997}. 
Other mechanisms that include cresol OH-adduct species do not include ozonolysis.
Including ozonolysis of aromatic OH-adduct species in RACM results in non-representative \ce{O_x} production, these ozonolysis reactions are not included in the updated RACM2.

Maximum \ce{O_x} production in CRI v2 and RACM2 is reached on the second day unlike the MCM v3.2 which produces peak \ce{O_x} on the first day.
The second day maximum of \ce{O_x} production in CRI v2 and RACM2 results from different treatments of unsaturated dicarbonyls during toluene degradation to MCM v3.2.

In CRI v2, unsaturated dicarbonyls from toluene degradation lead to either $\beta$-hydroyalkyl peroxy radical (\ce{HOCH2CH2O2}) or \ce{C2H5O2} production.
Both \ce{C2H5O2} and \ce{HOCH2CH2O2} contribute significantly to \ce{O_x} production on the second day and are not produced during toluene degradation in MCM v3.2.

Unsaturated dicarbonyls in RACM2 also produce \ce{C2H5O2} but this is reached via different pathways to the CRI v2.
\ce{C2H5O2} is produced from propionaldehyde (\ce{C2H5CHO}) degradation which is a produced directly from unsaturated dicarbonyl degradation but also from the degradation of other unsaturated dicarbonyl products, such as higher peroxides (OP2) and propyl peroxy radical (HC3P).

All reduced mechanisms are unable to produce similar amounts of \ce{O_x} on the first day as the MCM v3.2 due to a faster break down of VOC in reduced mechanisms, detailed in \mbox{Section \ref{ss:products}}.
The faster break down of VOC coupled with different degradation chemistry in reduced mechanisms compared to the MCM v3.2 lead to the different time dependent \ce{O_x} production profiles in Figure \ref{f:TOPP_dailies}.


\subsection{Treatment of Degradation Products} \label{ss:products} 

The time dependent \ce{O_x} production of different VOC in Figure \ref{f:TOPP_dailies} results from the varying rates at which VOC break up into smaller fragments \citep{Butler:2011}.
Varying break down rates of VOC between mechanisms could explain the different time dependent \ce{O_x} production between mechanisms.
The break down of pentane and toluene between mechanisms is compared in \mbox{Figure \ref{f:carbon}} by allocating the \ce{O_x} production to the different sizes of degradation products.

\begin{figure}
    \centering
    \includegraphics[width=1.10\textwidth]{img/Ox_production_by_C_number}
    \vspace{0mm}
    \caption{Day-time \ce{O_x} production during pentane and toluene degradation is attributed to the number of carbons of the degradation products for each mechanism.}
    \vspace{-4mm}
    \label{f:carbon}
\end{figure}

Some mechanism species in RADM2, RACM and RACM2 have fractional carbon numbers \citep{Stockwell:1990, Stockwell:1997, Goliff:2013} and \ce{O_x} production from these species was reassigned as \ce{O_x} production of the nearest integral carbon number.  
Many reduced mechanisms use an operator species as a surrogate for \ce{RO2} during VOC degradation enabling these mechanisms to produce \ce{O_x} whilst minimising the number of \ce{RO2} species.
\ce{O_x} production from operator species is assigned as \ce{O_x} production from the organic degradation species producing the operator.
This allocation technique is also used to assign \ce{O_x} production from \ce{HO2} via \reactionref{r:HO2_NO}.

During pentane degradation, more \ce{O_x} is produced from degradation products having the same carbon number as pentane in explicit mechanisms throughout the model run (Figure \ref{f:carbon}).
Reduced mechanisms produce similar amounts of \ce{O_x} on the first day to the MCM v3.2 by producing more \ce{O_x} from smaller degradation products.
This increased influence of smaller degradation products indicates that pentane is broken down faster in reduced mechanisms.

\begin{figure}
    \centering
    \includegraphics[width=0.8\textwidth]{img/net_reactive_carbon_loss_pentane_toluene}
    \vspace{0mm}
    \caption{Daily net carbon loss rate during \ce{O_x} production from pentane and toluene degradation.}
    \vspace{-4mm}
    \label{f:net_carbon_loss}
\end{figure}

Pentane is broken down faster in RADM2, RACM and RACM2 by losing reactive carbon quicker than the MCM v3.2 (Figure \ref{f:net_carbon_loss}).
Whilst in MOZART-4, pentane is broken down into smaller fragments quicker than the MCM v3.2 as each product of the pentyl peroxy radical reaction with NO has less than five carbons.
The quicker break down of pentane limits the amount of reactive carbon available to produce further \ce{O_x} leading to lower \ce{O_x} production in reduced mechanisms.

CBM-IV and CB05 represent pentane degradation chemistry by that of PAR, the \ce{C-C} bond, which involves only C1 and C2 species.
Thus pentane degradation in CBM-IV and CB05 has a low rate of reactive carbon loss in Figure \ref{f:net_carbon_loss}.

\ce{O_x} produced during toluene degradation has a high spread between all mechanisms \mbox{(Figure \ref{f:TOPP_dailies})} and Figure \ref{f:carbon} shows differing distributions of the degradation products sizes producing \ce{O_x}.
All reduced mechanisms omit \ce{O_x} production from at least one degradation fragment size which produces \ce{O_x} in the MCM v3.2 indicating that toluene is also broken down quicker than more explicit mechanisms.
For example, toluene degradation in RACM2 does not produce \ce{O_x} from degradation products with six carbons unlike the \mbox{MCM v3.2}.

All reduced mechanisms lose reactive carbon faster than the MCM v3.2 during toluene degradation (Figure \ref{f:net_carbon_loss}).
Thus reduced mechanisms are unable to reach the \ce{O_x} production levels of explicit mechanisms.
