%
\subsection[O3 Time Series and Ox Production Budgets]{Ozone Time Series and \ce{O_x} Production Budgets} \label{ss:O3_time_series}
%
\begin{figure}
    \centering
    \includegraphics[width=\textwidth]{img/O3_mixing_ratios}
    \vspace{0mm}
    \caption{Time series of \ce{O3} mixing ratios obtained using each mechanism.}
    \vspace{-4mm}
    \label{f:time_series}
\end{figure}
%
\begin{figure}
    \centering
    \includegraphics[width=\textwidth]{img/Ox_production_budgets_by_VOC_de-allocated}
    \vspace{1mm}
    \caption{Day-time \ce{O_x} production budgets in each mechanism allocated to individual VOC.}
    \vspace{-4mm}
    \label{f:Ox_tagged_budgets}
\end{figure}
%
Figure \ref{f:time_series} shows the time series of \ce{O3} mixing ratios obtained with each mechanism.
There is an \mbox{$8$ ppbv} difference in \ce{O3} mixing ratios on the first day between RADM2, highest \ce{O3}, and RACM2, which has the lowest \ce{O3} mixing ratios when not considering the outlier time series of RACM.
The \ce{O3} mixing ratios in the CRI v2 are higher than those in the MCM v3.1, mirroring the results of \citet{Jenkin:2008}.

The day-time \ce{O_x} production budgets allocated to VOC for each mechanism are shown in \mbox{Figure \ref{f:Ox_tagged_budgets}}.
The trends in \ce{O3} mixing ratios are mirrored in Figure \ref{f:Ox_tagged_budgets} where mechanisms producing high amounts of \ce{O_x} have high \ce{O3} mixing ratios in Figure \ref{f:time_series}.

The first day mixing ratios of \ce{O3} in RACM are lower than other mechanisms due to a lack of \ce{O_x} production from aromatic VOC on the first day in RACM (Figure \ref{f:Ox_tagged_budgets}).
Aromatic degradation chemistry in RACM results in net loss of \ce{O_x} on the first day, described later in \mbox{Section \ref{sss:day1}}.

RADM2 is the only reduced mechanism producing higher \ce{O3} mixing ratios than more detailed mechanisms (MCM v3.2, MCM v3.1 and CRI v2).
Higher mixing ratios of \ce{O3} in RADM2 are produced due to increased \ce{O_x} production from propane, butane and $2$-methylpropane compared to the \mbox{MCM v3.2} (Figure \ref{f:Ox_tagged_budgets}).
These alkanes are each represented as HC3 in RADM2 \citep{Stockwell:1990}.
HC3 degradation produces more acetaldehyde (\ce{CH3CHO}) than the MCM v3.2 which leads to more \ce{O_x} production in RADM2 through reactions of \ce{CH3CO3} and \ce{CH3O2}, degradation products of \ce{CH3CHO}, with NO. 
%
\subsection[Time Dependent Ox Production]{Time Dependent \ce{O_x} Production}
%
\begin{figure}
    \centering
    \includegraphics[width=\textwidth]{img/TOPP_daily_time_series_all_VOC}
    \vspace{0mm}
    \caption{TOPP value time series using each mechanism for each VOC.}
    \vspace{-4mm}
    \label{f:TOPP_dailies}
\end{figure}
%
Time series of daily TOPP values for each VOC are presented in \mbox{Figure \ref{f:TOPP_dailies}}. 
The lumped intermediate mechanism (CRI v2) produces the most similar \ce{O_x} to the \mbox{MCM v3.2} for each VOC.
Higher variability in the time dependent \ce{O_x} production is evident for VOC represented by lumped mechanism species.
For example, $2$-methylpropene, represented in reduced mechanisms by lumped species, has a higher spread in time dependent \ce{O_x} production than ethene, which is represented explicitly in each mechanism.

In general, aromatic and unsaturated species have their largest differences between mechanisms on the first day of the simulations, while alkanes show their largest inter-mechanism differences in \ce{O_x} production on the second and third days of the simulations.
\mbox{Section \ref{sss:day1}} examines differences in first day \ce{O_x} production between the chemical mechanisms and \mbox{Section \ref{sss:profiles}} examines the differences in \ce{O_x} production on subsequent days.
%
\subsubsection{First Day Ozone Production} \label{sss:day1} %first day comparison
%
\begin{figure}
    \centering
    \includegraphics[width=\textwidth]{img/first_day_values}
    \vspace{1mm}
    \caption{The first day TOPP values for each VOC calculated using MCM v3.2 and the corresponding values in each mechanism. The root mean square error (RMSE) of each set of TOPP values is also displayed. The black line represents the $1:1$ line.}
    \vspace{-4mm}
    \label{f:first_day}
\end{figure}
%
The first day TOPP values of each VOC, representing \ce{O3} production from freshly emitted VOC near their source region, from each mechanism are compared to those obtained with the \mbox{MCM v3.{2}} in Figure \ref{f:first_day}.
The root mean square error (RMSE) of all first day TOPP values in each mechanism relative to those in the MCM v3.2 are also included in Figure \ref{f:first_day}.  
The RMSE values show that lumped structure and molecule mechanisms have a higher spread in their first day \ce{O_x} production than the CRI v2, which lumps intermediate species.

Aromatic VOC in reduced mechanisms are represented by mechanism species describing the degradation of many other aromatic VOC.
For example, all aromatic VOC in MOZART-4 are represented as toluene, thus less reactive aromatic VOC, such as benzene, produce higher \ce{O_x} whilst more reactive aromatic VOC, such as the xylenes, produce less \ce{O_x} in MOZART-4 than the \mbox{MCM v3.2}.
RACM2 includes explicit species representing benzene, toluene and each xylene resulting in \ce{O_x} production from aromatic VOC that is the most similar to the MCM v3.2 than other reduced mechanisms.

\ce{O_x} production during toluene degradation has a high spread on the first day in \mbox{Figure \ref{f:TOPP_dailies}} indicating that aromatic degradation is treated differently between mechanisms.
The reactions contributing to \ce{O_x} production and loss during toluene degradation are determined by following the ``toluene'' tags in each mechanism and are compared in Figure \ref{f:toluene_Ox}.

%
\begin{figure}
    \centering
    \includegraphics[height=0.95\textheight]{img/TOL_Ox_intermediates}
    \vspace{0mm}
    \caption{Day-time \ce{O_x} production and loss budgets allocated to the responsible reactions during toluene degradation in all mechanisms.}
    \vspace{-4mm}
    \label{f:toluene_Ox}
\end{figure}
%
Toluene degradation in RACM includes several reactions consuming \ce{O_x} that are not present in the MCM resulting in net loss of \ce{O_x} on the first two days.
Ozonolysis of the cresol OH-adduct mechanism species ADDC contributes significantly to \ce{O_x} loss in RACM.
This reaction was included in RACM due to improved cresol product yields when comparing RACM predictions with experimental data \citep{Stockwell:1997}. 
Other mechanisms that include cresol OH-adduct species do not include ozonolysis.
Ozonolysis of aromatic OH-adduct species in RACM results in non-representative \ce{O_x} production, these reactions are not included in the updated RACM2.

The total \ce{O_x} produced on the first day during toluene degradation in all reduced mechanisms is less than that in the \mbox{MCM v3.2} (Figure \ref{f:toluene_Ox}).
Less \ce{O_x} is produced in all reduced mechanisms due to a faster break down of the VOC into smaller fragments than the MCM, described later in \mbox{Section \ref{ss:products}}.
Moreover in CBM-IV and CB05, less \ce{O_x} is produced during toluene degradation as reactions of the toluene degradation products \ce{CH3O2} and CO contribute less to the \ce{O_x} production budgets than in any other mechanism (Figure \ref{f:toluene_Ox}).

Maximum \ce{O_x} production from toluene degradation in CRI v2 and RACM2 is reached on the second day unlike the MCM v3.2 which produces peak \ce{O_x} on the first day.
The second day maximum of \ce{O_x} production in CRI v2 and RACM2 from toluene degradation results from increased \ce{C2H5O2} production from degradation of unsaturated dicarbonyls; \ce{C2H5O2} is not produced during degradation of unsaturated dicarbonyls in the MCM v3.2.

Unsaturated VOC generally produce similar amounts of \ce{O_x} between mechanisms, especially VOC represented explicitly, such as ethene and isoprene.
\ce{O_x} production from unsaturated VOC that are not represented explicitly, such as $2$-methylpropene, show greater variation \mbox{(Figure \ref{f:TOPP_dailies}).}
The variation in \ce{O_x} production from $2$-methylpropene degradation in reduced mechanisms arises from differing rate constants of initial oxidation reactions and unrepresentative secondary chemistry, further details are found in the online supplement to this paper.

Non-explicit representations of aromatic and unsaturated VOC coupled with differing degradation chemistry and a faster break down into smaller size degradation products results in different \ce{O_x} production in lumped molecule and lumped structure mechanisms compared to the MCM v3.2.
%
\subsubsection{Ozone Production on Subsequent Days} \label{sss:profiles} %TOPP time series of all species
%
Alkane degradation in CRI v2 and both MCM mechanisms produces a second day maximum in \ce{O_x} that increases with alkane carbon number (Figure \ref{f:TOPP_dailies}).
The increase in \ce{O_x} production on the second day is reproduced for each alkane by the reduced mechanisms; except octane in RADM2, RACM and RACM2.
However, larger alkanes produce lower \ce{O_x} on the second day in all lumped molecule and structure mechanisms.

The lumped molecule mechanisms (MOZART-4, RADM2, RACM and RACM2) represent many alkanes by mechanism species which may lead to unrepresentative secondary chemistry.
For example, more \ce{O_x} is produced during the degradation of propane, butane and $2$-methylpropane in RADM2 than the MCM v3.2 (Figure \ref{f:Ox_tagged_budgets}).
These alkanes are all represented in RADM2 by the mechanism species HC3 which also represents other classes of VOC, such as alcohols.
The secondary chemistry of HC3 is tailored to produce \ce{O_x} from these different VOC and differs from the alkane degradation in the MCM v3.2 by producing more \ce{CH3CHO} in RADM2.

%
\begin{figure}
    \centering
    \begin{subfigure}[t]{0.4\textwidth}
        \includegraphics[width=\textwidth]{img/HC5P_NO_C_numbers}
        \caption{The total reactive carbon of the products of the reaction between NO and the pentyl peroxy radical in lumped molecule mechanisms compared to the MCM. The black dot represents the reactive carbon of the pentyl peroxy radical.}
        \label{f:HC5P_NO}
    \end{subfigure}
    \hspace{1cm}
    \begin{subfigure}[t]{0.4\textwidth}
        \includegraphics[width=\textwidth]{img/alkane_unreacted_after_day1}
        \caption{The percentage of unreacted alkane at the beginning of the second day of simulations in the lumped structure mechanisms compared to the MCM.}
        \label{f:Unreacted_PAR}
    \end{subfigure}
    \vspace{3mm}
    \caption{Treatment of alkane degradation in lumped molecule and structure mechanisms.}
    \vspace{-4mm}
    \label{f:alkanes}
\end{figure}
%
As will be shown in Section \ref{ss:products}, another feature of lumped molecule mechanisms is that the breakdown of emitted VOC into smaller sized degradation products is faster than more explicit mechanisms.
Alkanes are broken down quicker in RADM2, RACM and RACM2 through a higher rate of reactive carbon loss than the MCM v3.2 (shown for pentane in \mbox{Figure \ref{f:net_carbon_loss}}); reactive carbon is lost through reactions not conserving carbon.
Despite many alkane degradation reactions in MOZART-4 almost conserving carbon, the organic products have less carbon atoms than the organic reactant speeding up the breakdown of the alkane compared to the MCM v3.2.
As an example, Figure \ref{f:HC5P_NO} shows the total reactive carbon of the reactants and products from the reaction of NO with the pentyl peroxy radical in both MCM mechanisms and each lumped molecule mechanism.
In the lumped molecule mechanisms, the individual organic products have less reactive carbon than the organic reactant; whilst, in RADM2, RACM and RACM2 this reaction does not conserve reactive carbon.  

A further example of the speed of alkane breakdown impacting the amount of \ce{O_x} produced on the second day is octane in RADM2, RACM and RACM2 which produces peak \ce{O_x} on the first day, in contrast to all other mechanisms where peak \ce{O_x} is produced on the second day.
The break down of octane in RADM2, RACM and RACM2 on the first day proceeds so quickly that there are not enough degradation products available on the second day to produce peak \ce{O_x} on the second day.
The rate of carbon loss during octane degradation is shown in the online supplement to this paper.

Alkane degradation in lumped structure mechanisms (CBM-IV and CB05) also loses reactive carbon quicker than the MCM, shown in Figure \ref{f:net_carbon_loss}, but still produce peak \ce{O_x} on the second day.
Alkanes are represented by PAR, the \ce{C-C} bond, with the amount of PAR emitted determined by multiplying the initial condition of the alkane by its carbon number.
This approach leads to a larger amount of unreacted PAR after the first day which reacts with OH during the subsequent days of the simulation producing peak \ce{O_x} on the second day despite the loss of reactive carbon.
\mbox{Figure \ref{f:Unreacted_PAR}} shows the percentage of unreacted alkane at the beginning of the second day in the CBM-IV, CB05 and the MCM mechanisms.
The lumped structure approach to alkane degradation produces similar amounts of \ce{O_x} on the first day to the MCM v3.2 (\mbox{Figure \ref{f:first_day}}) but does not produce similar amounts of \ce{O_x} on the second day to the MCM v3.2.

The faster breakdown of alkanes in lumped molecule and structure mechanisms on the first day limits the amount of \ce{O_x} produced on the second day, as less of the larger sized degradation products are available for further degradation and \ce{O_x} production.  
%
\subsection{Treatment of Degradation Products} \label{ss:products} 
%
\begin{figure}
    \centering
    \includegraphics[width=1.10\textwidth]{img/Ox_production_by_C_number}
    \vspace{0mm}
    \caption{Day-time \ce{O_x} production during pentane and toluene degradation is attributed to the number of carbons of the degradation products for each mechanism.}
    \vspace{-4mm}
    \label{f:carbon}
\end{figure}
%
The time dependent \ce{O_x} production of the different VOC in Figure \ref{f:TOPP_dailies} results from the varying rates at which VOC break up into smaller fragments \citep{Butler:2011}.
Varying break down rates of the same VOC between mechanisms could explain the different time dependent \ce{O_x} production between mechanisms.
The break down of pentane and toluene between mechanisms is compared in \mbox{Figure \ref{f:carbon}} by allocating the \ce{O_x} production to the number of carbon atoms in the degradation products.
Some mechanism species in RADM2, RACM and RACM2 have fractional carbon numbers \citep{Stockwell:1990, Stockwell:1997, Goliff:2013} and \ce{O_x} production from these species was reassigned as \ce{O_x} production of the nearest integral carbon number.  

During degradation of pentane, a five-carbon VOC, up to \mbox{$50$ \%} more \ce{O_x} is produced from degradation products also having five carbon atoms on the first day using the MCM v3.2. 
Moreover the contribution of the degradation products having five carbon atoms in the MCM v3.2 is consistently higher throughout the model run than in reduced mechanisms \mbox{(Figure \ref{f:carbon}).}
Reduced mechanisms produce similar amounts of \ce{O_x} on the first day to the \mbox{MCM v3.2} by producing up to double the amount of \ce{O_x} from smaller degradation products than in the MCM v3.2.
The lower contribution of larger degradation products indicates that pentane is broken down faster in reduced mechanisms.

%
\begin{figure}
    \centering
    \includegraphics[width=0.8\textwidth]{img/net_reactive_carbon_loss_pentane_toluene}
    \vspace{0mm}
    \caption{Daily net carbon loss rate during pentane and toluene degradation.}
    \vspace{-4mm}
    \label{f:net_carbon_loss}
\end{figure}
%
The net loss rate of reactive carbon during pentane and toluene degradation was determined by multiplying the rate of each reaction occurring during pentane and toluene degradation by its net carbon loss, shown in \mbox{Figure \ref{f:net_carbon_loss}}.
Pentane is broken down faster in CBM-IV, CB05, RADM2, RACM and RACM2 by losing reactive carbon more quickly than the MCM v3.2.
MOZART-4 also breaks pentane down into smaller sized products quicker than the MCM v3.2 as reactions during pentane degradation in MOZART-4 have organic products whose carbon number is less than the organic reactant, described in Section \ref{sss:profiles}.
The faster break down of pentane on the first day limits the amount of reactive carbon available to produce further \ce{O_x} on subsequent days leading to lower \ce{O_x} production after the first day in reduced mechanisms.

As seen in Figure \ref{f:TOPP_dailies}, \ce{O_x} produced during toluene degradation has a high spread between the mechanisms.
Figure \ref{f:carbon} shows differing distributions of the sizes of the degradation products that produce \ce{O_x}.
All reduced mechanisms omit \ce{O_x} production from at least one degradation fragment size which produces \ce{O_x} in the MCM v3.2, indicating that toluene is also broken down more quickly than more explicit mechanisms.
For example, toluene degradation in RACM2 does not produce \ce{O_x} from degradation products with six carbons unlike the \mbox{MCM v3.2}.  
Figure \ref{f:net_carbon_loss} shows that all reduced mechanisms lose reactive carbon during toluene degradation faster than the MCM v3.2.
Thus reduced mechanisms are unable to reach the \ce{O_x} production levels of explicit mechanisms.
