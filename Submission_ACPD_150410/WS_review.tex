\documentclass{article}

\usepackage{setspace}
\setstretch{1.5}
\usepackage{microtype}
\usepackage[round]{natbib}
\setlength{\bibhang}{0pt}

\begin{document}

We would like to thank Dr Stockwell for the review of our paper and feel that this review has enable us to improve our paper.
Our responses to the review are below.

We agree with Dr Stockwell that older versions of the chemical mechanisms in our study are relics of the past.
However, as highlighted in the recent review by \citet{Baklanov:2014}, all versions of the chemical mechanisms used in our study are actively used for modelling studies.
We shall update Section~2.1 stating that all chemical mechanisms are actively used in other modelling studeS and update Table~1 referencing a recent study using each of the reduced mechanisms used in our study.
Examples of such studies are \citet{Derwent:2015} for the CRI~v2, \citet{Li:2014} using RADM2, \citet{Ahmadov:2015} with RACM, \citet{Goliff:2015} using RACM2, \citet{Hou:2015} using MOZART-4, \citet{Foster:2014} with the CBM-IV and \citet{Dunker:2015} which uses the CB05.

We also agree with Dr Stockwell that the MCM should not be considered ``correct'' and we shall update Section~2.1 of the research article to avoid giving this impresssion.
We shall include references to the studies of \citet{Bloss:2005} and \citet{Pinho:2005} which illustrate that the MCM has difficulities in reproducing the results of chamber studies using aromatic VOC and isoprene.

As mentioned by Dr Stockwell, our paper compares the effects on ozone production from different approaches to simplifying the detailed atmospheric chemistry.
Our results show that the approach used to create the chemical mechanism rather than its explicitness influences ozone production.
For example, in Fig.~1 of our paper, the ozone mixing ratios obtained using the Carbon Bond mechanisms (CBM-IV and CB05) compare well with the MCM despite both Carbon Bond mechanisms being less explicit than the MCM.
Also, the ozone mixing ratios from RACM2 and RADM2 show similar absolute differences from that of the MCM despite RACM2 being more explicit than RADM2.
We shall update Section~3.1 to include the above discussion of the explicitness of a chemical mechanism in relation to ozone mixing ratios produced from the different chemical mechanisms.

Dr Stockwell mentions that we have incorrectly referred to the RADM2 mechanism as RADM.
In our paper, we have used the acronyms for the different chemical mechanisms provided by the base literature listed in Table~1 of the research article.
\citet{Stockwell:1990} is the base reference for RADM2 and we have accordingly used this acronym.
We have checked our paper and find no instance of any mechanism being called RADM.

\begin{thebibliography}{12}

    \bibitem[{Ahmadov et~al.(2015)Ahmadov, McKeen, Trainer, Banta, Brewer, Brown, Edwards, de Gouw, Frost, Gilman, Helmig, Johnson, Karion, Koss, Langford, Lerner, Olson, Oltmans, Peischl, P\'etron, Pichugina, Roberts, Ryerson, Schnell, Senff, Sweeney, Thompson, Veres, Warneke, Wild, Williams, Yuan, and Zamora}]{Ahmadov:2015} Ahmadov,~R. and McKeen,~S. and Trainer,~M. and Banta,~R. and Brewer,~A. and Brown,~S. and Edwards,~P.~M. and de Gouw,~J.~A. and Frost,~G.~J. and Gilman,~J. and Helmig,~D. and Johnson,~B. and Karion,~A. and Koss,~A. and Langford,~A. and Lerner,~B. and Olson,~J. and Oltmans,~S. and Peischl,~J. and P\'etron,~G. and Pichugina,~Y. and Roberts,~J.~M. and Ryerson,~T. and Schnell,~R. and Senff,~C. and Sweeney,~C. and Thompson,~C. and Veres,~P.~R. and Warneke,~C. and Wild,~R. and Williams,~E.~J. and Yuan,~B. and Zamora,~R.: Understanding high wintertime ozone pollution events in an oil- and natural gas-producing region of the western US, Atmospheric Chemistry and Physics, 15, 1, 411--429, 2015.

    \bibitem[{Archer-Nicholls et~al.(2014) Archer-Nicholls, Lowe, Utembe, Allan, Zaveri, Fast, Hodnebrog, Denier van der Gon, and McFiggans}]{Archer-Nicholls:2014} Archer-Nicholls,~S. and Lowe,~D. and Utembe,~S. and Allan,~J. and Zaveri,~R.~A. and Fast,~J.~D. and Hodnebrog,~{\O}. and Denier van der Gon,~H. and McFiggans,~G., Gaseous chemistry and aerosol mechanism developments for version 3.5.1 of the online regional model, WRF-Chem, Geoscientific Model Development, 7, 6, 2557--2579, 2014.

    \bibitem[{Baklanov et~al.(2014) Baklanov, Schl{\"u}nzen, Suppan, Baldasano, Brunner, Aksoyoglu, Carmichael, Douros, Flemming, Forkel, Galmarini, Gauss, Grell, Hirtl, Joffre, Jorba, Kaas, Kaasik, Kallos, Kong, Korsholm, Kurganskiy, Kushta, Lohmann, Mahura, Manders-Groot, Maurizi, Moussiopoulos, Rao, Savage, Seigneur, Sokhi, Solazzo, Solomos, S{\o}rensen, Tsegas, Vignati, Vogel, and Zhang}]{Baklanov:2014} Baklanov,~A., Schl{\"u}nzen,~K., Suppan,~P., Baldasano,~J., Brunner,~D., Aksoyoglu,~S., Carmichael,~G., Douros,~J., Flemming,~J., Forkel,~R., Galmarini,~S., Gauss,~M., Grell,~G., Hirtl,~M., Joffre,~S., Jorba,~O., Kaas,~E., Kaasik,~M., Kallos,~G., Kong,~X., Korsholm,~U., Kurganskiy,~A., Kushta,~J., Lohmann,~U., Mahura,~A., Manders-Groot,~A., Maurizi,~A., Moussiopoulos,~N., Rao,~S. T., Savage,~N., Seigneur,~C., Sokhi,~R. S., Solazzo,~E., Solomos,~S., S{\o}rensen,~B., Tsegas,~G., Vignati,~E., Vogel,~B., Zhang,~Y.: Online coupled regional meteorology chemistry models in Europe: current status and prospects, Atmospheric Chemistry and Physics, 14, 1, 317--398, 2014.

    \bibitem[{Bloss et~al.(2005)Bloss, Wagner, Jenkin, Vollamer, Bloss, Lee, Heard, Wirtz, Martin-Reviejo, Rea, Wenger, and Pilling}]{Bloss:2005} Bloss,~C., Wagner,~V., Jenkin,~M.~E., Volkamer,~R., Bloss,~W.~J., Lee,~J.~D., Heard,~D.~E., Wirtz,~K., Martin-Reviejo,~M., Rea,~G., Wenger,~J.~C., and Pilling,~M.~J.: Development of a detailed chemical mechanism (MCMv3.1) for the atmospheric oxidation of aromatic hydrocarbons, Atmos. Chem. Phys., 5, 641--664. %, doi:\href{http://dx.doi.org/10.5194/acp-5-641-2005}{10.5194/acp-5-641-2005}.

    \bibitem[{Derwent et~al.(2015)Derwent, Utembe, Jenkin, and Shallcross}]{Derwent:2015} Derwent,~R.~G., Utember,~S.~R., Jenkin,~M.~E., and Shallcross,~D.~E.: Tropospheric ozone production regions and the intercontinental origins of surface ozone over Europe, Atmospheric Environment, 112, 216--224, 2015

    \bibitem[{Dunker et.~al.(2015)Dunker, Koo, and Yarwood}]{Dunker:2015} Dunker,~A.~M., Koo,~B, and Yarwood,~G.: Source Apportionment of the Anthropogenic Increment to Ozone, Formaldehyde, and Nitrogen Dioxide by the Path- Integral Method in a 3D Model, Environmental Science \& Technology, 49, 11, 6751--6759, 2015.

    \bibitem[{Foster et~al.(2014)Foster, Prentice, Morfopoulos, Siddall, and van Weele}]{Foster:2014}Foster,~P.~N., Prentice,~I.~C., Morfopoulos,~C., Siddall,~M., and van Weele,~M.: Isoprene emissions track the seasonal cycle of canopy temperature, not primary production: evidence from remote sensing, Biogeosciences, 11, 13, 3427--3451, 2014.

    \bibitem[{Goliff et.~al.(2015)Goliff, Luria, Blake, Zielinska, Hallar, Valente, Lawson, and Stockwell}]{Goliff:2015} Goliff,~W.~S., Luria,~M., Blake,~D.~R., Zielinska,~B., Hallar,~G., Valente,~R.~J., Lawson,~C.~V., and Stockwell,~W.~R.:Nighttime air quality under desert conditions, Atmospheric Environment, 114, 102--111, 2015.

    \bibitem[{Hou et~al.(2015)Hou, Zhu, Fei, and Wang}]{Hou:2015} Hou,~X., Zhu,~B., Fei,~D., and Wang,~D.: The impacts of summer monsoons on the ozone budget of the atmospheric boundary layer of the Asia-Pacific region, Science of The Total Environment, 502, 641--649, 2015.

    \bibitem[{Li et.~al.(2014)Li, Georgescu, Hydpe, Mahalov, and Moustaui}]{Li:2014}Li,~J, Georgescu,~M., Hyde,~P., Mahalov,~A. and Moustaoui,~M.: Achieving accurate simulations of urban impacts on ozone at high resolution, Environmental Research Letters, 9, 11, 114019, 2014.

    \bibitem[{Pinho et~al.(2005)Pinho, Pio, and Jenkin}]{Pinho:2005} Pinho,~P.~G., Pio~C.~A., and Jenkin~ M.~E.: Evaluation of isoprene degradation in the detailed tropospheric chemical mechanism, MCM~v3, using environmental chamber data, Atmospheric Environment, 39, 7, 1303--1322, 2005.  2005.

    \bibitem[{Stockwell et~al.(1990)Stockwell, Middleton, Chang, and                                        Tang}]{Stockwell:1990} Stockwell,~W.~R., Middleton,~P., Chang,~J.~S., and Tang,~X.: {The second generation regional acid deposition model chemical mechanism for regional air quality modeling}, J. Geophys. Res., 95, 16343--16367, 1990.

\end{thebibliography}

\end{document}
