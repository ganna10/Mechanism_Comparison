\documentclass{article}

\usepackage{setspace}
\setstretch{1.5}
\usepackage{microtype}
\usepackage[round]{natbib}
\setlength{\bibhang}{0pt}

\begin{document}

We would first like to thank Dr Stockwell for the review of our paper and are grateful for the acknowledgment that our study compares the effects of different approaches to air quality mechanisms on maximum ozone production.

It is interesting that Dr Stockwell refers to the previous versions of the RACM2 mechanism as relics of the past as these mechanisms (RADM2 and RACM) as well as all the other reduced mechanisms in our paper, are still used by modelling groups for various air quality studies. 
We shall update the paper to illustrate that the mechanisms chosen in the mechanism comparison study are still being used by modelling groups.
In particular we shall include references to a recent study by \citet{Baklanov:2014} which lists the currently used online meteorology-chemistry models used for regional modelling studies in Europe.
Table 4 in \citet{Baklanov:2014} shows that each reduced mechanism consider in our study (except CRI v2) is used by some modelling group in Europe.
The CRI v2 has recently been implemented in the WRF-Chem model as outlined in \citet{Archer-Nicholls:2014}.

We shall also update Table 1 of the research paper with a recent study using each of the reduced mechanisms used in our study.
Examples of such studies are \citet{Derwent:2015} for the CRI v2, \citet{Li:2014a} using RADM2, \citet{Ahmadov:2015} with RACM, \citet{Goliff:2015} using RACM2, \citet{Hou:2015} using MOZART-4, \citet{Foster:2014} with the CBM-IV and \citet{Dunker:2015} which uses the CB05.

We agree with Dr Stockwell that the MCM should not be considered ``correct'' and our paper giving that impression, we shall update our paper with references to the fact that the MCM mechanism has shown difficulties in simulating chamber studies.
In particular, \citet{Bloss:2005} outlines the shortcomings of the MCM when simulating chamber studies including aromatic species and \citet{Pinho:2005} compares the isoprene degradation of the MCM v3 to chamber data.

In our paper, we have consistently named the RADM2 mechanism according to \citet{Stockwell:1990}, if this is incorrect we shall update the paper accordingly.

\bibliographystyle{copernicus}
\bibliography{References.bib}

\end{document}
