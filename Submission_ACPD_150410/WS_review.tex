\documentclass{article}

\usepackage{setspace}
\setstretch{1.5}
\usepackage{microtype}
\usepackage[round]{natbib}
\setlength{\bibhang}{0pt}

\begin{document}

We would like to thank Dr Stockwell for the review of our paper and our responses are below.

Dr Stockwell refers to previous versions of the RACM2 mechanism and many of the older chemical mechanisms in our study as relics of the past.
We chose the chemical mechanisms in our study as those being used by various modelling groups for air quality studies.
We shall update Section 2.1 of our paper to clarify the choice of chemical mechanisms by referencing the revent review of the regional meteorology chemistry models used in Europe by \citet{Baklanov:2014} in which each reduced chemical mechanism in our study is used by some modelling group.

We shall further update Table 1 of the research paper referencing a recent study using each of the reduced mechanisms used in our study.
Examples of such studies are \citet{Derwent:2015} for the CRI v2, \citet{Li:2014a} using RADM2, \citet{Ahmadov:2015} with RACM, \citet{Goliff:2015} using RACM2, \citet{Hou:2015} using MOZART-4, \citet{Foster:2014} with the CBM-IV and \citet{Dunker:2015} which uses the CB05.

We agree with Dr Stockwell that the MCM should not be considered ``correct'' and we shall update Section 2.1 of the research article to avoid giving this impresssion.
We shall include references to the studies of \citet{Bloss:2005} and \citet{Pinho:2005} which illustrate that the MCM has difficulities in reproducing the results of chamber studies using aromatic VOC and isoprene.

Finally, Dr Stockwell mentions that we have incorrectly referred to the RADM2 mechanism as RADM.
In our paper, we refer to the RADM2 mechanism as indicated by \citet{Stockwell:1990}.

\bibliographystyle{copernicus}
\bibliography{References.bib}

\end{document}
