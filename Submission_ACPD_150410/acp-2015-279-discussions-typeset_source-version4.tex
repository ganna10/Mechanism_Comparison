\documentclass[acpd, online, hvmath]{copernicus}

\begin{document}\hack{\sloppy}

\title{A comparison of chemical mechanisms using Tagged Ozone Production Potential (TOPP) analysis}


\Author{J.}{Coates}
\Author{T.~M.}{Butler}

\affil{Insititute for Advanced Sustainability Studies, Potsdam, Germany}



\runningtitle{A comparison of chemical mechanisms using TOPP analysis}

\runningauthor{J.~Coates and T.~M.~Butler}

\correspondence{J.~Coates (jane.coates@iass-potsdam.de)}


\received{10~April~2015}
\accepted{13~April~2015}
\published{}

\firstpage{1}

\maketitle



\begin{abstract}
  Ground-level ozone is a~secondary pollutant produced photochemically
  from reactions of NO$_{x}$ with peroxy radicals produced during
  VOC degradation.  Chemical transport models use simplified
  representations of this complex gas-phase chemistry to predict
  \chem{O_3} levels and inform emission control strategies.  Accurate
  representation of \chem{O_3} production chemistry is vital for
  effective predictions.  In this study, VOC degradation chemistry in
  simplified mechanisms is compared to that in the near-explicit MCM
  mechanism using a~boxmodel and by ``tagging'' all organic
  degradation products over multi-day runs, thus calculating the
  Tagged Ozone Production Potential (TOPP) for a~selection of VOC
  representative of urban airmasses.  Simplified mechanisms that
  aggregate VOC degradation products instead of aggregating emitted
  VOC produce comparable amounts of \chem{O_3} from VOC degradation to
  the MCM.  First day TOPP values are similar across mechanisms for
  most VOC, with larger discrepancies arising over the course of the
  model run.  Aromatic and unsaturated aliphatic VOC have largest
  inter-mechanisms differences on the first day, while alkanes show
  largest differences on the second day.  Simplified mechanisms break
  down VOC into smaller sized degradation products on the first day
  faster than the MCM impacting the total amount of \chem{O_3}
  produced on subsequent days due to secondary chemistry.
\end{abstract}



\introduction

Ground-level ozone (\chem{O_3}) is both an air pollutant and a~climate
forcer that is detrimental to human health and crop growth
\citep{Stevenson:2013}.  \chem{O_3} is produced from the reactions of
volatile organic compounds (VOCs) and nitrogen oxides
(NO$_{x}=\chem{NO}+\chem{NO_2}$) in the presence of sunlight
\citep{Atkinson:2000}.

Background \chem{O_3} concentrations have increased during the last
several decades due to the increase of overall global anthropogenic
emissions of \chem{O_3} precursors \citep{HTAP:2010}.  Despite
decreases in emissions of \chem{O_3} precursors over Europe since
1990, \citet{AQEU:2014} reports that $98$\,{\%} of Europe's urban
population are exposed to levels exceeding the WHO air quality
guideline of $100$\,\unit{{\mu}g\,m^{-3}} over an $8$\,h mean.  These
exceedances result from local and regional \chem{O_3} precursor gas
emissions, their intercontinental transport and the non-linear
relationship of \chem{O_3} concentrations to NO$_{x}$ and VOC
levels \citep{AQEU:2014}.

Effective strategies for emission reductions rely on accurate
predictions of \chem{O_3} concentrations using chemical transport
models (CTMs).  These predictions require adequate representation of
gas-phase chemistry in the chemical mechanism used by the CTM.  For
reasons of computational efficiency, the chemical mechanisms used by
global and regional CTMs must be simpler than the nearly-explicit
mechanisms which can be used in box modelling studies.  This study
compares the impacts of different simplification approaches of
chemical mechanisms on \chem{O_3} production chemistry focusing on the
role of VOC degradation products.
\hack{\arraycolsep0pt}
\begin{rxnarray}
&&    \chem{NO} + \chem{O_3} \rightarrow \chem{NO_2} + \chem{O_2} \label{r:NO_O3}\\
&&    \chem{NO_2} + h\nu \rightarrow \chem{NO} + \chem{O(^3P)} \label{r:NO2_hv}\\
&&    \chem{O_2} + \chem{O(^3P)} + \chem{M} \rightarrow \chem{O_3} + \chem{M} \label{r:O2_O3P}
\end{rxnarray}
The photochemical cycle (Reactions~\ref{r:NO_O3}--\ref{r:O2_O3P})
rapidly produces and destroys \chem{O_3}.  \chem{NO} and \chem{NO_2}
reach a~near-steady state via Reactions~(\ref{r:NO_O3}) and
(\ref{r:NO2_hv}) which is disturbed in two cases.  Firstly, via
\chem{O_3} removal (deposition or Reaction~\ref{r:NO_O3} during
night-time and near large \chem{NO} sources) and secondly, when
\chem{O_3} is produced through VOC--NO$_{x}$ chemistry
\citep{Sillman:1999}.

VOCs (RH) are oxidised in the troposphere by the hydoxyl radical (OH)
forming peroxy radicals (\chem{RO_2}) in the presence of \chem{O_2}
(Reaction~\ref{r:RH_OH}).  In high-NO$_{x}$ conditions, typical of
urban environments, \chem{RO_2} react with \chem{NO}
(Reaction~\ref{r:RO2_NO}) to form alkoxy radicals (\chem{RO}), which
react quickly with \chem{O_2} (Reaction~\ref{r:RO_O2}) producing
a~hydroperoxy radical (\chem{HO_2}) and a~carbonyl species
(\chem{R^\prime CHO}).  The secondary chemistry of these first
generation carbon-containing oxidation products is analogous to the
sequence (Reactions~\ref{r:RH_OH}--\ref{r:RO_O2}), producing further
\chem{HO_2} and \chem{RO_2} radicals.  Subsequent generation oxidation
products can continue to react, producing \chem{HO_2} and \chem{RO_2}
until they have been completely oxidised to \chem{CO_2} and
\chem{H_2O}.  Both \chem{RO_2} and \chem{HO_2} react with \chem{NO} to
produce \chem{NO_2} (Reactions~\ref{r:RO2_NO} and \ref{r:HO2_NO})
leading to \chem{O_3} production via Reactions~(\ref{r:NO2_hv}) and
(\ref{r:O2_O3P}).  Thus the amount of \chem{O_3} produced from VOC
degradation is related to the number of \chem{NO} to \chem{NO_2}
conversions by \chem{RO_2} and \chem{HO_2} radicals formed during VOC
degradation \citep{Atkinson:2000}.  \hack{\arraycolsep0pt}
\begin{rxnarray}
&&    \chem{RH} + \chem{OH} + \chem{O_2} \rightarrow \chem{RO_2} + \chem{H_2O}\label{r:RH_OH}\\
&&    \chem{RO_2} + \chem{NO} \rightarrow \chem{RO} + \chem{NO_2}\label{r:RO2_NO}\\
&&    \chem{RO} + \chem{O_2} \rightarrow \chem{R{^\prime}CHO} + \chem{HO_2}\label{r:RO_O2}\\
&&    \chem{HO_2} + \chem{NO} \rightarrow \chem{OH} + \chem{NO_2}\label{r:HO2_NO}
\end{rxnarray}
Three atmospheric regimes with respect to \chem{O_3} production can be
defined \citep{Jenkin:2000}.  In the NO$_{x}$-sensitive regime, VOC
concentrations are much higher than those of NO$_{x}$ and
\chem{O_3} production depends on NO$_{x}$ concentrations.  On the
other hand, when NO$_{x}$ concentrations are much higher than those
of VOC (VOC-sensitive regime), VOC concentrations determine the amount
of \chem{O_3} produced.  Finally, the NO$_{x}$-VOC-sensitive regime
produces maximal \chem{O_3} and is controlled by both VOC and
NO$_{x}$ concentrations.

These atmospheric regimes remove radicals through distinct mechanisms
\citep{Kleinman:1991}.  In the NO$_{x}$-sensitive regime, radical
concentrations are high relative to NO$_{x}$ leading to radical
removal by radical combination Reaction~(\ref{r:RO2_HO2}) and
bimolecular destruction Reaction~(\ref{r:HO2_OH})
\citep{Kleinman:1994}.  \hack{\arraycolsep0pt}
\begin{rxnarray}
&&    \chem{RO_2} + \chem{HO_2} \rightarrow \chem{ROOH} + \chem{O_2} \label{r:RO2_HO2}\\
&&    \chem{HO_2} + \chem{OH} \rightarrow \chem{H_2O} + \chem{O_2} \label{r:HO2_OH}
\end{rxnarray}
Whereas in the VOC-sensitive regime, radicals are removed by reacting
with \chem{NO_2} leading to nitric acid (\chem{HNO_3})
(Reaction~\ref{r:NO2_OH}) and PAN species
(Reaction~\ref{rRCOO}).  \hack{\arraycolsep0pt}
\begin{rxnarray}
  &&    \chem{NO_2} + \chem{OH} \rightarrow \chem{HNO_3} \label{r:NO2_OH}\\
  && \chem{RC(O)O_2} + \chem{NO_2} \rightarrow
  \chem{RC(O)O_2NO_2} \label{rRCOO}
\end{rxnarray}
The NO$_{x}$-VOC-sensitive regime has no dominant radical removal
mechanism as radical and NO$_{x}$ amounts are comparable.  This
chemistry results in \chem{O_3} concentrations being a~non-linear
function of NO$_{x}$ and VOC concentrations.

Individual VOC impact \chem{O_3} production differently through their
diverse reaction rates and degradation pathways.  These impacts can be
quantified using Ozone Production Potentials (OPP) which can be
calculated through incremental reactivity (IR) studies using
photochemical models.  In IR studies, VOC concentrations are changed
by a~known increment and the change in \chem{O_3} production is
compared to that of a~standard VOC mixture.  Examples of IR scales are
the Maximal Incremental Reactivity (MIR) and Maximum Ozone Incremental
Reactivity (MOIR) scales in \citet{Carter:1994}, as well as the
Photochemical Ozone Creation Potential (POCP) scale of
\citet{Derwent:1996,Derwent:1998}.  The MIR, MOIR and POCP
scales were calculated under different NO$_{x}$ conditions, thus
calculating OPPs in different atmospheric regimes.

\citet{Butler:2011} calculate the maximum potential of VOC to produce
\chem{O_3} by using NO$_{x}$ conditions inducing
NO$_{x}$-VOC-sensitive chemistry over multi-day scenarios using a
``tagging'' approach -- the Tagged Ozone Production Potential (TOPP).
Tagging involves labelling all organic degradation products produced
during VOC degradation with the name of the emitted VOC.  Tagging
enables the attribution of \chem{O_3} production from VOC degradation
products back to the emitted VOC, thus providing a~detailed insight
into VOC degradation chemistry.  \citet{Butler:2011}, using
a~near-explicit chemical mechanism, showed that some VOC, such as
alkanes, produce maximum \chem{O_3} on the second day of the model
run; in contrast to unsaturated aliphatic and aromatic VOC which
produce maximum \chem{O_3} on the first day.  In this study, the
tagging approach of \citet{Butler:2011} is applied to several chemical
mechanisms of reduced complexity, using conditions of maximum
\chem{O_3} production (NO$_{x}$-VOC-sensitive regime), to compare
the effects of different representations of VOC degradation chemistry
on \chem{O_3} production in the different chemical mechanisms.

A~near-explicit mechanism, such as the Master Chemical Mechanism (MCM)
\citep{Jenkin:2003, Saunders:2003, Bloss:2005}, includes detailed
degradation chemistry making the MCM ideal as a~reference for
comparing chemical mechanisms.  Reduced mechanisms generally take two
approaches to simplifying the representation of VOC degradation
chemistry: lumped structure approaches; and lumped molecule approaches
\citep{Dodge:2000}.

Lumped structure mechanisms speciate VOC by the carbon bonds of the
emitted VOC, examples are the Carbon Bond mechanisms, CBM-IV
\citep{Gery:1989} and CB05 \citep{Yarwood:2005}.  Lumped molecule
mechanisms represent VOC explicitly or by aggregating (lumping) many
VOC into a~single mechanism species.  Mechanism species may lump VOC
by functionality (MOdel for Ozone and Related chemical Tracers,
MOZART-4, \citealp{Emmons:2010}) or OH-reactivity (Regional Acid
Deposition Model, RADM2 \citep{Stockwell:1990}, Regional Atmospheric
Chemistry Mechanisms, RACM \citep{Stockwell:1997} and RACM2
\citep{Goliff:2013}).  The Common Representative Intermediates
mechanism (CRI) lumps the degradation products of VOC rather than the
emitted VOC \citep{Jenkin:2008}.

Many comparison studies of chemical mechanisms consider modelled time
series of \chem{O_3} concentrations over varying VOC and NO$_{x}$
concentrations.  Examples are \citet{Dunker:1984,Kuhn:1998} and
\citet{Emmerson:2009}.  The largest discrepancies between the time
series of \chem{O_3} concentrations in different mechanisms from these
studies arise when modelling urban rather than rural conditions and
are attributed to the treatment of radical production, organic nitrate
and night-time chemistry.  \citet{Emmerson:2009} also compare the
inorganic gas-phase chemistry of different chemical mechanisms,
differences in inorganic chemistry arise from inconsistencies between
IUPAC and JPL reaction rate constants.

Mechanisms have also been compared using OPP scales.  OPPs are
a~useful comparison tool as they relate \chem{O_3} production to
a~single value.  \citet{Derwent:2010} compared the near-explicit MCM
v3.1 and SAPRC-07 mechanisms using first-day POCP values calculated
under VOC-sensitive conditions.  The POCP values were comparable
between the mechanisms.  \citet{Butler:2011} compared first day TOPP
values to the corresponding published MIR, MOIR and POCP values.  TOPP
values were most comparable to MOIR and POCP values due to the
similarity of the chemical regimes used in their calculation.

In this study, we compare TOPP values of VOC using a~number of
mechanisms to those calculated with the MCM v3.2, under standardised
conditions which maximise \chem{O_3} production.  Differences in
\chem{O_3} production are explained by the differing treatments of
secondary VOC degradation in these mechanisms.



\section{Methodology}
\label{s:methodology}


\subsection{Chemical mechanisms}
\label{ss:mechanisms}

The nine chemical mechanisms compared in this study are outlined in
Table~\ref{t:mechanisms} with a~brief summary below.  The reduced
mechanisms in this study were chosen as they are commonly used in 3-D
models and apply different approaches to representing secondary VOC
chemistry.

The MCM \citep{Jenkin:1997, Jenkin:2003, Saunders:2003, Bloss:2005,
  MCM_Site} is a~near-explicit mechanism describing the degradation of
$125$ primary VOC.  The MCM v3.2 is the reference mechanism in this
study.

The CRI \citep{Jenkin:2008} is a~reduced chemical mechanism describing
the oxidation of the same primary VOC as the MCM v3.1.  VOC
degradation in the CRI is simplified by lumping the degradation
products of many VOC into mechanism species whose overall \chem{O_3}
production reflects that of the MCM v3.1.  The full version of the CRI
v2 (\url{http://mcm.leeds.ac.uk/CRI}) is used in this study.
Differences in \chem{O_3} production between the CRI v2 and MCM v3.2
may be due to changes in the MCM versions rather than the CRI
reduction techniques, hence the MCM v3.1 is also included in this
study.

MOZART-4 represents global tropospheric and stratospheric chemistry
\citep{Emmons:2010}.  Explicit species exist for methane, ethane,
propane, ethene, propene, isoprene and $\alpha$-pinene.  All other VOC
are represented by lumped species determined by the functionality of
the VOC.

RADM2 \citep{Stockwell:1990} describes regional scale atmospheric
chemistry with explicit species representing methane, ethane, ethene
and isoprene.  All other VOC are assigned to lumped species based on
OH-reactivity and molecular weight.  RADM2 was updated to RACM
\citep{Stockwell:1997} with more explicit and lumped species
representing VOC as well as revised chemistry.  RACM2 is the updated
RACM version \citep{Goliff:2013} with substantial updates to the
chemistry, including more lumped and explicit species representing
emitted VOC.

CBM-IV \citep{Gery:1989} simulates polluted urban conditions and
represents ethene, formaldehyde and isoprene explicitly while all
other emitted VOC are lumped by their carbon bond types.  All primary
VOC were assigned to lumped species in CBM-IV as described in
\citet{Hogo:1989}.  For example, the mechanism species PAR represents
the C--C bond.  Pentane, having five carbon atoms, is represented as
$5$ PAR.  A~pentane mixing ratio of $1200$\,\unit{pptv} would be
assigned to $6000$ ($= 1200 \times 5$)\,pptv of PAR in CBM-IV.  CBM-IV
was updated to CB05 \citep{Yarwood:2005} by including further explicit
species representing methane, ethane and acetaldehyde.  Other updates
include revised allocation of primary VOC and updated rate constants.
\subsection{Model Setup} \label{ss:model_setup} The modelling approach
and set-up follows the original TOPP study of \citet{Butler:2011}.
The approach is summarised here; further details can be found in the
Supplement and in \citet{Butler:2011}.  We use
the MECCA boxmodel, originally described by \citep{Sander:2005}, and
as subsequently modified by \citet{Butler:2011} to include MCM
chemistry.  In this study, the model is run under conditions
representative of $34${\degree}\,N at the
equinox (broadly representative of the city of Los Angeles, USA).

Maximum \chem{O_3} production is achieved in each model run by
balancing the chemical source of radicals and NO$_{x}$ at each
timestep by emitting the appropriate amount of NO.  These NO$_{x}$
conditions induce NO$_{x}$-VOC-sensitive chemistry.  Ambient
NO$_{x}$ conditions are not required as this study calculates the
maximum potential of VOC to produce \chem{O_3}.  Future work should
verify the extent to which the maximum potential of VOC to produce
\chem{O_3} is reached under ambient NO$_{x}$ conditions.

VOCs typical of Los Angeles and their initial mixing ratios are taken
from \citet{Baker:2008}, listed in Table~\ref{t:initial_conditions}.
Following \citet{Butler:2011}, the associated emissions required to
keep the initial mixing ratios of each VOC constant until noon of the
first day were determined for the MCM v3.2.  These emissions are
subsequently used for each mechanism, ensuring the amount of each VOC
emitted was the same in every model run.  Methane (\chem{CH_4}) was
fixed at $1.8$\,\unit{ppmv} while \chem{CO} and \chem{O_3} were
initialised at $200$ and $40$\,\unit{ppbv} and then
allowed to evolve freely.

The VOCs used in this study are assigned to mechanism species
following the recommendations from the literature of each mechanism
(Table~\ref{t:mechanisms}), the representation of each VOC in the
mechanisms is found in Table~\ref{t:initial_conditions}.  Emissions of
lumped species are weighted by the carbon number of the mechanism
species ensuring the total amount of emitted reactive carbon was the
same in each model run.

The MECCA boxmodel is based upon the Kinetic Pre-Processor (KPP)
\citep{Damian:2002}.  Hence, all chemical mechanisms were adapted into
modularised KPP format.  The inorganic gas-phase chemistry described
in the MCM v3.2 was used in each run to remove any differences between
treatments of inorganic chemistry in each mechanism.  Thus differences
between the \chem{O_3} produced by the mechanisms are due to the
treatment of organic degradation chemistry.

The MCM v3.2 approach to photolysis, dry deposition of VOC oxidation
intermediates and \chem{RO_2-\chem{RO_2}} reactions was used for each
mechanism; details of these adaptations can be found in the Supplement.  Some mechanisms include reactions which are
only important in the stratosphere or free troposphere.  For example,
PAN photolysis is only important in the free troposphere
\citep{Harwood:2003} and was removed from MOZART-4, RACM2 and CB05 for
the purpose of the study, as this study considers processes occurring
within the planetary boundary layer.


\subsection{Tagged Ozone Production Potential (TOPP)}

This section summarises the tagging approach described in \citet{Butler:2011} which is applied in this study.

\subsubsection{O$_{x}$ family and tagging approach}
\label{ss:tagging}

\chem{O_3} production and loss is dominated by rapid photochemical
cycles, such as Reactions~(\ref{r:NO_O3})--(\ref{r:O2_O3P}).  The
effects of rapid production and loss cycles can be removed by using
chemical families that include rapidly inter-converting species.  In
this study, we define the O$_{x}$ family to include \chem{O_3},
\chem{O(^3P)}, \chem{O(^1D)}, \chem{NO_2} and other species involved
in fast cycling with \chem{NO_2}, such as \chem{HO_2NO_2} and PAN
species.  Thus, production of O$_{x}$ can be used as a~proxy for
production of \chem{O_3}.

The tagging approach follows the degradation of emitted VOC through
all possible pathways by labelling every organic degradation product
with the name of the emitted VOC.  Thus, each emitted VOC effectively
has its own set of degradation reactions.  \citet{Butler:2011} showed
that O$_{x}$ production can be attributed to the VOC by following
the tags of each VOC.

O$_{x}$ production from lumped mechanism species are re-assigned to
the VOC of Table~\ref{t:initial_conditions} by scaling the O$_{x}$
production of the mechanism species by the fractional contribution of
each represented VOC.  For example, TOL in RACM2 represents toluene
and ethylbenzene with fractional contributions of $0.87$ and $0.13$ to
TOL emissions.  Scaling the O$_{x}$ production from TOL by these
factors gives the O$_{x}$ production from toluene and ethylbenzene
in RACM2.

Many reduced mechanisms use an operator species as a~surrogate for
\chem{RO_2} during VOC degradation enabling these mechanisms to
produce O$_{x}$ while minimising the number of \chem{RO_2} species
represented.  O$_{x}$ production from operator species is assigned
as O$_{x}$ production from the organic degradation species
producing the operator.  This allocation technique is also used to
assign O$_{x}$ production from \chem{HO_2} via
Reaction~(\ref{r:HO2_NO}).

\subsubsection{Definition of the Tagged Ozone Production Potential
  (TOPP)}
\label{sss:TOPP}

Attributing O$_{x}$ production to individual VOC using the tagging
approach is the basis for calculating the TOPP of a~VOC, which is
defined as the number of O$_{x}$ molecules produced per emitted
molecule of VOC.  The TOPP value of a~VOC that is not represented
explicitly in a~chemical mechanism is calculated by multiplying the
TOPP value of the mechanism species representing the VOC by the ratio
of the carbon numbers of the VOC to the mechanism species.  For
example, CB05 represents hexane as $6$ PAR, so the TOPP value of
hexane in the CB05 is $6$ times the TOPP of PAR.  MOZART-4 represents
hexane by the five carbon species BIGALK.  Thus hexane emissions are
represented molecule for molecule as $\frac{6}{5}$ of the equivalent
number of molecules of \mbox{BIGALK}, and the TOPP value of hexane in
MOZART-4 is calculated by multiplying the TOPP value of BIGALK by
$\frac{6}{5}$.



\section{Results}
\label{s:results}


\subsection{Ozone time series and O$_{x}$ production budgets}
\label{ss:O3_time_series}

Figure~\ref{f:time_series} shows the time series of \chem{O_3} mixing
ratios obtained with each mechanism.  There is an $8$\,\unit{ppbv}
difference in \chem{O_3} mixing ratios on the first day between RADM2,
which has the highest \chem{O_3}, and RACM2, which has the lowest
\chem{O_3} mixing ratios when not considering the outlier time series
of RACM.  The difference between RADM2 and RACM, the low outlier, was
$21$\,\unit{ppbv} on the first day.  The \chem{O_3} mixing ratios in
the CRI v2 are larger than those in the MCM v3.1, which is similar to
the results in \citet{Jenkin:2008} where the \chem{O_3} mixing ratios
of the CRI v2 and MCM v3.1 are compared over a~five day period.

The day-time O$_{x}$ production budgets allocated to VOC for each
mechanism are shown in Fig.~\ref{f:Ox_tagged_budgets}.  The
relationships between \chem{O_3} mixing ratios in
Fig.~\ref{f:time_series} are mirrored in
Fig.~\ref{f:Ox_tagged_budgets} where mechanisms producing high amounts
of O$_{x}$ also have high \chem{O_3} mixing ratios.  The conditions
in the box model lead to a~daily maximum of OH that increases with
each day leading to an increase on each day in both the reaction rate
of the OH-oxidation of \chem{CH_4} and the daily contribution of
\chem{CH_4} to O$_{x}$ production.

The first day mixing ratios of \chem{O_3} in RACM are lower than other
mechanisms due to a~lack of O$_{x}$ production from aromatic VOC on
the first day in RACM (Fig.~\ref{f:Ox_tagged_budgets}).  Aromatic
degradation chemistry in RACM results in net loss of O$_{x}$ on the
first day, described later in Sect.~\ref{sss:day1}.

RADM2 is the only reduced mechanism producing higher \chem{O_3} mixing
ratios than the more detailed mechanisms (MCM v3.2, MCM v3.1 and CRI
v2).  Higher mixing ratios of \chem{O_3} in RADM2 are produced due to
increased O$_{x}$ production from propane compared to the MCM v3.2;
on the first day, the O$_{x}$ production from propane in RADM2 is
triple that of the MCM v3.2 (Fig.~\ref{f:Ox_tagged_budgets}).  Propane
is represented as HC3 in RADM2 \citep{Stockwell:1990} and on the first
day HC3 degradation produces about $17$ times the amount of
acetaldehyde (\chem{CH_3CHO}) produced by the MCM v3.2.  The
OH-oxidation of \chem{CH_3CHO} starts a~degradation chain that
produces O$_{x}$ through the reactions of \chem{CH_3CO_3} and
\chem{CH_3O_2} with \chem{NO}; thus the higher amounts of
\chem{CH_3CHO} in RADM2 during propane degradation leads to increased
O$_{x}$ production from propane degradation in RADM2 compared to
the MCM v3.2.


\subsection{Time dependent O$_{x}$ production}

Time series of daily TOPP values for each VOC are presented in
Fig.~\ref{f:TOPP_dailies} and the cumulative TOPP values at the end of
the model run obtained for each VOC using each of the mechanisms,
normalised by the number of atoms of C in each VOC are presented in
Table~\ref{t:cumulative_TOPPs_per_C}.  In the MCM and CRI v2, the
cumulative TOPP values obtained for each VOC show that by the end of
the model run larger alkanes have produced more O$_{x}$ per unit of
reactive C than alkenes or aromatic VOC.  By the end of the runs using
the lumped structure mechanisms (CBM-IV and CB05), alkanes produce
similar amounts of O$_{x}$ per reactive C while aromatic VOC and
some alkenes produce less O$_{x}$ per reactive C than the MCM.
Whereas in lumped molecule mechanisms (MOZART-4, RADM2, RACM, RACM2),
practically all VOC produce less O$_{x}$ per reactive C than the
MCM by the end of the run.  This lower efficiency of O$_{x}$
production from many individual VOC in lumped molecule and structure
mechanisms would lead to an underestimation of \chem{O_3} levels
downwind of an emission source, and a~smaller contribution to
background \chem{O_3} when using lumped molecule and structure
mechanisms.

The lumped intermediate mechanism (CRI v2) produces the most similar
O$_{x}$ to the MCM v3.2 for each VOC, seen in
Fig.~\ref{f:TOPP_dailies} and Table~\ref{t:cumulative_TOPPs_per_C}.
Higher variability in the time dependent O$_{x}$ production is
evident for VOC represented by lumped mechanism species.  For example,
2-methylpropene, represented in the reduced mechanisms by a~variety of
lumped species, has a~higher spread in time dependent O$_{x}$
production than ethene, which is explicitly represented in each
mechanism.

In general, the largest differences in O$_{x}$ produced by aromatic
and alkene species are on the first day of the simulations, while the
largest inter-mechanism differences in O$_{x}$ produced by alkanes
are on the second and third days of the simulations.  The reasons for
these differences in behaviour will be explored in
Sect.~\ref{sss:day1} which examines differences in first day
O$_{x}$ production between the chemical mechanisms and
Sect.~\ref{sss:profiles} which examines the differences in O$_{x}$
production on subsequent days.

\subsubsection{First day ozone production}
\label{sss:day1}

The first day TOPP values of each VOC from each mechanism,
representing \chem{O_3} production from freshly emitted VOC near their
source region, are compared to those obtained with the MCM v3.{2} in
Fig.~\ref{f:first_day}.  The root mean square error (RMSE) of all
first day TOPP values in each mechanism relative to those in the MCM
v3.2 are also included in Fig.~\ref{f:first_day}.  The RMSE value of
the CRI v2 shows that O$_{x}$ production on the first day from
practically all the individual VOC matches that in the MCM v3.2.  All
other reduced mechanisms have much larger RMSE values indicating that
the first day O$_{x}$ production from the majority of the VOC
differs from that in the MCM v3.2.

The reduced complexity of reduced mechanisms means that aromatic VOC
are typically represented by one or two mechanism species leading to
differences in O$_{x}$ production of the actual VOC compared to the
MCM v3.2.  For example, all aromatic VOC in MOZART-4 are represented
as toluene, thus less reactive aromatic VOC, such as benzene, produce
higher O$_{x}$ whilst more reactive aromatic VOC, such as the
xylenes, produce less O$_{x}$ in MOZART-4 than the MCM v3.2.  RACM2
includes explicit species representing benzene, toluene and each
xylene resulting in O$_{x}$ production that is the most similar to
the MCM v3.2 than other reduced mechanisms.

Figure~\ref{f:TOPP_dailies} shows a~high spread in O$_{x}$
production from aromatic VOC on the first day indicating that aromatic
degradation is treated differently between mechanisms.  Toluene
degradation is examined in more detail by comparing the reactions
contributing to O$_{x}$ production and loss in each mechanism,
shown in Fig.~\ref{f:toluene_Ox}.  These reactions are determined by
following the ``toluene'' tags in the tagged version of each
mechanism.

Toluene degradation in RACM includes several reactions consuming
O$_{x}$ that are not present in the MCM resulting in net loss of
O$_{x}$ on the first two days.  Ozonolysis of the cresol OH-adduct
mechanism species ADDC contributes significantly to O$_{x}$ loss in
RACM.  This reaction was included in RACM due to improved cresol
product yields when comparing RACM predictions with experimental data
\citep{Stockwell:1997}.  Other mechanisms that include cresol
OH-adduct species do not include ozonolysis and these reactions are
not included in the updated RACM2.

The total O$_{x}$ produced on the first day during toluene
degradation in each reduced mechanism is less than that in the MCM
v3.2 (Fig.~\ref{f:toluene_Ox}).  Less O$_{x}$ is produced in all
reduced mechanisms due to a~faster break down of the VOC into smaller
fragments than the MCM, described later in Sect.~\ref{ss:products}.
Moreover in CBM-IV and CB05, less O$_{x}$ is produced during
toluene degradation as reactions of the toluene degradation products
\chem{CH_3O_2} and \chem{CO} do not contribute to the O$_{x}$
production budgets, which is not the case in any other mechanism
(Fig.~\ref{f:toluene_Ox}).

Maximum O$_{x}$ production from toluene degradation in CRI v2 and
RACM2 is reached on the second day in contrast to the MCM v3.2 which
produces peak O$_{x}$ on the first day.  The second day maximum of
O$_{x}$ production in CRI v2 and RACM2 from toluene degradation
results from increased \chem{C_2H_5O_2} production from degradation of
unsaturated dicarbonyls; \chem{C_2H_5O_2} is not produced during
degradation of unsaturated dicarbonyls in the MCM v3.2.

Unsaturated aliphatic VOC generally produce similar amounts of
O$_{x}$ between mechanisms, especially explicitly represented VOC,
such as ethene and isoprene.  On the other hand, unsaturated aliphatic
VOC that are not explicitly represented produce differing amounts of
O$_{x}$ between mechanisms (Fig.~\ref{f:TOPP_dailies}).  For
example, the O$_{x}$ produced during $2$-methylpropene degradation
varies between mechanisms; differing rate constants of initial
oxidation reactions and non-realistic secondary chemistry lead to
these differences, further details are found in the Supplement.

Non-explicit representations of aromatic and unsaturated aliphatic VOC
coupled with differing degradation chemistry and a~faster break down
into smaller size degradation products results in different O$_{x}$
production in lumped molecule and lumped structure mechanisms compared
to the MCM v3.2.

\subsubsection{Ozone production on subsequent days}
\label{sss:profiles}

Alkane degradation in CRI v2 and both MCM mechanisms produces a~second
day maximum in O$_{x}$ that increases with alkane carbon number
(Fig.~\ref{f:TOPP_dailies}).  The increase in O$_{x}$ production on
the second day is reproduced for each alkane by the reduced
mechanisms; except octane in RADM2, RACM and RACM2.  However, larger
alkanes produce less O$_{x}$ than the MCM on the second day in all
lumped molecule and structure mechanisms.

The lumped molecule mechanisms (MOZART-4, RADM2, RACM and RACM2)
represent many alkanes by mechanism species which may lead to
unrepresentative secondary chemistry for alkane degradation.  For
example, three times more O$_{x}$ is produced during the
degradation of propane in RADM2 than the MCM v3.2 on the first day
(Fig.~\ref{f:Ox_tagged_budgets}).  Propane is represented in RADM2 by
the mechanism species HC3 which also represents other classes of VOC,
such as alcohols.  The secondary chemistry of HC3 is tailored to
produce O$_{x}$ from these different VOC and differs from alkane
degradation in the MCM v3.2 by producing more \chem{CH_3CHO} in RADM2.

As will be shown in Sect.~\ref{ss:products}, another feature of
reduced mechanisms is that the breakdown of emitted VOC into smaller
sized degradation products is faster than the MCM.  Alkanes are broken
down quicker in CBM-IV, CB05, RADM2, RACM and RACM2 through a~higher
rate of reactive carbon loss than the MCM v3.2 (shown for pentane and
octane in Fig.~\ref{f:net_carbon_loss}); reactive carbon is lost
through reactions not conserving carbon.  Despite many degradation
reactions of alkanes in MOZART-4 almost conserving carbon, the organic
products have less reactive carbon than the organic reactant also
speeding up the breakdown of the alkane compared to the MCM v3.2.

For example, Fig.~\ref{f:HC5P_NO} shows the distribution of reactive
carbon in the reactants and products from the reaction of \chem{NO}
with the pentyl peroxy radical in both MCM mechanisms and each lumped
molecule mechanism.  In all the lumped molecule mechanisms, the
individual organic products have less reactive carbon than the organic
reactant.  Moreover, in RADM2, RACM and RACM2 this reaction does not
conserve reactive carbon leading to faster loss rates of reactive
carbon.

The faster breakdown of alkanes in lumped molecule and structure
mechanisms on the first day limits the amount of O$_{x}$ produced
on the second day, as less of the larger sized degradation products
are available for further degradation and O$_{x}$ production.


\subsection{Treatment of degradation products}
\label{ss:products}

The time dependent O$_{x}$ production of the different VOC in
Fig.~\ref{f:TOPP_dailies} results from the varying rates at which VOC
break up into smaller fragments \citep{Butler:2011}.  Varying break
down rates of the same VOC between mechanisms could explain the
different time dependent O$_{x}$ production between mechanisms.
The break down of pentane and toluene between mechanisms is compared
in Fig.~\ref{f:carbon} by allocating the O$_{x}$ production to the
number of carbon atoms in the degradation products responsible for
O$_{x}$ production on each day of the model run in each mechanism.
Some mechanism species in RADM2, RACM and RACM2 have fractional carbon
numbers \citep{Stockwell:1990, Stockwell:1997, Goliff:2013} and
O$_{x}$ production from these species was reassigned as O$_{x}$
production of the nearest integral carbon number.

The degradation of pentane, a~five-carbon VOC, on the first day in the
MCM v3.2 produces up to $50$\,{\%} more O$_{x}$ from degradation
products also having five carbon atoms than any reduced mechanism.
Moreover, the contribution of the degradation products having five
carbon atoms in the MCM v3.2 is consistently higher throughout the
model run than in reduced mechanisms (Fig.~\ref{f:carbon}).  Despite
producing less total O$_{x}$, reduced mechanisms produce up to
double the amount of O$_{x}$ from degradation products with one
carbon atom than in the MCM v3.2.  The lower contribution of larger
degradation products indicates that pentane is generally broken down
faster in reduced mechanisms, consistent with the specific example
shown for the breakdown of the pentyl peroxy radical in
Fig.~\ref{f:HC5P_NO}.

The rate of change in reactive carbon during pentane, octane and
toluene degradation was determined by multiplying the rate of each
reaction occurring during pentane, octane and toluene degradation by
its net change in carbon, shown in Fig.~\ref{f:net_carbon_loss}.
Pentane is broken down faster in CBM-IV, CB05, RADM2, RACM and RACM2
by losing reactive carbon more quickly than the MCM v3.2.  MOZART-4
also breaks pentane down into smaller sized products quicker than the
MCM v3.2 as reactions during pentane degradation in MOZART-4 have
organic products whose carbon number is less than the organic
reactant, described in Sect.~\ref{sss:profiles}.  The faster break
down of pentane on the first day limits the amount of reactive carbon
available to produce further O$_{x}$ on subsequent days leading to
lower O$_{x}$ production after the first day in reduced mechanisms.

Figure~\ref{f:TOPP_dailies} showed that octane degradation produces
peak O$_{x}$ on the first day in RADM2, RACM and RACM2 in contrast
to all other mechanisms where peak O$_{x}$ is produced on the
second day.  Octane degradation in RADM2, RACM and RACM2 loses
reactive carbon much faster than any other mechanism on the first day
so that there are not enough degradation products available on the
second day to produce peak O$_{x}$ on the second day
(Fig.~\ref{f:net_carbon_loss}).  This loss of reactive carbon during
alkane degradation leads to the lower accumulated ozone production
from these VOC shown in Table~\ref{t:cumulative_TOPPs_per_C}.

As seen in Fig.~\ref{f:TOPP_dailies}, O$_{x}$ produced during
toluene degradation has a~high spread between the mechanisms.
Figure~\ref{f:carbon} shows differing distributions of the sizes of
the degradation products that produce O$_{x}$.  All reduced
mechanisms omit O$_{x}$ production from at least one degradation
fragment size which produces O$_{x}$ in the MCM v3.2, indicating
that toluene is also broken down more quickly in the reduced
mechanisms than the more explicit mechanisms.  For example, toluene
degradation in RACM2 does not produce O$_{x}$ from degradation
products with six carbons, as is the case in the MCM v3.2.
Figure~\ref{f:net_carbon_loss} shows that all reduced mechanisms lose
reactive carbon during toluene degradation faster than the MCM v3.2.
Thus the degradation of aromatic VOC in reduced mechanisms are unable
to produce similar amounts of O$_{x}$ as the explicit mechanisms.



\conclusions

Tagged Ozone Production Potentials (TOPPs) were used to compare
O$_{x}$ production during VOC degradation in reduced chemical
mechanisms to the near-explicit MCM v3.2.  First day mixing ratios of
\chem{O_3} are similar to the MCM v3.2 for most mechanisms; the
\chem{O_3} mixing ratios in RACM were much lower than the MCM v3.2 due
to a~lack of O$_{x}$ production from the degradation of aromatic
VOC.  Thus, RACM may not be the appropriate chemical mechanism when
simulating atmospheric conditions having a~large fraction of aromatic
VOC.

The lumped intermediate mechanism, CRI v2, produces the most similar
amounts of O$_{x}$ to the MCM v3.2 for each VOC.  The largest
differences between O$_{x}$ production in CRI v2 and MCM v3.2 were
obtained for aromatic VOC, however overall these differences were much
lower than any other reduced mechanism.  Thus, when developing
chemical mechanisms the technique of using lumped intermediate species
whose degradation are based upon more detailed mechanism should be
considered.

Many VOC are broken down into smaller sized degradation products
faster on the first day in reduced mechanisms than the MCM v3.2
leading to lower amounts of larger sized degradation products that can
further degrade and produce O$_{x}$.  Thus, many VOC in reduced
mechanisms produce a~lower maximum of O$_{x}$ and lower total
O$_{x}$ per reactive C by the end of the run than the MCM v3.2.
This lower O$_{x}$ production from many VOC in reduced mechanisms
leads to lower \chem{O_3} mixing ratios compared to the MCM v3.2.

Alkanes produce maximum \chem{O_3} on the second day of simulations
and this maximum is lower in reduced mechanisms than the MCM v3.2 due
to the faster break down of alkanes into smaller sized degradation
products on the first day.  The lower maximum in \chem{O_3} production
during alkane degradation in reduced mechanisms would lead to an
underestimation of the \chem{O_3} levels downwind of VOC emissions,
and an underestimation of the VOC contribution to tropospheric
background \chem{O_3} when using reduced mechanisms in regional or
global modelling studies.

This study has determined the maximum potential of VOC represented in
reduced mechanisms to produce \chem{O_3}, this potential may not be
reached as ambient NO$_{x}$ conditions may not induce
NO$_{x}$-VOC-sensitive chemistry.  Moreover, the maximum potential
of the VOC to produce \chem{O_3} may not be reached when using these
reduced mechanisms in 3-D models due to the influence of additional
processes, such as mixing and meteorology.  Future work shall examine
the extent to which the maximum potential of VOC to produce \chem{O_3}
in reduced chemical mechanisms is reached using ambient NO$_{x}$
conditions and including processes found in 3-D models.



\Supplementary{pdf}



\begin{acknowledgements}
  The authors would like to thank Mark Lawrence and
  Peter~J.~H.~Builtjes for valuable discussions during the preparation
  of this manuscript.
\end{acknowledgements}



\begin{thebibliography}{34}

\bibitem[{Atkinson(2000)}]{Atkinson:2000} Atkinson,~R.: {Atmospheric
chemistry of VOCs and NO$_{x}$}, Atmos. Environ., 34, 2063--2101, 2000.


\bibitem[{Baker et~al.(2008)Baker, Beyersdorf, Doezema, Katzenstein,
    Meinardi, Simpson, Blake, and Rowland}]{Baker:2008} Baker,~A.~K.,
  Beyersdorf,~A.~J., Doezema,~L.~A., Katzenstein,~A., Meinardi,~S.,
  Simpson,~I.~J., Blake,~D.~R., and Rowland,~F.~S.: {Measurements of
    nonmethane hydrocarbons in 28 United States cities},
  Atmos. Environ., 42, 170--182, 2008.


\bibitem[{Bloss et~al.(2005)Bloss, Wagner, Jenkin, Vollamer, Bloss,
    Lee, Heard, Wirtz, Martin-Reviejo, Rea, Wenger, and
    Pilling}]{Bloss:2005} Bloss,~C., Wagner,~V., Jenkin,~M.~E.,
  Volkamer,~R., Bloss,~W.~J., Lee,~J.~D., Heard,~D.~E., Wirtz,~K.,
  Martin-Reviejo,~M., Rea,~G., Wenger,~J.~C., and Pilling,~M.~J.:
  Development of a detailed chemical mechanism (MCMv3.1) for the
  atmospheric oxidation of aromatic hydrocarbons, Atmos. Chem. Phys.,
  5, 641--664,
doi:\href{http://dx.doi.org/10.5194/acp-5-641-2005}{10.5194/acp-5-641-2005},
  2005.



\bibitem[{Butler et~al.(2011)Butler, Lawrence, Taraborrelli, and
    Lelieveld}]{Butler:2011} Butler,~T.~M., Lawrence,~M.~G.,
  Taraborrelli,~D., and Lelieveld,~J.: {Multi-day ozone production
    potential of volatile organic compounds calculated with a tagging
    approach}, Atmos. Environ., 45, 4082--4090, 2011.


\bibitem[{Carter(1994)}]{Carter:1994} Carter,~W.~P.~L.: {Development
    of ozone reactivity scales for volatile organic compounds}, J. Air
  Waste Manage., 44, 881--899, 1994.


\bibitem[{Damian et~al.(2002)Damian, Sandu, Damian, Potra, and
    Carmichael}]{Damian:2002} Damian,~V., Sandu,~A., Damian,~M.,
  Potra,~F., and Carmichael,~G.: {The kinetic preprocessor KPP -- a
    software environment for solving chemical kinetics},
  Comput. Chem. Eng., 26, 1567--1579, 2002.


\bibitem[{Derwent et~al.(1996)Derwent, Jenkin, and
    Saunders}]{Derwent:1996} Derwent,~R.~G., Jenkin,~M.~E., and
  Saunders,~S.~M.: {Photochemical ozone creation potentials for a
    large number of reactive hydrocarbons under European conditions},
  Atmos. Environ., 30, 181--199, 1996.


\bibitem[{Derwent et~al.(1998)Derwent, Jenkin, Saunders, and
    Pilling}]{Derwent:1998} Derwent,~R.~G., Jenkin,~M.~E.,
  Saunders,~S.~M., and Pilling,~M.~J.: {Photochemical ozone creation
    potentials for organic compounds in Northwest Europe calculated
    with a master chemical mechanism}, Atmos. Environ., 32,
  2429--2441, 1998.


\bibitem[{Derwent et~al.(2010)Derwent, Jenkin, Pilling, Carter, and
    Kaduwela}]{Derwent:2010} Derwent,~R.~G., Jenkin,~M.~E.,
  Pilling,~M.~J., Carter,~W.~P.~L., and Kaduwela,~A.: {Reactivity
    scales as comparative tools for chemical mechanisms}, J. Air
  Waste Manage., 60, 914--924, 2010.


\bibitem[{Dodge(2000)}]{Dodge:2000} Dodge,~M.: {Chemical oxidant
    mechanisms for air quality modeling: critical review},
  Atmos. Environ., 34, 2103--2130, 2000.


\bibitem[{Dunker et~al.(1984)Dunker, Kumar, and Berzins}]{Dunker:1984}
  Dunker,~A.~M., Kumar,~S., and Berzins,~P.~H.: {A comparison of
    chemical mechanisms used in atmospheric models}, Atmos. Environ.,
  18, 311--321, 1984.


\bibitem[{EEA(2014)}]{AQEU:2014} EEA: {Air quality in Europe -- 2014
    report}, Tech. Rep. 5/2014, European Environmental Agency, Publications Office of the European
    Union,
  doi:\href{http://dx.doi.org/10.2800/22847}{10.2800/22847}, 2014.


\bibitem[{Emmerson and Evans(2009)}]{Emmerson:2009}
  Emmerson,~K.~M. and Evans,~M.~J.: Comparison of tropospheric
  gas-phase chemistry schemes for use within global models,
  Atmos. Chem. Phys., 9, 1831--1845,
doi:\href{http://dx.doi.org/10.5194/acp-9-1831-2009}{10.5194/acp-9-1831-2009},
  2009.



\bibitem[{Emmons et~al.(2010)Emmons, Walters, Hess, Lamarque, Pfister,
    Fillmore, Granier, Guenther, Kinnison, Laepple, Orlando, Tie,
    Tyndall, Wiedinmyer, Baughcum, and Kloster}]{Emmons:2010}
  Emmons,~L.~K., Walters,~S., Hess,~P.~G., Lamarque,~J.-F.,
  Pfister,~G.~G., Fillmore,~D., Granier,~C., Guenther,~A.,
  Kinnison,~D., Laepple,~T., Orlando,~J., Tie,~X., Tyndall,~G.,
  Wiedinmyer,~C., Baughcum,~S.~L., and Kloster,~S.: Description and
  evaluation of the Model for Ozone and Related chemical Tracers,
  version 4 (MOZART-4), Geosci. Model Dev., 3, 43--67,
  doi:\href{http://dx.doi.org/10.5194/gmd-3-43-2010}{10.5194/gmd-3-43-2010}, 2010.



\bibitem[{Gery et~al.(1989)Gery, Whitten, Killus, and
    Dodge}]{Gery:1989} Gery,~M.~W., Whitten,~G.~Z., Killus,~J.~P., and
  Dodge,~M.~C.: {A photochemical kinetics mechanism for urban and
    regional scale computer modeling}, J. Geophys. Res., 94,
  12925--12956, 1989.


\bibitem[{Goliff et~al.(2013)Goliff, Stockwell, and
    Lawson}]{Goliff:2013} Goliff,~W.~S., Stockwell,~W.~R., and
  Lawson,~C.~V.: {The regional atmospheric chemistry mechanism,
    version 2}, Atmos. Environ., 68, 174--185, 2013.


\bibitem[{Harwood et~al.(2003)Harwood, Roberts, Frost, Ravishankara,
    and Burkholder}]{Harwood:2003} Harwood,~M., Roberts,~J.,
  Frost,~G., Ravishankara,~A., and Burkholder,~J.: {Photochemical
    studies of \chem{CH_3C(O)OONO_2} (PAN) and
    \chem{CH_3CH_2C(O)OONO_2} (PPN): \chem{NO_3} quantum yields},
  J. Phys. Chem. A, 107, 1148--1154, 2003.


\bibitem[{Hogo and Gery(1989)}]{Hogo:1989} Hogo,~H. and Gery,~M.:
  User's guide for executing OZIPM-4 (Ozone Isopleth Plotting with
  Optional Mechanisms, Version 4) with CBM-IV (Carbon-Bond
  Mechanisms-IV) or optional mechanisms. Volume 1. Description of the
  ozone isopleth plotting package. Version 4, Tech. rep.,~US
  Environmental Protection Agency, Durham, North Carolina, USA, 1989.


\bibitem[{HTAP(2010)}]{HTAP:2010} HTAP: Hemispheric Transport of Air
    Pollution 2010, Part A: Ozone and Particulate Matter, Air
    Pollution Studies No.17, Geneva, Switzerland, 2010.


\bibitem[{Jenkin and Clemitshaw(2000)}]{Jenkin:2000} Jenkin,~M.~E. and
  Clemitshaw,~K.~C.: {Ozone and other secondary photochemical
    pollutants: chemical processes governing their formation in the
    planetary boundary layer}, Atmos. Environ., 34, 2499--2527,
  2000.


\bibitem[{Jenkin et~al.(1997)Jenkin, Saunders, and
    Pilling}]{Jenkin:1997} Jenkin,~M.~E., Saunders,~S.~M., and
  Pilling,~M.~J.: {The tropospheric degradation of volatile organic
    compounds: a protocol for mechanism development}, Atmos. Environ.,
  31, 81--104, 1997.


\bibitem[{Jenkin et~al.(2003)Jenkin, Saunders, Wagner, and
    Pilling}]{Jenkin:2003} Jenkin,~M.~E., Saunders,~S.~M., Wagner,~V.,
  and Pilling,~M.~J.: Protocol for the development of the Master
  Chemical Mechanism, MCM v3 (Part B): tropospheric degradation of
  aromatic volatile organic compounds, Atmos. Chem. Phys., 3,
  181--193,
doi:\href{http://dx.doi.org/10.5194/acp-3-181-2003}{10.5194/acp-3-181-2003},
  2003.



\bibitem[{Jenkin et~al.(2008)Jenkin, Watson, Utembe, and
    Shallcross}]{Jenkin:2008} Jenkin,~M.~E., Watson,~L.~A.,
  Utembe,~S.~R., and Shallcross,~D.~E.: {A Common Representative
    Intermediates (CRI) mechanism for VOC degradation. Part 1: Gas
    phase mechanism development}, Atmos. Environ., 42, 7185--7195,
  2008.


\bibitem[{Kleinman(1991)}]{Kleinman:1991} Kleinman,~L.~I.: {Seasonal
    dependence of boundary layer peroxide concentration: the low and
    high NO$_{x}$ regimes}, J. Geophys. Res., 96, 20721--20733,
  1991.


\bibitem[{Kleinman(1994)}]{Kleinman:1994} Kleinman,~L.~I.: {Low and
    high NO$_{x}$ tropospheric photochemistry}, J. Geophys. Res.,
  99, 16831--16838, 1994.


\bibitem[{Kuhn et~al.(1998)Kuhn, Builtjes, Poppe, Simpson, Stockwell,
    Andersson-Sk{\"o}ld, Baart, Das, Fiedler, Hov, Kirchner, Makar,
    Milford, Roemer, Ruhnke, Strand, Vogel, and Vogel}]{Kuhn:1998}
  Kuhn,~M., Builtjes,~P.~J.~H., Poppe,~D., Simpson,~D.,
  Stockwell,~W.~R., Andersson-Sk{\"o}ld,~Y., Baart,~A., Das,~M.,
  Fiedler,~F., Hov,~{\O}., Kirchner,~F., Makar,~P.~A., Milford,~J.~B.,
  Roemer,~M.~G.~M., Ruhnke,~R., Strand,~A., Vogel,~B., and Vogel,~H.:
  {Intercomparison of the gas-phase chemistry in several chemistry and
    transport models}, Atmos. Environ., 32, 693--709, 1998.


\bibitem[{Rickard et~al.(2015)Rickard, Young, and Pascoe}]{MCM_Site}
  Rickard,~A., Young,~J., and Pascoe,~S.: {The Master Chemical
    Mechanism Version MCM v3.2}, available at:
  \url{http://mcm.leeds.ac.uk/MCMv3.2/} (last access: 25 March 2015),
  2015.


\bibitem[{Sander et~al.(2005)Sander, Kerkweg, J{\"o}ckel, and
    Lelieveld}]{Sander:2005} Sander,~R., Kerkweg,~A., J\"{o}ckel,~P.,
  and Lelieveld,~J.: Technical note: The new comprehensive atmospheric
  chemistry module MECCA, Atmos. Chem. Phys., 5, 445--450,
  doi:\href{http://dx.doi.org/10.5194/acp-5-445-2005}{10.5194/acp-5-445-2005}, 2005.



\bibitem[{Saunders et~al.(2003)Saunders, Jenkin, Derwent, and
    Pilling}]{Saunders:2003} Saunders,~S.~M., Jenkin,~M.~E.,
  Derwent,~R.~G., and Pilling,~M.~J.: Protocol for the development of
  the Master Chemical Mechanism, MCM v3 (Part A): tropospheric
  degradation of non-aromatic volatile organic compounds,
  Atmos. Chem. Phys., 3, 161--180,
doi:\href{http://dx.doi.org/10.5194/acp-3-161-2003}{10.5194/acp-3-161-2003},
  2003.



\bibitem[{Sillman(1999)}]{Sillman:1999} Sillman,~S.: {The relation
    between ozone, NO$_{x}$ and hydrocarbons in urban and polluted
    rural environments}, Atmos. Environ., 33, 1821--1845,
  1999.


\bibitem[{Stevenson et~al.(2013)Stevenson, Young, Naik, Lamarque,
    Shindell, Voulgarakis, Skeie, Dalsoren, Myhre, Berntsen, Folberth,
    Rumbold, Collins, MacKenzie, Doherty, Zeng, van Noije, Strunk,
    Bergmann, Cameron-Smith, Plummer, Strode, Horowitz, Lee, Szopa,
    Sudo, Nagashima, Josse, Cionni, Righi, Eyring, Conley, Bowman,
    Wild, and Archibald}]{Stevenson:2013} Stevenson,~D.~S.,
  Young,~P.~J., Naik,~V., Lamarque,~J.-F., Shindell,~D.~T.,
  Voulgarakis,~A., Skeie,~R.~B., Dalsoren,~S.~B., Myhre,~G.,
  Berntsen,~T.~K., Folberth,~G.~A., Rumbold,~S.~T., Collins,~W.~J.,
  MacKenzie,~I.~A., Doherty,~R.~M., Zeng,~G., van~Noije,~T.~P.~C.,
  Strunk,~A., Bergmann,~D., Cameron-Smith,~P., Plummer,~D.~A.,
  Strode,~S.~A., Horowitz,~L., Lee,~Y.~H., Szopa,~S., Sudo,~K.,
  Nagashima,~T., Josse,~B., Cionni,~I., Righi,~M., Eyring,~V.,
  Conley,~A., Bowman,~K.~W., Wild,~O., and Archibald,~A.: Tropospheric
  ozone changes, radiative forcing and attribution to emissions in the
  Atmospheric Chemistry and Climate Model Intercomparison Project
  (ACCMIP), Atmos. Chem. Phys., 13, 3063--3085,
  doi:\href{http://dx.doi.org/10.5194/acp-13-3063-2013}{10.5194/acp-13-3063-2013}, 2013.



\bibitem[{Stockwell et~al.(1990)Stockwell, Middleton, Chang, and
    Tang}]{Stockwell:1990} Stockwell,~W.~R., Middleton,~P.,
  Chang,~J.~S., and Tang,~X.: {The second generation regional acid
    deposition model chemical mechanism for regional air quality
    modeling}, J. Geophys. Res., 95, 16343--16367, 1990.


\bibitem[{Stockwell et~al.(1997)Stockwell, Kirchner, Kuhn, and
    Seefeld}]{Stockwell:1997} Stockwell,~W.~R., Kirchner,~F.,
  Kuhn,~M., and Seefeld,~S.: {A new mechanism for regional atmospheric
    chemistry modeling}, J. Geophys. Res.-Atmos., 102, 25847--25879,
  1997.


\bibitem[{Yarwood et~al.(2005)Yarwood, Rao, Yocke, and Whitten}]{Yarwood:2005}
Yarwood,~G., Rao,~S., Yocke,~M., and Whitten,~G.~Z.: Updates to the Carbon
Bond Chemical Mechanism: CB05, Tech. rep.,~US Environmental Protection
Agency, Novato, California, USA, 2005.

\end{thebibliography}


\begin{table}[t]
    \caption{The chemical mechanisms used in the study, MCM v3.2 is the reference mechanism.}
\scalebox{.88}[.88]{\begin{tabular}{lrrll}
        \tophline
        {Chemical} &{Number of} &{Number of} &{Type of} &\multirow{2}{*}{{Reference}} \\
        {Mechanism} &{Organic Species} &{Organic Reactions} &{Lumping}
        &\\
\middlehline
        MCM v3.2 &$5708$ &$16\,349$ &No lumping &\citet{MCM_Site} \\
        {MCM v3.1} &{$4351$} &{$12\,691$} &{No lumping} &\citet{Jenkin:1997} \\
        &&&&\citet{Saunders:2003} \\
        &&&&\citet{Jenkin:2003} \\
        &&&&\citet{Bloss:2005} \\
        CRI v2 &$411$ &$1145$ &Lumped intermediates &\citet{Jenkin:2008} \\
        MOZART-4 &$69$ &$145$ &Lumped molecule &\citet{Emmons:2010} \\
        RADM2 &$44$ &$103$ &Lumped molecule &\citet{Stockwell:1990} \\
        RACM &$58$ &$193$ &Lumped molecule &\citet{Stockwell:1997} \\
        RACM2 &$99$ &$315$ &Lumped molecule &\citet{Goliff:2013} \\
        CBM-IV &$20$ &$45$ &Lumped structure &\citet{Gery:1989} \\
        CB05 &$37$ &$99$ &Lumped structure &\citet{Yarwood:2005} \\
        \bottomhline
    \end{tabular}}
\label{t:mechanisms}
\end{table}


\begin{table}[t]
  \caption{VOC present in Los Angeles, mixing ratios are taken from \citet{Baker:2008} and their representation in each chemical mechanism. The representation of the VOC in each mechanism is based upon the recommendations of the literature for each mechanism (Table~\ref{t:mechanisms}).}
  \scalebox{.72}[.72]{\begin{tabular}{lllllllll}
      \tophline
      \multirow{2}{*}{{NMVOC}} &{Mixing} &{MCM v3.1, v3.2,} &\multirow{2}{*}{{MOZART-4}} &\multirow{2}{*}{{RADM2}} &\multirow{2}{*}{{RACM}} &\multirow{2}{*}{{RACM2}} &\multirow{2}{*}{{CBM-IV}} &\multirow{2}{*}{{CB05}}\\
      &{Ratio (pptv)} &{CRI v2} &&&&&&\\
      \middlehline \multicolumn{9}{c}{{Alkanes}}  \\ \middlehline
      Ethane &$6610$ &C2H6 &C2H6 &ETH &ETH &ETH &$0.4$ PAR &ETHA \\
      Propane  &$6050$ &C3H8 &C3H8 &HC3 &HC3 &HC3 &$1.5$ PAR &$1.5$ PAR \\
      Butane &$2340$ &NC4H10 &BIGALK &HC3 &HC3 &HC3 &$4$ PAR &$4$ PAR \\
      $2$-Methylpropane &$1240$ &IC4H10 &BIGALK &HC3 &HC3 &HC3 &$4$ PAR &$4$ PAR \\
      Pentane &$1200$ &NC5H12 &BIGALK &HC5 &HC5 &HC5 &$5$ PAR &$5$ PAR \\
      $2$-Methylbutane &$2790$ &IC5H12 &BIGALK &HC5 &HC5 &HC5 &$5$ PAR &$5$ PAR \\
      Hexane &$390$ &NC6H14 &BIGALK &HC5 &HC5 &HC5 &$6$ PAR &$6$ PAR \\
      Heptane &$160$ &NC7H16 &BIGALK &HC5 &HC5 &HC5 &$7$ PAR &$7$ PAR \\
      Octane &$80$ &NC8H18 &BIGALK &HC8 &HC8 &HC8 &$8$ PAR &$8$ PAR
      \\
      \middlehline
      \multicolumn{9}{c}{{Alkenes}} \\
      \middlehline
      Ethene &$2430$ &C2H4 &C2H4 &OL2 &ETE &ETE &ETH &ETH \\
      Propene &$490$ &C3H6 &C3H6 &OLT &OLT &OLT &OLE\,$+$\,PAR &OLE\,$+$\,PAR \\
      Butene &$65$ &BUT1ENE &BIGENE &OLT &OLT &OLT &OLE\,$+$\,$2$ PAR &OLE\,$+$\,$2$ PAR \\
      \multirow{2}{*}{$2$-Methylpropene} &\multirow{2}{*}{$130$}
      &\multirow{2}{*}{MEPROPENE} &\multirow{2}{*}{BIGENE}
      &\multirow{2}{*}{OLI} &\multirow{2}{*}{OLI}
      &\multirow{2}{*}{OLI} &PAR\,$+$\,FORM &FORM\,$+$\\
      &&&&&&&$+$\,ALD2 &$3$ PAR \\        Isoprene &$270$ &C5H8 &ISOP &ISO
      &ISO &ISO &ISOP &ISOP \\
      \middlehline
      \multicolumn{9}{c}{{Aromatics}} \\
      \middlehline
      Benzene &$480$ &BENZENE &TOLUENE &TOL &TOL &BEN &PAR &PAR \\
      Toluene &$1380$ &TOLUENE &TOLUENE &TOL &TOL &TOL &TOL &TOL \\
      m-Xylene &$410$ &MXYL &TOLUENE &XYL &XYL &XYM &XYL &XYL \\
      p-Xylene &$210$ &PXYL &TOLUENE &XYL &XYL &XYP &XYL &XYL \\
      o-Xylene &$200$ &OXYL &TOLUENE &XYL &XYL &XYO &XYL &XYL \\
      Ethylbenzene &$210$ &EBENZ &TOLUENE &TOL &TOL &TOL
      &TOL\,$+$\,PAR &TOL\,$+$\,PAR \\
      \bottomhline
    \end{tabular}}
    \label{t:initial_conditions}
\end{table}


\begin{table}
  \caption{Cumulative TOPP values at the end of the model run for all VOCs with each mechanism, normalised by the number of C atoms in each VOC.}
  \scalebox{.77}[.77]{\begin{tabular}{llllllllll}
      \tophline
      {NMVOC} &{MCM v3.2} &{MCM v3.1} &{CRI v2} &{MOZART-4} &{RADM2} &{RACM} &{RACM2} &{CBM-IV} &{CB05} \\
      \middlehline
\multicolumn{10}{c}{{Alkanes}}  \\
\middlehline
      Ethane &$0.9$ &$1.0$ &$0.9$ &$0.9$ &$1.0$ &$1.0$ &$0.9$ &$0.3$ &$0.9$ \\
      Propane &$1.1$ &$1.2$ &$1.2$ &$1.1$ &$1.8$ &$1.8$ &$1.4$ &$0.9$ &$1.0$ \\
      Butane &$2.0$ &$2.0$ &$2.0$ &$1.7$ &$1.8$ &$1.8$ &$1.4$ &$1.7$ &$2.1$ \\
      $2$-Methylpropane &$1.3$ &$1.3$ &$1.3$ &$1.7$ &$1.8$ &$1.8$ &$1.4$ &$1.7$ &$2.1$ \\
      Pentane &$2.1$ &$2.1$ &$2.2$ &$1.7$ &$1.5$ &$1.6$ &$1.1$ &$1.7$ &$2.1$ \\
      $2$-Methylbutane &$1.6$ &$1.6$ &$1.5$ &$1.7$ &$1.5$ &$1.6$ &$1.1$ &$1.7$ &$2.1$ \\
      Hexane &$2.1$ &$2.1$ &$2.2$ &$1.7$ &$1.5$ &$1.6$ &$1.1$ &$1.7$ &$2.1$ \\
      Heptane &$2.0$ &$2.1$ &$2.2$ &$1.7$ &$1.5$ &$1.6$ &$1.1$ &$1.7$ &$2.1$ \\
      Octane &$2.0$ &$2.0$ &$2.2$ &$1.7$ &$1.2$ &$1.0$ &$1.0$ &$1.7$
      &$2.1$ \\
\middlehline
      \multicolumn{10}{c}{{Alkenes}} \\
\middlehline
      Ethene &$1.9$ &$1.9$ &$1.9$ &$1.4$ &$2.0$ &$2.0$ &$2.2$ &$1.9$ &$2.2$ \\
      Propene &$1.9$ &$2.0$ &$1.9$ &$1.7$ &$1.5$ &$1.6$ &$1.5$ &$1.2$ &$1.4$ \\
      Butene &$1.9$ &$2.0$ &$2.0$ &$1.5$ &$1.5$ &$1.6$ &$1.5$ &$0.8$ &$0.9$ \\
      $2$-Methylpropene &$1.1$ &$1.2$ &$1.2$ &$1.5$ &$1.1$ &$1.5$ &$1.6$ &$0.5$ &$0.5$ \\
      Isoprene &$1.8$ &$1.8$ &$1.8$ &$1.3$ &$1.2$ &$1.6$ &$1.7$ &$1.9$
      &$2.1$ \\
\middlehline
      \multicolumn{10}{c}{{Aromatics}} \\
\middlehline
      Benzene &$0.8$ &$0.8$ &$1.1$ &$0.6$ &$0.9$ &$0.6$ &$0.9$ &$0.3$ &$0.3$ \\
      Toluene &$1.3$ &$1.3$ &$1.5$ &$0.6$ &$0.9$ &$0.6$ &$1.0$ &$0.3$ &$0.3$ \\
      m-Xylene &$1.5$ &$1.5$ &$1.6$ &$0.6$ &$0.9$ &$0.6$ &$1.7$ &$0.9$ &$1.0$ \\
      p-Xylene &$1.5$ &$1.5$ &$1.6$ &$0.6$ &$0.9$ &$0.6$ &$1.7$ &$0.9$ &$1.0$ \\
      o-Xylene &$1.5$ &$1.5$ &$1.6$ &$0.6$ &$0.9$ &$0.6$ &$1.7$ &$0.9$ &$1.0$ \\
      Ethylbenzene &$1.3$ &$1.4$ &$1.5$ &$0.6$ &$0.9$ &$0.6$ &$1.0$ &$0.2$ &$0.3$ \\
      \bottomhline
    \end{tabular}}
    \label{t:cumulative_TOPPs_per_C}
\end{table}



\begin{figure}[t]
    \includegraphics[width=120mm]{acp-2015-279-discussions-f01.pdf}
    \caption{Time series of \chem{O_3} mixing ratios obtained using
      each mechanism.}
    \label{f:time_series}
\end{figure}
\begin{figure}
    \includegraphics[height=127mm]{acp-2015-279-discussions-f02.pdf}
    \caption{Day-time O$_{x}$ production budgets in each mechanism
      allocated to individual VOC.}
    \label{f:Ox_tagged_budgets}
\end{figure}
\begin{figure}[t]
  \includegraphics[width=120mm]{acp-2015-279-discussions-f03.pdf}
  \caption{TOPP value time series using each mechanism for each VOC.}
  \label{f:TOPP_dailies}
\end{figure}
\begin{figure}[t]
  \includegraphics[height=120mm]{acp-2015-279-discussions-f04.pdf}
  \caption{The first day TOPP values for each VOC calculated using MCM
    v3.2 and the corresponding values in each mechanism. The root mean
    square error (RMSE) of each set of TOPP values is also
    displayed. The black line represents the $1:1$ line.}
    \label{f:first_day}
\end{figure}
\begin{figure}[t]
    \includegraphics[height=120mm]{acp-2015-279-discussions-f05.pdf}
    \caption{Day-time O$_{x}$ production and loss budgets allocated
      to the responsible reactions during toluene degradation in all
      mechanisms. These reactions are presented using the species
      defined in each mechanism Table~\ref{t:mechanisms}.}
    \label{f:toluene_Ox}
\end{figure}
\begin{figure}[t]
    \includegraphics[width=120mm]{acp-2015-279-discussions-f06.pdf}
    \caption{The distribution of reactive carbon in the products of
      the reaction between \chem{NO} and the pentyl peroxy radical in
      lumped molecule mechanisms compared to the MCM. The black dot
      represents the reactive carbon of the pentyl peroxy radical.}
    \label{f:HC5P_NO}
\end{figure}
\begin{figure}[t]
    \includegraphics[height=124mm]{acp-2015-279-discussions-f07.pdf}
    \caption{Day-time O$_{x}$ production during pentane and toluene
      degradation is attributed to the number of carbon atoms of the
      degradation products for each mechanism.}
    \label{f:carbon}
\end{figure}
\begin{figure}[t]
    \includegraphics[width=120mm]{acp-2015-279-discussions-f08.pdf}
    \caption{Daily rate of change in reactive carbon during pentane,
      octane and toluene degradation. Octane is represented by the
      five carbon species, BIGALK, in MOZART-4.}
    \label{f:net_carbon_loss}
\end{figure}

\end{document}
