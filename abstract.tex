Ground-level ozone is a secondary pollutant produced photochemically from reactions of nitrogen oxides with peroxy radicals produced during volatile organic compound (VOC) degradation. 
Chemical transport models use simplified representations of this complex gas-phase chemistry to predict \ce{O3} levels and inform emission control strategies. 
Thus accurate representation of \ce{O3} production chemistry is vital for effective predictions.
VOC degradation chemistry in simplified mechanisms were compared to that in the near-explicit MCM mechanism. 
This chemistry was compared by ``tagging'' all organic degradation products over multi-day runs and calculating the Tagged Ozone Production Potential (TOPP). 
First day TOPP values were similar for most VOCs, larger discrepancies arose over the course of the model run. 
Differences in net radical to \ce{NO_x} yields show that different atmospheric regimes are represented in different mechanisms.
Reduced mechanisms were unable to attain the \ce{O_x} levels in near-explicit mechanisms due to increased reactive carbon loss.
