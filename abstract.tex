Ground-level ozone is a secondary pollutant produced photochemically from reactions of nitrogen oxides with peroxy radicals produced during volatile organic compound (VOC) degradation. 
Chemical transport models use simplified representations of this complex gas-phase chemistry to predict \ce{O3} levels and inform emission control strategies. 
Accurate representation of \ce{O3} production chemistry is vital for effective predictions.
\ce{O_x} production during VOC degradation in simplified mechanisms is compared to that in the near-explicit MCM mechanism. 
This chemistry is compared by ``tagging'' all organic degradation products over multi-day runs and calculating the Tagged Ozone Production Potential (TOPP). 
First day TOPP values were similar for most VOCs, larger discrepancies arose over the course of the model run. 
Using lumped degradation products produces more comparable \ce{O_x} to the MCM than lumping the initial VOC.
VOC in reduced mechanisms produce more \ce{O_x} from smaller degradation products than in the MCM due to either faster loss of reactive carbon or increased acetaldehyde production.
