Ground-level ozone is a secondary pollutant produced photochemically from reactions of \ce{NO_x} with peroxy radicals produced during VOC degradation. 
Chemical transport models use simplified representations of this complex gas-phase chemistry to predict \ce{O3} levels and inform emission control strategies. 
Accurate representation of \ce{O3} production chemistry is vital for effective predictions.
VOC degradation chemistry in simplified mechanisms is compared to that in the near-explicit MCM mechanism by ``tagging'' all organic degradation products over multi-day runs and calculating the Tagged Ozone Production Potential (TOPP). 
First day TOPP values are similar for most VOCs, larger discrepancies arise over the course of the model run. 
VOC in simplified mechanisms produce more \ce{O_x} from smaller degradation products than the MCM due to a faster loss of reactive carbon or lower \ce{O_x} production efficiency.
Simplified mechanisms that lump VOC degradation products instead of the initial VOC produce comparable \ce{O_x} amounts to the MCM.
