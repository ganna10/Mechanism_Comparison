Ground-level ozone is a secondary pollutant produced photochemically from the reactions of nitrogen oxides with peroxy radicals produced during volatile organic compound (VOC) degradation. 
Chemical transport models use simplified representations of this complex gas-phase chemistry to predict \ce{O3} levels and inform emission control strategies. 
Thus an accurate representation of \ce{O3} production chemistry is vital for effective predictions, VOC degradation chemistry in MCM v3.1, CRI v2, MOZART-4, RADM2, RACM, RACM2, CBM-IV and CB05 were compared to that of the near-explicit MCM v3.2 mechanism. 
This chemistry was compared by tagging all organic degradation products over multi-day runs and calculating the Tagged Ozone Production Potential (TOPP) for each VOC. 
First day TOPP values were similar for most VOCs, larger discrepancies arose over the model run. 
\ce{O_x} production was attributed to the number of carbon atoms in each degradation product showing that reduced mechanisms break down the emitted VOC quicker than more explicit mechanisms. 
Analysing the radical and PAN net production budgets illustrated that inclusion of production sources not included in the MCM v3.2 impacts \ce{O_x} production.
