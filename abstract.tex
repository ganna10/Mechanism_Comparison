Ground-level ozone is a secondary pollutant produced photochemically from reactions of \ce{NO_x} with peroxy radicals produced during VOC degradation. 
Chemical transport models use simplified representations of this complex gas-phase chemistry to predict \ce{O3} levels and inform emission control strategies. 
Accurate representation of \ce{O3} production chemistry is vital for effective predictions.
VOC degradation chemistry in simplified mechanisms is compared to that in the near-explicit MCM mechanism by ``tagging'' all organic degradation products over multi-day runs and calculating the Tagged Ozone Production Potential (TOPP) for a selection of VOC representative of urban airmasses.
First day TOPP values are similar for most VOCs, larger discrepancies arise over the course of the model run.
Simplified mechanisms that lump VOC degradation products instead of the initial VOC produce comparable amounts of \ce{O_x} to the MCM.
VOC in simplified mechanisms produce more \ce{O_x} from smaller degradation products than the MCM due to a faster break down of the VOC on the first day which limits the amount of \ce{O_x} produced on subsequent days.
In particular, \ce{O_x} production after the first day during alkane degradation is underestimated in all simplified mechanisms impacting predictions of \ce{O3} levels downwind of their emission source.
