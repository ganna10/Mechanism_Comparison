Ground-level ozone is a secondary pollutant produced photochemically from reactions of \ce{NO_x} with peroxy radicals produced during VOC degradation. 
Chemical transport models use simplified representations of this complex gas-phase chemistry to predict \ce{O3} levels and inform emission control strategies. 
Accurate representation of \ce{O3} production chemistry is vital for effective predictions.
In this study, VOC degradation chemistry in simplified mechanisms is compared to that in the near-explicit MCM mechanism using a boxmodel and by ``tagging'' all organic degradation products over multi-day runs, thus calculating the Tagged Ozone Production Potential (TOPP) for a selection of VOC representative of urban airmasses.
Simplified mechanisms that aggregate VOC degradation products instead of aggregating emitted VOC produce comparable amounts of \ce{O3} from VOC degradation to the MCM.
First day TOPP values are similar across mechanisms for most VOC, with larger discrepancies arising over the course of the model run.
Aromatic and unsaturated aliphatic VOC have largest inter-mechanisms differences on the first day, while alkanes show largest differences on the second day.
Simplified mechanisms break down VOC into smaller sized degradation products on the first day faster than the MCM impacting the total amount of \ce{O3} produced on subsequent days due to secondary chemistry.
